\nonstopmode{}
\documentclass[a4paper]{book}
\usepackage[times,inconsolata,hyper]{Rd}
\usepackage{makeidx}
\usepackage[utf8]{inputenc} % @SET ENCODING@
% \usepackage{graphicx} % @USE GRAPHICX@
\makeindex{}
\begin{document}
\chapter*{}
\begin{center}
{\textbf{\huge Package `WaverideR'}}
\par\bigskip{\large \today}
\end{center}
\inputencoding{utf8}
\ifthenelse{\boolean{Rd@use@hyper}}{\hypersetup{pdftitle = {WaverideR: Extracting Signals from Wavelet Spectra}}}{}
\ifthenelse{\boolean{Rd@use@hyper}}{\hypersetup{pdfauthor = {Michiel Arts}}}{}
\begin{description}
\raggedright{}
\item[Type]\AsIs{Package}
\item[Title]\AsIs{Extracting Signals from Wavelet Spectra}
\item[Version]\AsIs{0.3.2}
\item[Maintainer]\AsIs{Michiel Arts }\email{michiel.arts@stratigraphy.eu}\AsIs{}
\item[Depends]\AsIs{R (>= 3.5.0)}
\item[Imports]\AsIs{DecomposeR, DescTools, Hmisc, Matrix,utils,colorednoise, doSNOW, fANCOVA, foreach,stats,tcltk,matrixStats,reshape2,truncnorm,grDevices,graphics,parallel,astrochron,biwavelet,WaveletComp,RColorBrewer,colorRamps,viridis,magick,rlist}
\item[Description]\AsIs{The continuous wavelet transform enables the observation of transient/non-stationary cyclicity in time-series. The goal of cyclostratigraphic studies is to define frequency/period in the depth/time domain. By conducting the continuous wavelet transform on cyclostratigraphic data series one can observe and extract cyclic signals/signatures from signals. These results can then be visualized and interpreted enabling one to identify/interpret cyclicity in the geological record, which can be used to construct astrochronological age-models and identify and interpret cyclicity in past and present climate systems. The 'WaverideR' R package builds upon existing literature and existing codebase. The list of articles which are relevant can be grouped in four subjects; cyclostratigraphic data analysis,example data sets,the (continuous) wavelet transform and astronomical solutions. References for the cyclostratigraphic data analysis articles are: Stephen Meyers (2019) <}\Rhref{https://doi.org/10.1016/j.earscirev.2018.11.015}{doi:10.1016/j.earscirev.2018.11.015}\AsIs{>. Mingsong Li, Linda Hinnov, Lee Kump (2019) <}\Rhref{https://doi.org/10.1016/j.cageo.2019.02.011}{doi:10.1016/j.cageo.2019.02.011}\AsIs{> Stephen Meyers (2012)<}\Rhref{https://doi.org/10.1029/2012PA002307}{doi:10.1029/2012PA002307}\AsIs{> Mingsong Li, Lee R. Kump, Linda A. Hinnov, Michael E. Mann (2018) <}\Rhref{https://doi.org/10.1016/j.epsl.2018.08.041}{doi:10.1016/j.epsl.2018.08.041}\AsIs{>. Wouters, S., Crucifix, M., Sinnesael, M., Da Silva, A.C., Zeeden, C., Zivanovic, M., Boulvain, F., Devleeschouwer, X. (2022) <}\Rhref{https://doi.org/10.1016/j.earscirev.2021.103894}{doi:10.1016/j.earscirev.2021.103894}\AsIs{>. Wouters, S., Da Silva, A.-C., Boulvain, F., and Devleeschouwer, X. (2021) <}\Rhref{https://doi.org/10.32614/RJ-2021-039}{doi:10.32614/RJ-2021-039}\AsIs{>. Huang, Norden E., Zhaohua Wu, Steven R. Long, Kenneth C. Arnold, Xianyao Chen, and Karin Blank  (2009) <}\Rhref{https://doi.org/10.1142/S1793536909000096}{doi:10.1142/S1793536909000096}\AsIs{>. Cleveland, W. S. (1979)<}\Rhref{https://doi.org/10.1080/01621459.1979.10481038}{doi:10.1080/01621459.1979.10481038}\AsIs{> Hurvich, C.M., Simonoff, J.S., and Tsai, C.L. (1998) <}\Rhref{https://doi.org/10.1111/1467-9868.00125}{doi:10.1111/1467-9868.00125}\AsIs{>, Golub, G., Heath, M. and Wahba, G. (1979) <}\Rhref{https://doi.org/10.2307/1268518}{doi:10.2307/1268518}\AsIs{>. References for the example data articles are: Damien Pas, Linda Hinnov, James E. (Jed) Day, Kenneth Kodama, Matthias Sinnesael, Wei Liu (2018) <}\Rhref{https://doi.org/10.1016/j.epsl.2018.02.010}{doi:10.1016/j.epsl.2018.02.010}\AsIs{>. Steinhilber, Friedhelm, Abreu, Jacksiel, Beer, Juerg , Brunner, Irene, Christl, Marcus, Fischer, Hubertus, HeikkilA, U., Kubik,  Peter, Mann, Mathias, Mccracken, K. , Miller, Heinrich, Miyahara, Hiroko, Oerter, Hans , Wilhelms, Frank. (2012 <}\Rhref{https://doi.org/10.1073/pnas.1118965109}{doi:10.1073/pnas.1118965109}\AsIs{>. Christian Zeeden, Frederik Hilgen, Thomas Westerhold, Lucas Lourens, Ursula Röhl, Torsten Bickert (2013) <}\Rhref{https://doi.org/10.1016/j.palaeo.2012.11.009}{doi:10.1016/j.palaeo.2012.11.009}\AsIs{>. References for the (continuous) wavelet transform articles are: Morlet, Jean, Georges Arens, Eliane Fourgeau, and Dominique Glard  (1982a) <}\Rhref{https://doi.org/10.1190/1.1441328}{doi:10.1190/1.1441328}\AsIs{>. J. Morlet, G. Arens, E. Fourgeau, D. Giard (1982b) <}\Rhref{https://doi.org/10.1190/1.1441329}{doi:10.1190/1.1441329}\AsIs{>. Torrence, C., and G. P. Compo (1998)<}\url{https://paos.colorado.edu/research/wavelets/bams_79_01_0061.pdf}\AsIs{>, Gouhier TC, Grinsted A, Simko V (2021) <}\url{https://github.com/tgouhier/biwavelet}\AsIs{>. Angi Roesch and Harald Schmidbauer (2018) <}\url{https://CRAN.R-project.org/package=WaveletComp}\AsIs{>. Russell, Brian, and Jiajun Han (2016)<}\url{https://www.crewes.org/Documents/ResearchReports/2016/CRR201668.pdf}\AsIs{>. Gabor, Dennis (1946) <}\url{http://genesis.eecg.toronto.edu/gabor1946.pdf}\AsIs{>. J. Laskar, P. Robutel, F. Joutel, M. Gastineau, A.C.M. Correia, and B. Levrard, B. (2004) <}\Rhref{https://doi.org/10.1051/0004-6361:20041335}{doi:10.1051/0004-6361:20041335}\AsIs{>. Laskar, J., Fienga, A., Gastineau, M., Manche, H. (2011a) <}\Rhref{https://doi.org/10.1051/0004-6361/201116836}{doi:10.1051/0004-6361/201116836}\AsIs{>. References for the astronomical solutions articles are: Laskar, J., Gastineau, M., Delisle, J.-B., Farres, A., Fienga, A. (2011b <}\Rhref{https://doi.org/10.1051/0004-6361/201117504}{doi:10.1051/0004-6361/201117504}\AsIs{>. J. Laskar (2019) <}\Rhref{https://doi.org/10.1016/B978-0-12-824360-2.00004-8}{doi:10.1016/B978-0-12-824360-2.00004-8}\AsIs{>. Zeebe, Richard E (2017) <}\Rhref{https://doi.org/10.3847/1538-3881/aa8cce}{doi:10.3847/1538-3881/aa8cce}\AsIs{>. Zeebe, R. E. and Lourens, L. J. (2019) <}\Rhref{https://doi.org/10.1016/j.epsl.2022.117595}{doi:10.1016/j.epsl.2022.117595}\AsIs{>. Richard E. Zeebe Lucas J. Lourens (2022) <}\Rhref{https://doi.org/10.1126/science.aax0612}{doi:10.1126/science.aax0612}\AsIs{>.}
\item[License]\AsIs{GPL (>= 2)}
\item[URL]\AsIs{}\url{https://github.com/stratigraphy/WaverideR}\AsIs{}
\item[Encoding]\AsIs{UTF-8}
\item[LazyData]\AsIs{true}
\item[RoxygenNote]\AsIs{7.2.3}
\item[NeedsCompilation]\AsIs{no}
\item[Repository]\AsIs{CRAN}
\item[Suggests]\AsIs{testthat (>= 3.0.0)}
\item[Config/testthat/edition]\AsIs{3}
\end{description}
\Rdcontents{\R{} topics documented:}
\inputencoding{utf8}
\HeaderA{add\_wavelet}{Add a wavelet plot}{add.Rul.wavelet}
%
\begin{Description}
Generates a plot of a wavelet scalogram which can be integrated into
a larger composite plot
\end{Description}
%
\begin{Usage}
\begin{verbatim}
add_wavelet(
  wavelet = NULL,
  lowerPeriod = NULL,
  upperPeriod = NULL,
  lower_depth_time = NULL,
  upper_depth_time = NULL,
  n.levels = 100,
  plot.COI = TRUE,
  color_brewer = "grDevices",
  palette_name = "rainbow",
  plot_dir = FALSE,
  add_lines = NULL,
  add_points = NULL,
  add_abline_h = NULL,
  add_abline_v = NULL,
  plot_horizontal = TRUE,
  period_ticks = 1,
  periodlab = "period (m)",
  main = NULL,
  yaxt = "s",
  xaxt = "s",
  depth_time_lab = "depth (m)"
)
\end{verbatim}
\end{Usage}
%
\begin{Arguments}
\begin{ldescription}
\item[\code{wavelet}] wavelet object created using the \code{\LinkA{analyze\_wavelet}{analyze.Rul.wavelet}} function.

\item[\code{lowerPeriod}] Lowest period value which will be plotted

\item[\code{upperPeriod}] Highest period value which will be plotted

\item[\code{lower\_depth\_time}] lowest depth/time value which will be plotted

\item[\code{upper\_depth\_time}] Highest depth/time value which will be plotted

\item[\code{n.levels}] Number of color levels \code{Default=100}.

\item[\code{plot.COI}] Option to plot the cone of influence \code{Default=TRUE}.

\item[\code{color\_brewer}] Name of the R package from which the color palette is chosen from.
The included R packages from which palettes can be chosen
are; the RColorBrewer, grDevices, ColorRamps and Viridis R packages.
There are many options to choose from so please
read the documentation of these packages. "\code{Default=grDevices}

\item[\code{palette\_name}] Name of the color palette which is used for plotting.
The color palettes than can be chosen depends on which the R package is specified in
the color\_brewer parameter. The included R packages from which palettes can be chosen
from are; the 'RColorBrewer', 'grDevices', 'ColorRamps' and 'Viridis' R packages.
There are many options to choose from so please
read the documentation of these packages \code{Default=rainbow}.
The R package 'viridis' has the color palette options: “magma”, “plasma”,
“inferno”, “viridis”, “mako”, and “rocket”  and “turbo”
To see the color palette options of the The R pacakge 'RColorBrewer' run
the RColorBrewer::brewer.pal.info() function
The R package 'colorRamps' has the color palette options:"blue2green",
"blue2green2red", "blue2red",    "blue2yellow", "colorRamps",    "cyan2yellow",
"green2red", "magenta2green", "matlab.like", "matlab.like2" and    "ygobb"
The R package 'grDevices' has the built in  palette options:"rainbow",
"heat.colors", "terrain.colors","topo.colors" and "cm.colors"
To see even more color palette options of the The R pacakge 'grDevices' run
the grDevices::hcl.pals() function

\item[\code{plot\_dir}] The direction of the proxy record which is assumed for tuning if time increases with increasing depth/time values
(e.g. bore hole data which gets older with increasing depth ) then plot\_dir should be set to TRUE
if time decreases with depth/time values (eg stratospheric logs where 0m is the bottom of the section)
then plot\_dir should be set to FALSE \code{plot\_dir=TRUE}

\item[\code{add\_lines}] Add  lines to the wavelet plot input should be matrix with first axis being depth/time the columns after that
should be period values  \code{Default=NULL}

\item[\code{add\_points}] Add points to the wavelet plot input should be matrix with first axis being depth/time and columns after that
should be period values \code{Default=NULL}

\item[\code{add\_abline\_h}] Add horizontal lines to the plot. Specify the lines as a vector e.g. c(2,3,5,6)  \code{Default=NULL}

\item[\code{add\_abline\_v}] Add vertical lines to the plot. Specify the lines as a vector e.g. c(2,3,5,6)  \code{Default=NULL}

\item[\code{plot\_horizontal}] plot the wavelet horizontal or vertical eg y axis is depth or y axis power \code{Default=TRUE}

\item[\code{period\_ticks}] tick mark spacing 1 is all tickmarks and higher value removes tick marks by
the fraction of the tick mark spacing value, the opposite is true for value lower than 1 which
will add aditional tickmarks

\item[\code{periodlab}] lable for the the period column

\item[\code{main}] main title

\item[\code{yaxt}] turn on of off the yaxis "s" is on "n" is off \code{Default="s"}

\item[\code{xaxt}] turn on of off the xaxis "s" is on "n" is off \code{Default="s"}

\item[\code{depth\_time\_lab}] lable for the the depth/time column
\end{ldescription}
\end{Arguments}
%
\begin{Value}
returns a plot of a wavelet scalogram
\end{Value}
%
\begin{Author}
Code based on the \LinkA{analyze.wavelet}{analyze.wavelet} and \LinkA{wt.image}{wt.image} functions of the 'WaveletComp' R package
and \LinkA{wt}{wt} function of the 'biwavelet' R package which are based on the
wavelet MATLAB code written by Christopher Torrence and Gibert P. Compo (1998).
The MTM analysis is from the astrochron R package of Meyers et al., (2012)
\end{Author}
%
\begin{References}
Angi Roesch and Harald Schmidbauer (2018). WaveletComp: Computational
Wavelet Analysis. R package version 1.1.
\url{https://CRAN.R-project.org/package=WaveletComp}

Gouhier TC, Grinsted A, Simko V (2021). R package biwavelet: Conduct Univariate and Bivariate Wavelet Analyses. (Version 0.20.21),
\url{https://github.com/tgouhier/biwavelet}

Torrence, C., and G. P. Compo. 1998. A Practical Guide to Wavelet Analysis.
Bulletin of the American Meteorological Society 79:61-78.
\url{https://paos.colorado.edu/research/wavelets/bams_79_01_0061.pdf}

Morlet, Jean, Georges Arens, Eliane Fourgeau, and Dominique Glard.
"Wave propagation and sampling theory—Part I: Complex signal and scattering in multilayered media.
" Geophysics 47, no. 2 (1982): 203-221.
\url{https://pubs.geoscienceworld.org/geophysics/article/47/2/203/68601/Wave-propagation-and-sampling-theory-Part-I}

J. Morlet, G. Arens, E. Fourgeau, D. Giard;
Wave propagation and sampling theory; Part II, Sampling theory and complex waves.
Geophysics 1982 47 (2): 222–236. \url{https://pubs.geoscienceworld.org/geophysics/article/47/2/222/68604/Wave-propagation-and-sampling-theory-Part-II}
\end{References}
%
\begin{Examples}
\begin{ExampleCode}

#generate a plot for the magnetic susceptibility data set of Pas et al., (2018)

plot.new()
layout.matrix <- matrix(c(rep(0, 2), 1, 0,0,seq(2, 6, by = 1)),
                       nrow = 2,
                      ncol = 5 ,
                       byrow = TRUE)
graphics::layout(mat = layout.matrix,
                heights = c(0.25, 1),
                # Heights of the two rows
                widths = c(rep(c(1, 2, 4,2,2), 2)))

par(mar = c(0, 0.5, 1, 0.5))


mag_wt <-
 analyze_wavelet(
   data = mag,
   dj = 1 / 100,
  lowerPeriod = 0.1,
   upperPeriod = 254,
   verbose = FALSE,
   omega_nr = 10
 )

 add_wavelet_avg(
 wavelet = mag_wt,
 plot_horizontal = TRUE,
 add_abline_h = NULL,
 add_abline_v = NULL,
 lowerPeriod = 0.15,
 upperPeriod = 80
)

par(mar = c(4, 4, 0, 0.5))


plot(
 x = c(0, 1),
 y = c(max(mag[, 1]), min(mag[, 1])),
 col = "white",
 xlab = "",
 ylab = "Time (Ma)",
 xaxt = "n",
 xaxs = "i",
 yaxs = "i",
 ylim = rev(c(max(mag[, 1]), min(mag[, 1])))
)            # Draw empty plot


polygon(
 x = c(0, 1, 1, 0),
 y = c(max(mag[, 1]), max(mag[, 1]), min(mag[, 1]), min(mag[, 1])),
 col = geo_col("Famennian")
)

text(
 0.5,
 (max(mag[, 1]) - min(mag[, 1])) / 2,
 "Fammenian",
 cex = 1,
 col = "black",
 srt = 90
)
par(mar = c(4, 0.5, 0, 0.5))


plot(
 mag[, 2],
 mag[, 1],
 type = "l",
 ylim = rev(c(max(mag[, 1]), min(mag[, 1]))),
 yaxs = "i",
 yaxt = "n",
 xlab = "Mag. suc.",
 ylab = ""
)


add_wavelet(
 wavelet = mag_wt,
 lowerPeriod = 0.15,
 upperPeriod = 80,
 lower_depth_time = NULL,
 upper_depth_time = NULL,
 n.levels = 100,
 plot.COI = TRUE,
 color_brewer = "grDevices",
 palette_name = "rainbow",
 plot_dir = FALSE,
 add_lines = NULL,
 add_points = NULL,
 add_abline_h = NULL,
 add_abline_v = NULL,
 plot_horizontal = TRUE,
 period_ticks = 1,
 periodlab = "period (m)",
 main = NULL,
 yaxt = "n",
 xaxt = "s",
 depth_time_lab = ""
)


lines(log2(mag_track_solution[,2]),mag_track_solution[,1],lwd=4,lty=4)

mag_405 <- extract_signal(
 tracked_cycle_curve = mag_track_solution,
 wavelet = mag_wt,
 period_up = 1.2,
 period_down = 0.8,
 add_mean = TRUE,
 tracked_cycle_period = 405,
 extract_cycle = 405,
 tune = FALSE,
 plot_residual = FALSE
)

plot(mag_405[,2],mag_405[,1],type="l",
    yaxt="n", yaxs = "i",
    xlab="405-kyr ecc")


mag_110 <- extract_signal(
 tracked_cycle_curve = mag_track_solution,
 wavelet = mag_wt,
 period_up = 1.25,
 period_down = 0.75,
 add_mean = TRUE,
 tracked_cycle_period = 405,
 extract_cycle = 110,
 tune = FALSE,
 plot_residual = FALSE
)

mag_110_hil <- Hilbert_transform(mag_110,demean=FALSE)

plot(mag_110[,2],mag_110[,1],type="l",
    yaxt="n", yaxs = "i",
    xlab="110-kyr ecc")

lines(mag_110_hil[,2],mag_110_hil[,1])

\end{ExampleCode}
\end{Examples}
\inputencoding{utf8}
\HeaderA{add\_wavelet\_avg}{Add a plot of a the average spectral power of a continous wavelet transform}{add.Rul.wavelet.Rul.avg}
%
\begin{Description}
Generates a plot of a the average spectral power of a continous wavelet transform
which can be added to a larger composite plot
\end{Description}
%
\begin{Usage}
\begin{verbatim}
add_wavelet_avg(
  wavelet = NULL,
  plot_horizontal = TRUE,
  add_abline_h = NULL,
  add_abline_v = NULL,
  lowerPeriod = NULL,
  upperPeriod = NULL
)
\end{verbatim}
\end{Usage}
%
\begin{Arguments}
\begin{ldescription}
\item[\code{wavelet}] wavelet object created using the \code{\LinkA{analyze\_wavelet}{analyze.Rul.wavelet}} function.

\item[\code{plot\_horizontal}] plot the wavelet horizontal or vertical eg y axis is depth or y axis power \code{Default=TRUE}

\item[\code{add\_abline\_h}] Add horizontal lines to the plot. Specify the lines as a vector e.g. c(2,3,5,6)  \code{Default=NULL}

\item[\code{add\_abline\_v}] Add vertical lines to the plot. Specify the lines as a vector e.g. c(2,3,5,6)  \code{Default=NULL}

\item[\code{lowerPeriod}] Lowest period value which will be plotted

\item[\code{upperPeriod}] Highest period value which will be plotted
\end{ldescription}
\end{Arguments}
%
\begin{Value}
returns a plot of a the average spectral power of a continuous wavelet transform
\end{Value}
%
\begin{Author}
Code based on the \LinkA{analyze.wavelet}{analyze.wavelet} and \LinkA{wt.image}{wt.image} functions of the 'WaveletComp' R package
and \LinkA{wt}{wt} function of the 'biwavelet' R package which are based on the
wavelet MATLAB code written by Christopher Torrence and Gibert P. Compo (1998).
The MTM analysis is from the astrochron R package of Meyers et al., (2012)
\end{Author}
%
\begin{References}
Angi Roesch and Harald Schmidbauer (2018). WaveletComp: Computational
Wavelet Analysis. R package version 1.1.
\url{https://CRAN.R-project.org/package=WaveletComp}

Gouhier TC, Grinsted A, Simko V (2021). R package biwavelet: Conduct Univariate and Bivariate Wavelet Analyses. (Version 0.20.21),
\url{https://github.com/tgouhier/biwavelet}

Torrence, C., and G. P. Compo. 1998. A Practical Guide to Wavelet Analysis.
Bulletin of the American Meteorological Society 79:61-78.
\url{https://paos.colorado.edu/research/wavelets/bams_79_01_0061.pdf}

Morlet, Jean, Georges Arens, Eliane Fourgeau, and Dominique Glard.
"Wave propagation and sampling theory—Part I: Complex signal and scattering in multilayered media.
" Geophysics 47, no. 2 (1982): 203-221.
\url{https://pubs.geoscienceworld.org/geophysics/article/47/2/203/68601/Wave-propagation-and-sampling-theory-Part-I}

J. Morlet, G. Arens, E. Fourgeau, D. Giard;
Wave propagation and sampling theory; Part II, Sampling theory and complex waves.
Geophysics 1982 47 (2): 222–236. \url{https://pubs.geoscienceworld.org/geophysics/article/47/2/222/68604/Wave-propagation-and-sampling-theory-Part-II}
\end{References}
%
\begin{Examples}
\begin{ExampleCode}

#generate a plot for the magnetic susceptibility data set of Pas et al., (2018)

plot.new()
layout.matrix <- matrix(c(rep(0, 2), 1, 0,0,seq(2, 6, by = 1)),
                       nrow = 2,
                      ncol = 5 ,
                       byrow = TRUE)
graphics::layout(mat = layout.matrix,
                heights = c(0.25, 1),
                # Heights of the two rows
                widths = c(rep(c(1, 2, 4,2,2), 2)))

par(mar = c(0, 0.5, 1, 0.5))


mag_wt <-
 analyze_wavelet(
   data = mag,
   dj = 1 / 100,
  lowerPeriod = 0.1,
   upperPeriod = 254,
   verbose = FALSE,
   omega_nr = 10
 )

add_wavelet_avg(
 wavelet = mag_wt,
 plot_horizontal = TRUE,
 add_abline_h = NULL,
 add_abline_v = NULL,
 lowerPeriod = 0.15,
 upperPeriod = 80
)


par(mar = c(4, 4, 0, 0.5))


plot(
 x = c(0, 1),
 y = c(max(mag[, 1]), min(mag[, 1])),
 col = "white",
 xlab = "",
 ylab = "Time (Ma)",
 xaxt = "n",
 xaxs = "i",
 yaxs = "i",
 ylim = rev(c(max(mag[, 1]), min(mag[, 1])))
)            # Draw empty plot


polygon(
 x = c(0, 1, 1, 0),
 y = c(max(mag[, 1]), max(mag[, 1]), min(mag[, 1]), min(mag[, 1])),
 col = geo_col("Famennian")
)

text(
 0.5,
 (max(mag[, 1]) - min(mag[, 1])) / 2,
 "Fammenian",
 cex = 1,
 col = "black",
 srt = 90
)
par(mar = c(4, 0.5, 0, 0.5))


plot(
 mag[, 2],
 mag[, 1],
 type = "l",
 ylim = rev(c(max(mag[, 1]), min(mag[, 1]))),
 yaxs = "i",
 yaxt = "n",
 xlab = "Mag. suc.",
 ylab = ""
)

add_wavelet(
 wavelet = mag_wt,
 lowerPeriod = 0.15,
 upperPeriod = 80,
 lower_depth_time = NULL,
 upper_depth_time = NULL,
 n.levels = 100,
 plot.COI = TRUE,
 color_brewer = "grDevices",
 palette_name = "rainbow",
 plot_dir = FALSE,
 add_lines = NULL,
 add_points = NULL,
 add_abline_h = NULL,
 add_abline_v = NULL,
 plot_horizontal = TRUE,
 period_ticks = 1,
 periodlab = "period (m)",
 main = NULL,
 yaxt = "n",
 xaxt = "s",
 depth_time_lab = ""
)

lines(log2(mag_track_solution[,2]),mag_track_solution[,1],lwd=4,lty=4)

mag_405 <- extract_signal(
 tracked_cycle_curve = mag_track_solution,
 wavelet = mag_wt,
 period_up = 1.2,
 period_down = 0.8,
 add_mean = TRUE,
 tracked_cycle_period = 405,
 extract_cycle = 405,
 tune = FALSE,
 plot_residual = FALSE
)

plot(mag_405[,2],mag_405[,1],type="l",
    yaxt="n", yaxs = "i",
    xlab="405-kyr ecc")


mag_110 <- extract_signal(
 tracked_cycle_curve = mag_track_solution,
 wavelet = mag_wt,
 period_up = 1.25,
 period_down = 0.75,
 add_mean = TRUE,
 tracked_cycle_period = 405,
 extract_cycle = 110,
 tune = FALSE,
 plot_residual = FALSE
)

mag_110_hil <- Hilbert_transform(mag_110,demean=FALSE)

plot(mag_110[,2],mag_110[,1],type="l",
    yaxt="n", yaxs = "i",
    xlab="110-kyr ecc")

lines(mag_110_hil[,2],mag_110_hil[,1])

\end{ExampleCode}
\end{Examples}
\inputencoding{utf8}
\HeaderA{age\_model\_zeeden}{Age model of Zeeden et al., (2013) for the (154-174m) interval of the IODP 926 grey scale record}{age.Rul.model.Rul.zeeden}
%
\begin{Description}
Age model (anchor points) of the IODP 926 grey scale (154-174m) record of Zeeden et al., (2013) \\{}
Anchored to the eccentricity-tilt-precession model p-0.5t of la 2004.
\end{Description}
%
\begin{Details}
Column 1: Depth (meters)\\{}
Column 2: Age (kyr)\\{}
\end{Details}
%
\begin{References}
Christian Zeeden, Frederik Hilgen, Thomas Westerhold, Lucas Lourens, Ursula Röhl, Torsten Bickert,
Revised Miocene splice, astronomical tuning and calcareous plankton biochronology of ODP Site 926 between 5 and 14.4Ma,
Palaeogeography, Palaeoclimatology, Palaeoecology,Volume 369,2013,Pages 430-451,ISSN 0031-0182,
<doi:10.1016/j.palaeo.2012.11.009>


J. Laskar, P. Robutel, F. Joutel, M. Gastineau, A.C.M. Correia, and B. Levrard, B., 2004,
A long term numerical solution for the insolation quantities of the Earth: Astron. Astrophys.,
Volume 428, 261-285. <doi:10.1051/0004-6361:20041335>
\end{References}
\inputencoding{utf8}
\HeaderA{analyze\_wavelet}{Conduct the continuous wavelet transform on a time series/signal}{analyze.Rul.wavelet}
%
\begin{Description}
Compute the continuous wavelet transform (CWT) using a Morlet wavelet
\end{Description}
%
\begin{Usage}
\begin{verbatim}
analyze_wavelet(
  data = NULL,
  dj = 1/100,
  lowerPeriod = 2,
  upperPeriod = 1024,
  verbose = FALSE,
  omega_nr = 8,
  pval = FALSE,
  n_simulations = 10,
  run_multicore = FALSE
)
\end{verbatim}
\end{Usage}
%
\begin{Arguments}
\begin{ldescription}
\item[\code{data}] Input data, should be a matrix or data frame in which
the first column is depth or time and the second column is proxy record.

\item[\code{dj}] Spacing between successive scales. The CWT analyses analyses the signal using successive periods
which increase by the power of 2 (e.g.2\textasciicircum{}0=1,2\textasciicircum{}1=2,2\textasciicircum{}2=4,2\textasciicircum{}3=8,2\textasciicircum{}4=16). To have more resolution
in-between these steps the dj parameter exists, the dj parameter specifies how many extra steps/spacing in-between
the power of 2 scaled CWT is added. The amount of steps is 1/x with a higher x indicating a smaller spacing.
Increasing the increases the computational time of the CWT \code{Default=1/200}.

\item[\code{lowerPeriod}] Lowest period to be analyzed \code{Default=2}.
The CWT analyses the signal starting from the lowerPeriod to the upperPeriod so the proper selection these
parameters allows to analyze the signal for a specific range of cycles.
scaling is done using power 2 so for the best plotting results select a value to the power or 2.

\item[\code{upperPeriod}] Upper period to be analyzed \code{Default=1024}.
The CWT analyses the signal starting from the lowerPeriod to the upperPeriod so the proper selection these
parameters allows to analyze the signal for a specific range of cycles.
scaling is done using power 2 so for the best plotting results select a value to the power or 2.

\item[\code{verbose}] Print text \code{Default=FALSE}.

\item[\code{omega\_nr}] Number of cycles contained within the Morlet wavelet

\item[\code{pval}] calculate the P-value  \code{Default=FALSE}. The p-value is based on
Monte Carlo modelling runs on surrogate data generated based on autocorrelated noise (red noise) the calculated using a windowed
(the window is half the size of the data set) temporal autocorrelation
and on shuffling the data set resulting in a random data sets which has similar spectral characteristics
to the original data set.The shuffling of the data set creates white noise which ensures that high amplitude high frequency/short
period cycles do not result in statistical significant peaks. The part of the data generated using the  autocorrelated noise (red noise)
based on the windowed  (the window is half the size of the data set) temporal autocorrelation represent a spectral signature similar to
to that of the original data. The original data might include spectral peaks which are the result of astronomical
forcing. The result is that the spectral power profile is biased towards rejecting the 0-hypothesis (e.g. no astronomical forcing).
By combining the shuffling of the data set with autocorrelated noise a surrogate data set is created which rejects
high amplitude high frequency/short period cycles and a reduced biased towards towards rejecting the 0-hypothesis if the data was
solely the result of autocorrelated noise

\item[\code{n\_simulations}] Number of simulation to be ran to generate the p-value

\item[\code{run\_multicore}] Run p-value calculation with one core or multiple cores
\end{ldescription}
\end{Arguments}
%
\begin{Value}
The output is a list (wavelet object) which contain 20 objects which are the result of the continuous wavelet transform (CWT).
Object 1: Wave - Wave values of the wavelet
Object 2: Phase - Phase of the wavelet
Object 3: Ampl - Amplitude values of the wavelet
Object 4: Power - Power values of the wavelet
Object 5: dt - Step size
Object 6: dj - Scale size
Object 7: Power.avg  - Average power values
Object 8: Period - Period values
Object 9: Scale - Scale value
Object 10: coi.1 - Cone of influence values 1
Object 11: coi.2 - Cone of influence values 2
Object 12: nc - Number of columns
Object 13: nr - Number of rows
Object 14: axis.1 - axis values 1
Object 15: axis.2 - axis values 2
Object 16: omega\_nr - Number of cycles in the wavelet
Object 17: x - x values of the data set
Object 18: y - y values of the data set
Object 19: average p value of the spectral power
Object 20: p value of spectral power
\end{Value}
%
\begin{Author}
Code based on on the \LinkA{analyze.wavelet}{analyze.wavelet} function of the 'WaveletComp' R package
and \LinkA{wt}{wt} function of the 'biwavelet' R package which are based on the
wavelet MATLAB code written by Christopher Torrence and Gibert P. Compo.
\end{Author}
%
\begin{References}
Angi Roesch and Harald Schmidbauer (2018). WaveletComp: Computational
Wavelet Analysis. R package version 1.1.
\url{https://CRAN.R-project.org/package=WaveletComp}

Gouhier TC, Grinsted A, Simko V (2021). R package biwavelet: Conduct Univariate and Bivariate Wavelet Analyses. (Version 0.20.21),
\url{https://github.com/tgouhier/biwavelet}

Torrence, C., and G. P. Compo. 1998. A Practical Guide to Wavelet Analysis.
Bulletin of the American Meteorological Society 79:61-78.
\url{https://paos.colorado.edu/research/wavelets/bams_79_01_0061.pdf}

Morlet, Jean, Georges Arens, Eliane Fourgeau, and Dominique Glard.
"Wave propagation and sampling theory—Part I: Complex signal and scattering in multilayered media.
" Geophysics 47, no. 2 (1982): 203-221.
\url{https://pubs.geoscienceworld.org/geophysics/article/47/2/203/68601/Wave-propagation-and-sampling-theory-Part-I}

J. Morlet, G. Arens, E. Fourgeau, D. Giard;
Wave propagation and sampling theory; Part II, Sampling theory and complex waves.
Geophysics 1982 47 (2): 222–236. \url{https://pubs.geoscienceworld.org/geophysics/article/47/2/222/68604/Wave-propagation-and-sampling-theory-Part-II}
\end{References}
%
\begin{Examples}
\begin{ExampleCode}

#Example 1. Using the Total Solar Irradiance data set of Steinhilver et al., (2012)
TSI_wt <-
 analyze_wavelet(
   data = TSI,
   dj = 1/200,
   lowerPeriod = 16,
   upperPeriod = 8192,
   verbose = FALSE,
   omega_nr = 6,
   pval=FALSE,
   n_simulations=10,
   run_multicore = FALSE
 )


#Example 2. Using the magnetic susceptibility data set of Pas et al., (2018)
mag_wt <-
analyze_wavelet(
data = mag,
dj = 1/100,
lowerPeriod = 0.1,
upperPeriod = 254,
verbose = FALSE,
omega_nr = 10,
pval=FALSE,
n_simulations=10,
run_multicore = FALSE
)

#Example 3. Using the greyscale data set of Zeeden et al., (2013)
grey_wt <-
 analyze_wavelet(
   data = grey,
   dj = 1/200,
   lowerPeriod = 0.02,
   upperPeriod = 256,
   verbose = FALSE,
   omega_nr = 8,
   pval=FALSE,
   n_simulations=10,
   run_multicore = FALSE
 )


\end{ExampleCode}
\end{Examples}
\inputencoding{utf8}
\HeaderA{anchor2time}{Convert a proxy record to the time domain using anchor points}{anchor2time}
%
\begin{Description}
Convert a proxy record to the time domain using anchor points made using the \code{\LinkA{astro\_anchor}{astro.Rul.anchor}} function.
\end{Description}
%
\begin{Usage}
\begin{verbatim}
anchor2time(
  anchor_points = NULL,
  data = NULL,
  genplot = FALSE,
  keep_editable = FALSE
)
\end{verbatim}
\end{Usage}
%
\begin{Arguments}
\begin{ldescription}
\item[\code{anchor\_points}] Anchor points made using the \code{\LinkA{astro\_anchor}{astro.Rul.anchor}} function or a matrix in which the first column is depth
and the second column is time.

\item[\code{data}] Data set which needs to be converted from the depth to time domain using set anchor points.
The data set should consist of a matrix with 2 column the first column should be depth
and the second column should be a proxy value.

\item[\code{genplot}] If  \code{genplot=FALSE} then 3 plots stacked on top of each other will be plotted.
Plot 1: the original data set
Plot 2: the depth time plot
Plot 3: the data set in the time domain
set to TRUE to allow for anchoring using the GUI

\item[\code{keep\_editable}] Keep option to add extra features after plotting  \code{Default=FALSE}
\end{ldescription}
\end{Arguments}
%
\begin{Value}
The output is a matrix with 2 columns.
The first column is time.
The second column sedimentation proxy value.

If \code{genplot=TRUE} then 3 plots stacked on top of each other will be plotted.
Plot 1: the original data set.
Plot 2: the depth time plot.
Plot 3: the data set in the time domain.
\end{Value}
%
\begin{Examples}
\begin{ExampleCode}
# Use the age_model_zeeden example anchor points of Zeeden et al., (2013)
#to anchor the grey data set of Zeeden et al., (2013) in the time domain.

grey_time <- anchor2time(anchor_points=age_model_zeeden,
data=grey,
genplot=FALSE,
keep_editable=FALSE)

\end{ExampleCode}
\end{Examples}
\inputencoding{utf8}
\HeaderA{anchor\_points\_Bisciaro\_al}{XRF records of the Bisciaro Fm}{anchor.Rul.points.Rul.Bisciaro.Rul.al}
%
\begin{Description}
data set consist of the tie points between the Bisciaro\_al record
of Arts (2014) and the la2011 solution of laskar et al., (20111)\\{}
\end{Description}
%
\begin{Details}
The data set is a matrix with the 4 columns.
The first column is the depth/time of the al proxy record tie-points.
The second column is the time value of the la2011 astronomical solution tie-points.
The third column is the Al value of the a; tie-point.
The fourth column is the eccentricity value of the la2011 astronomical solution tie-point.
\end{Details}
%
\begin{References}
M.C.M. Arts, 2014,
Magnetostratigrpahy and geochemical analysis of the early Miocene Bisciaro Formation
in the Contessa Valley (Northern Italy). Unpublished Bsc. thesis \\{}

Laskar, J., M. Gastineau, J. B. Delisle, A. Farrés, and A. Fienga (2011b),
Strong chaos induced by close encounters with Ceres and Vesta, Astron. Astrophys.,
532, L4,<doi:10.1051/0004-6361/201117504> \\{}
\end{References}
\inputencoding{utf8}
\HeaderA{anchor\_points\_grey}{Example anchor points for the grey scale data set of Zeeden et al., (2013)}{anchor.Rul.points.Rul.grey}
%
\begin{Description}
An example of anchor points generated using \code{\LinkA{astro\_anchor}{astro.Rul.anchor}} function \\{}
The anchor points were generated for the \code{\LinkA{grey}{grey}} grey data set of \\{}
Zeeden et al., (2013) and anchored to the code\LinkA{astrosignal\_example}{astrosignal.Rul.example} \\{}
astronomical solution which is a pre-generated ETP (eccentricity-tilt-precession) \\{}
solution(p-0.5t based on the la2004 solution) based on Laskar et al., (20004) \\{}
astronomical solution.
\end{Description}
%
\begin{Details}
Column 1: depth proxy record\\{}
Column 2: time astronomical solution  \\{}
Column 3: y-scale value proxy record\\{}
Column 4: y-scale value astronomical solution  \\{}
\end{Details}
%
\begin{References}
Christian Zeeden, Frederik Hilgen, Thomas Westerhold, Lucas Lourens, Ursula Röhl, Torsten Bickert,
Revised Miocene splice, astronomical tuning and calcareous plankton biochronology of ODP Site 926 between 5 and 14.4Ma,
Palaeogeography, Palaeoclimatology, Palaeoecology,Volume 369,2013,Pages 430-451,ISSN 0031-0182,
<doi:10.1016/j.palaeo.2012.11.009>

J. Laskar, P. Robutel, F. Joutel, M. Gastineau, A.C.M. Correia, and B. Levrard, B., 2004,
A long term numerical solution for the insolation quantities of the Earth: Astron. Astrophys.,
Volume 428, 261-285. <doi:10.1051/0004-6361:20041335>
\end{References}
\inputencoding{utf8}
\HeaderA{astrosignal\_example}{An ETP astronomical solution}{astrosignal.Rul.example}
%
\begin{Description}
The \code{\LinkA{astrosignal\_example}{astrosignal.Rul.example}} is a pre-generated ETP \\{}
(eccentricity-tilt-precession) (p-0.5t based on the la2004 solution) \\{}
the \code{\LinkA{astrosignal\_example}{astrosignal.Rul.example}} can be used to anchor the \code{\LinkA{grey}{grey}}\\{}
data set to an astronomical solution eg. \code{\LinkA{astrosignal\_example}{astrosignal.Rul.example}} \\{}
using the  \code{\LinkA{astro\_anchor}{astro.Rul.anchor}} function. the data set was generated using the \\{}
\LinkA{etp}{etp} function of the 'astrochron' R package.
The pre-generated ETP spans 5000 to 6000kyr.
\end{Description}
%
\begin{Details}
Column 1: time (kyr)\\{}
Column 2: ETP  \\{}
\end{Details}
%
\begin{Author}
Generated using the \LinkA{etp}{etp}
function of the \LinkA{astrochron-package}{astrochron.Rdash.package}.
\end{Author}
%
\begin{References}
Stephen R. Meyers,Cyclostratigraphy and the problem of astrochronologic testing,
Earth-Science Reviews,Volume 190,2019,Pages 190-223,ISSN 0012-8252
<doi:10.1016/j.earscirev.2018.11.015>

J. Laskar, P. Robutel, F. Joutel, M. Gastineau, A.C.M. Correia, and B. Levrard, B., 2004,
A long term numerical solution for the insolation quantities of the Earth: Astron. Astrophys.,
Volume 428, 261-285. <doi:10.1051/0004-6361:20041335>
\end{References}
\inputencoding{utf8}
\HeaderA{astro\_anchor}{Anchor proxy record to an astronomical solution}{astro.Rul.anchor}
%
\begin{Description}
Anchor the extracted signal to an astronomical solution using a GUI.
The \code{\LinkA{astro\_anchor}{astro.Rul.anchor}} function allows one to tie minima or maxima in the proxy record
to minima or maxima in an astronomical solution.
By tying the proxy record to an astronomical solution one will generate tie-points which
can be used to generate a astrochronological age-model
As minima or maxima in the proxy record are tied to minima or maxima in an astronomical solution it is
important to provide input which has clearly definable minima and maxima.
As such input should be of a "sinusoidal" nature otherwise the \code{extract\_astrosolution=TRUE}
and/or \code{extract\_proxy\_signal=TRUE} options need to be set to TRUE to create sinusoidal signals.

Astronomical solutions option are:
\begin{itemize}

\item{} La2004 Eccentricity solution available via the \LinkA{getLaskar}{getLaskar} function or downloadable via \url{http://vo.imcce.fr/insola/earth/online/earth/earth.html}
\item{} La2004 Obliquity solution available via the \LinkA{getLaskar}{getLaskar} function or downloadable via \url{http://vo.imcce.fr/insola/earth/online/earth/earth.html}
\item{} La2004 Precession solution available via the \LinkA{getLaskar}{getLaskar} function or downloadable via \url{http://vo.imcce.fr/insola/earth/online/earth/earth.html}
\item{} La2010a Eccentricity solution available via the \LinkA{getLaskar}{getLaskar}function or downloadable via \url{http://vo.imcce.fr/insola/earth/online/earth/earth.html}
\item{} La2010a Obliquity solution downloadable via the \url{http://vo.imcce.fr/insola/earth/online/earth/earth.html}
\item{} La2010a Precession solution downloadable via \url{http://vo.imcce.fr/insola/earth/online/earth/earth.html}
\item{} La2010b Eccentricity solution available via the \LinkA{getLaskar}{getLaskar} function or downloadable via \url{http://vo.imcce.fr/insola/earth/online/earth/earth.html}
\item{} La2010b Obliquity solution downloadable via \url{http://vo.imcce.fr/insola/earth/online/earth/earth.html}
\item{} La2010b Precession solution downloadable via \url{http://vo.imcce.fr/insola/earth/online/earth/earth.html}
\item{} La2010c Eccentricity solution available via the \LinkA{getLaskar}{getLaskar} function or downloadable via \url{http://vo.imcce.fr/insola/earth/online/earth/earth.html}
\item{} La2010c Obliquity solution downloadable via \url{http://vo.imcce.fr/insola/earth/online/earth/earth.html}
\item{} La2010c Precession solution downloadable via \url{http://vo.imcce.fr/insola/earth/online/earth/earth.html}
\item{} La2010d Eccentricity solution available via the \LinkA{getLaskar}{getLaskar} function or downloadable via \url{http://vo.imcce.fr/insola/earth/online/earth/earth.html}
\item{} La2010d Obliquity solution downloadable via \url{http://vo.imcce.fr/insola/earth/online/earth/earth.html}
\item{} La2010d Precession solution downloadable via \url{http://vo.imcce.fr/insola/earth/online/earth/earth.html}
\item{} La2011 Eccentricity solution available via the \LinkA{getLaskar}{getLaskar} function or downloadable via \url{http://vo.imcce.fr/insola/earth/online/earth/earth.html}
\item{} ZB17a Eccentricity solution downloadable via \url{https://www.soest.hawaii.edu/oceanography/faculty/zeebe_files/Astro.html}
\item{} ZB17a Obliquity solution downloadable via \url{https://www.soest.hawaii.edu/oceanography/faculty/zeebe_files/Astro.html}
\item{} ZB17b Eccentricity solution downloadable via \url{https://www.soest.hawaii.edu/oceanography/faculty/zeebe_files/Astro.html}
\item{} ZB17b Obliquity solution downloadable via \url{https://www.soest.hawaii.edu/oceanography/faculty/zeebe_files/Astro.html}
\item{} ZB17c Eccentricity solution downloadable via \url{https://www.soest.hawaii.edu/oceanography/faculty/zeebe_files/Astro.html}
\item{} ZB17c Obliquity solution downloadable via \url{https://www.soest.hawaii.edu/oceanography/faculty/zeebe_files/Astro.html}
\item{} ZB17d Eccentricity solution downloadable via \url{https://www.soest.hawaii.edu/oceanography/faculty/zeebe_files/Astro.html}
\item{} ZB17d Obliquity solution downloadable via \url{https://www.soest.hawaii.edu/oceanography/faculty/zeebe_files/Astro.html}
\item{} ZB17e Eccentricity solution downloadable via \url{https://www.soest.hawaii.edu/oceanography/faculty/zeebe_files/Astro.html}
\item{} ZB17e Obliquity solution downloadable via \url{https://www.soest.hawaii.edu/oceanography/faculty/zeebe_files/Astro.html}
\item{} ZB17f Eccentricity solution downloadable via \url{https://www.soest.hawaii.edu/oceanography/faculty/zeebe_files/Astro.html}
\item{} ZB17f Obliquity solution downloadable via \url{https://www.soest.hawaii.edu/oceanography/faculty/zeebe_files/Astro.html}
\item{} ZB17h Eccentricity solution downloadable via \url{https://www.soest.hawaii.edu/oceanography/faculty/zeebe_files/Astro.html}
\item{} ZB17h Obliquity solution downloadable via \url{https://www.soest.hawaii.edu/oceanography/faculty/zeebe_files/Astro.html}
\item{} ZB17i Eccentricity solution downloadable via \url{https://www.soest.hawaii.edu/oceanography/faculty/zeebe_files/Astro.html}
\item{} ZB17i Obliquity solution downloadable via \url{https://www.soest.hawaii.edu/oceanography/faculty/zeebe_files/Astro.html}
\item{} ZB17j Eccentricity solution downloadable via \url{https://www.soest.hawaii.edu/oceanography/faculty/zeebe_files/Astro.html}
\item{} ZB17j Obliquity solution downloadable via \url{https://www.soest.hawaii.edu/oceanography/faculty/zeebe_files/Astro.html}
\item{} ZB17k Eccentricity solution downloadable via \url{https://www.soest.hawaii.edu/oceanography/faculty/zeebe_files/Astro.html}
\item{} ZB17k Obliquity solution downloadable via \url{https://www.soest.hawaii.edu/oceanography/faculty/zeebe_files/Astro.html}
\item{} ZB17p Eccentricity solution downloadable via \url{https://www.soest.hawaii.edu/oceanography/faculty/zeebe_files/Astro.html}
\item{} ZB17p Obliquity solution downloadable via \url{https://www.soest.hawaii.edu/oceanography/faculty/zeebe_files/Astro.html}
\item{} ZB18a Eccentricity solution downloadable via \url{https://www.soest.hawaii.edu/oceanography/faculty/zeebe_files/Astro.html}
\item{} ZB18a Obliquity solution downloadable via \url{https://www.soest.hawaii.edu/oceanography/faculty/zeebe_files/Astro.html}
\item{} ZB20a Eccentricity solution downloadable via \url{https://www.soest.hawaii.edu/oceanography/faculty/zeebe_files/Astro.html}
\item{} ZB20a Obliquity solution downloadable via \url{https://www.soest.hawaii.edu/oceanography/faculty/zeebe_files/Astro.html}
\item{} ZB20b Eccentricity solution downloadable via \url{https://www.soest.hawaii.edu/oceanography/faculty/zeebe_files/Astro.html}
\item{} ZB20b Obliquity solution downloadable via \url{https://www.soest.hawaii.edu/oceanography/faculty/zeebe_files/Astro.html}
\item{} ZB20c Eccentricity solution downloadable via \url{https://www.soest.hawaii.edu/oceanography/faculty/zeebe_files/Astro.html}
\item{} ZB20c Obliquity solution downloadable via \url{https://www.soest.hawaii.edu/oceanography/faculty/zeebe_files/Astro.html}
\item{} ZB20d Eccentricity solution downloadable via \url{https://www.soest.hawaii.edu/oceanography/faculty/zeebe_files/Astro.html}
\item{} ZB20d Obliquity solution downloadable via \url{https://www.soest.hawaii.edu/oceanography/faculty/zeebe_files/Astro.html}
\item{} 405kyr eccentricity 405 metronome can be generated using the formula:\\{}
e405=0.027558-0.010739*cos(0.0118+2(pi)*(t/405000)) (laskar et al., 2004 \& laskar 2020)
\item{} 173kyr obliquity  metronome can be generated using using the formula:\\{}
es3-s6(t) = 0.144*cos(1.961+2(pi)*(t/172800) (laskar et al., 2004 \& laskar 2020)
\item{} An etp model using the \LinkA{etp}{etp} function of the 'astrochron' R package

\end{itemize}

\end{Description}
%
\begin{Usage}
\begin{verbatim}
astro_anchor(
  astro_solution = NULL,
  proxy_signal = NULL,
  proxy_min_or_max = "max",
  clip_astrosolution = FALSE,
  astrosolution_min_or_max = "max",
  clip_high = NULL,
  clip_low = NULL,
  extract_astrosolution = FALSE,
  astro_period_up = 1.2,
  astro_period_down = 0.8,
  astro_period_cycle = NULL,
  extract_proxy_signal = FALSE,
  proxy_period_up = 1.2,
  proxy_period_down = 0.8,
  proxy_period_cycle = NULL,
  pts = 3,
  verbose = FALSE,
  time_dir = TRUE,
  genplot = FALSE
)
\end{verbatim}
\end{Usage}
%
\begin{Arguments}
\begin{ldescription}
\item[\code{astro\_solution}] Input is an astronomical solution which the proxy record will be anchored to,
the input should be a matrix or data frame with the first column being
age and the second column should be a insolation/angle/value

\item[\code{proxy\_signal}] Input is the proxy data set which will
be anchored to an astronomical solution, the input should be a matrix or
data frame with the first column being  depth/time and the second column should be a proxy value.
For the best results either the astronomical components need to be pre-extracted
before anchoring. This means that a filtering/cycle extracting need to be applied to
the input data or the  extract\_proxy\_signal option needs to be set to TRUE.

\item[\code{proxy\_min\_or\_max}] Tune proxy maxima or minima to the astronomical solution \code{Default="max"}.

\item[\code{clip\_astrosolution}] Clip the astronomical solution \code{Default=FALSE}.

\item[\code{astrosolution\_min\_or\_max}] Tune to maximum or minimum values of
the astronomical solution \code{Default="max"}

\item[\code{clip\_high}] Upper value to clip to.

\item[\code{clip\_low}] Lower value to clip to.

\item[\code{extract\_astrosolution}] Extract a certain astronomical cycle/component from a
astronomical solution prior to anchoring \code{Default=FALSE}.

\item[\code{astro\_period\_up}] Specifies the upper period of the astronomical cycle which is
extracted from an astronomical solution. The astro\_period\_up is a
factor with which the astronomical component is multiplied by.  \code{Default=1.2}

\item[\code{astro\_period\_down}] Specified the lower period of the astronomical cycle which is
extracted from an astronomical solution. The astro\_period\_down value is a
factor with which the astronomical component is multiplied by. \code{Default=0.8}

\item[\code{astro\_period\_cycle}] Period (in kyr) of the to be extracted astronomical component
from the astronomical solution.

\item[\code{extract\_proxy\_signal}] Extract a certain astronomical cycle/component from a
proxy signal  \code{Default=FALSE}.

\item[\code{proxy\_period\_up}] Specifies the upper period of the astronomical cycle to be extracted
from the proxy record. The upper bound value is a factor with which the
(proxy\_period\_cycle) value is multiplied by. \code{Default=1.2}.

\item[\code{proxy\_period\_down}] Specifies the upper period of the astronomical cycle to be
extracted from the proxy record. The lower bound value is a factor with
which the astronomical component (proxy\_period\_cycle) value is multiplied by. \code{Default=0.8}.

\item[\code{proxy\_period\_cycle}] Period in kyr of the astronomical cycle/component which is to be extracted
from the proxy record.

\item[\code{pts}] The pts parameter specifies how many points to the left/right up/down the peak detect algorithm goes in detecting
a peak. The peak detecting algorithm works by comparing the values left/right up/down of it, if the values are both higher or lower
then the value a peak. To deal with error produced by this algorithm the pts parameter can be changed which can
aid in peak detection. Usually increasing the pts parameter means more peak certainty, however it also means that minor peaks might not be
picked up by the algorithm \code{Default=3}

\item[\code{verbose}] print text \code{Default=FALSE} set verbose to TRUE to allow for anchoring using text feedback commands

\item[\code{time\_dir}] The direction of the proxy record which is assumed for tuning if time increases with increasing depth/time values
(e.g. bore hole data which gets older with increasing depth ) then time\_dir should be set to TRUE
if time decreases with depth/time values (eg stratigraphic logs where 0m is the bottom of the section)
then time\_dir should be set to FALSE \code{time\_dir=TRUE}

\item[\code{genplot}] Generate plot \code{Default="FALSE"}
\end{ldescription}
\end{Arguments}
%
\begin{Value}
The output is a matrix with the 4 columns.
The first column is the depth/time of the proxy tie-point.
The second column is the time value of the astronomical solution tie-point.
The third column is the proxy value of the proxy tie-point.
The fourth column is the proxy/insolation value of the astronomical solution  tie-point.
If genplot is set to true then at plot of the of the achored points will be plotted
\end{Value}
%
\begin{References}
J. Laskar, P. Robutel, F. Joutel, M. Gastineau, A.C.M. Correia, and B. Levrard, B., 2004,
A long term numerical solution for the insolation quantities of the Earth: Astron. Astrophys.,
Volume 428, 261-285. <doi:10.1051/0004-6361:20041335> \\{}

Laskar, J., Fienga, A., Gastineau, M., Manche, H., 2011a,
La2010: A new orbital solution for the long-term motion of the Earth: Astron. Astrophys.,
Volume 532, A89 <doi:10.1051/0004-6361/201116836> \\{}

Laskar, J., Gastineau, M., Delisle, J.-B., Farres, A., Fienga, A.:
2011b, Strong chaos induced by close encounters with Ceres and Vesta, Astron: Astrophys.,
Volume 532, L4.  <doi:10.1051/0004-6361/201117504>\\{}

J. Laskar,Chapter 4 - Astrochronology,Editor(s): Felix M. Gradstein, James G. Ogg, Mark D. Schmitz, Gabi M. Ogg,Geologic Time Scale 2020,Elsevier,2020,Pages 139-158,ISBN 9780128243602,
<doi:10.1016/B978-0-12-824360-2.00004-8> or \url{https://www.sciencedirect.com/science/article/pii/B9780128243602000048} \\{}

Zeebe, R. E. and Lourens, L. J.
Geologically constrained astronomical solutions for the Cenozoic era,
Earth and Planetary Science Letters, 2022 <\Rhref{https://doi.org/10.1016/j.epsl.2022.117595}{doi:10.1016\slash{}j.epsl.2022.117595}>\\{}

Richard E. Zeebe Lucas J. Lourens ,Solar System chaos and the Paleocene–Eocene boundary age constrained by geology and astronomy.Science365,926-929(2019)
<doi:10.1126/science.aax0612>\\{}

Zeebe, Richard E. "Numerical solutions for the orbital motion of the Solar System over the past 100 Myr: limits and new results."
The Astronomical Journal 154, no. 5 (2017): 193. <doi:10.3847/1538-3881/aa8cce> \\{}

Stephen R. Meyers,Cyclostratigraphy and the problem of astrochronologic testing,
Earth-Science Reviews,Volume 190,2019,Pages 190-223,ISSN 0012-8252
<doi:10.1016/j.earscirev.2018.11.015>
\end{References}
%
\begin{Examples}
\begin{ExampleCode}

# Use the grey_track example tracking points to anchor the grey scale data set
# of Zeeden et al., (2013) to the p-0.5t la2004 solution

grey_wt <-
 analyze_wavelet(
   data = grey,
   dj = 1/200,
   lowerPeriod = 0.02,
   upperPeriod = 256,
   verbose = FALSE,
   omega_nr = 8
 )
#Use the pre-tracked grey_track curve which traced the precession cycle
grey_track <- completed_series(
 wavelet = grey_wt,
 tracked_curve  = grey_track,
 period_up  = 1.25,
 period_down  = 0.75,
 extrapolate = TRUE,
genplot = FALSE
)
# Extract precession, obliquity and eccentricity to create a synthetic insolation curve

grey_prec <- extract_signal(
tracked_cycle_curve = grey_track[,c(1,2)],
wavelet = grey_wt,
period_up = 1.2,
period_down = 0.8,
add_mean = FALSE,
tracked_cycle_period = 22,
extract_cycle = 22,
tune = FALSE,
plot_residual = FALSE
)

grey_obl <- extract_signal(
 tracked_cycle_curve = grey_track[,c(1,2)],
 wavelet = grey_wt,
 period_up = 1.2,
 period_down = 0.8,
 add_mean = FALSE,
 tracked_cycle_period = 22,
 extract_cycle = 110,
 tune = FALSE,
 plot_residual = FALSE
)

grey_ecc <- extract_signal(
 tracked_cycle_curve = grey_track[,c(1,2)],
 wavelet = grey_wt,
 period_up = 1.25,
 period_down = 0.75,
 add_mean = FALSE,
 tracked_cycle_period = 22,
 extract_cycle = 40.8,
 tune = FALSE,
 plot_residual = FALSE
)

insolation_extract <- cbind(grey_ecc[,1],grey_prec[,2]+grey_obl[,2]+grey_ecc[,2]+mean(grey[,2]))
insolation_extract <- as.data.frame(insolation_extract)
insolation_extract_mins <- min_detect(insolation_extract,pts=3)

#use the astrosignal_example to tune to which is an \cr
# ETP solution (p-0.5t la2004 solution)
astrosignal_example <- na.omit(astrosignal_example)
astrosignal_example[,2] <- -1*astrosignal_example[,2]
astrosignal <- as.data.frame(astrosignal_example)

#anchor the synthetic insolation curve extracted from the grey scale record to the insolation curve.

anchor_pts <- astro_anchor(
astro_solution = astrosignal,
proxy_signal = insolation_extract,
proxy_min_or_max = "min",
clip_astrosolution = FALSE,
astrosolution_min_or_max = "min",
clip_high = NULL,
clip_low = NULL,
extract_astrosolution  = FALSE,
astro_period_up  = NULL,
astro_period_down  = NULL,
astro_period_cycle  = NULL,
extract_proxy_signal  = FALSE,
proxy_period_up  = NULL,
proxy_period_down  = NULL,
proxy_period_cycle  = NULL,
pts=3,
verbose=FALSE, #set verbose to TRUE to allow for anchoring using text feedback commands
genplot=FALSE
)

\end{ExampleCode}
\end{Examples}
\inputencoding{utf8}
\HeaderA{Bisciaro\_al\_wt\_track}{Period of the short kyr ecc cycle in the Al record of the Bisciaro Fm}{Bisciaro.Rul.al.Rul.wt.Rul.track}
%
\begin{Description}
Data points which give the period (in meters) of the short kyr eccentricity cycle tracked \\{}
in the wavelet scalogram of the aluminium (XRF) record of the Bisciaro Formation \\{}
The period was tracked using the \code{\LinkA{track\_period\_wavelet}{track.Rul.period.Rul.wavelet}} function\\{}
The tracking is based on a reinterpretation of Arts (2014)\\{}
\end{Description}
%
\begin{Details}
Column 1: depth proxy record\\{}
Column 2: period tracked in the wavelet scalogram of the Aluminium (XRF) record
\end{Details}
%
\begin{References}
M.C.M. Arts, 2014,
Magnetostratigrpahy and geochemical analysis of the early Miocene Bisciaro Formation
in the Contessa Valley (Northern Italy). Unpublished Bsc. thesis \\{}
\end{References}
\inputencoding{utf8}
\HeaderA{Bisciaro\_ca\_wt\_track}{Period of the short kyr ecc cycle in the Ca record of the Bisciaro Fm}{Bisciaro.Rul.ca.Rul.wt.Rul.track}
%
\begin{Description}
Data points which give the period (in meters) of the short kyr eccentricity cycle tracked \\{}
in the wavelet scalogram of the calcium (XRF) record of the Bisciaro Formation \\{}
The period was tracked using the \code{\LinkA{track\_period\_wavelet}{track.Rul.period.Rul.wavelet}} function\\{}
The tracking is based on a reinterpretation of Arts (2014)\\{}
\end{Description}
%
\begin{Details}
Column 1: depth proxy record\\{}
Column 2: period tracked in the wavelet scalogram of the calcium (XRF) record
\end{Details}
%
\begin{References}
M.C.M. Arts, 2014,
Magnetostratigrpahy and geochemical analysis of the early Miocene Bisciaro Formation
in the Contessa Valley (Northern Italy). Unpublished Bsc. thesis \\{}
\end{References}
\inputencoding{utf8}
\HeaderA{Bisciaro\_Mg\_wt\_track}{Period of the short kyr ecc cycle in the Mg record of the Bisciaro Fm}{Bisciaro.Rul.Mg.Rul.wt.Rul.track}
%
\begin{Description}
Data points which give the period (in meters) of the short kyr eccentricity cycle tracked \\{}
in the wavelet scalogram of the magnesium (XRF) record of the Bisciaro Formation \\{}
The period was tracked using the \code{\LinkA{track\_period\_wavelet}{track.Rul.period.Rul.wavelet}} function\\{}
The tracking is based on a reinterpretation of Arts (2014)\\{}
\end{Description}
%
\begin{Details}
Column 1: depth proxy record\\{}
Column 2: period tracked in the wavelet scalogram of the Magnesium (XRF) record
\end{Details}
%
\begin{References}
M.C.M. Arts, 2014,
Magnetostratigrpahy and geochemical analysis of the early Miocene Bisciaro Formation
in the Contessa Valley (Northern Italy). Unpublished Bsc. thesis \\{}
\end{References}
\inputencoding{utf8}
\HeaderA{Bisciaro\_Mn\_wt\_track}{Period of the short kyr ecc cycle in the Mn record of the Bisciaro Fm}{Bisciaro.Rul.Mn.Rul.wt.Rul.track}
%
\begin{Description}
Data points which give the period (in meters) of the short kyr eccentricity cycle tracked \\{}
in the wavelet scalogram of the manganese (XRF) record of the Bisciaro Formation \\{}
The period was tracked using the \code{\LinkA{track\_period\_wavelet}{track.Rul.period.Rul.wavelet}} function\\{}
The tracking is based on a reinterpretation of Arts (2014)\\{}
\end{Description}
%
\begin{Details}
Column 1: depth proxy record\\{}
Column 2: period tracked in the wavelet scalogram of the manganese (XRF) record
\end{Details}
%
\begin{References}
M.C.M. Arts, 2014,
Magnetostratigrpahy and geochemical analysis of the early Miocene Bisciaro Formation
in the Contessa Valley (Northern Italy). Unpublished Bsc. thesis \\{}
\end{References}
\inputencoding{utf8}
\HeaderA{Bisciaro\_sial\_wt\_track}{Period of the short kyr ecc cycle in the si/Al record of the Bisciaro Fm}{Bisciaro.Rul.sial.Rul.wt.Rul.track}
%
\begin{Description}
Data points which give the period (in meters) of the short kyr eccentricity cycle tracked \\{}
in the wavelet scalogram of the silicon/aluminium (XRF) record of the Bisciaro Formation \\{}
The period was tracked using the \code{\LinkA{track\_period\_wavelet}{track.Rul.period.Rul.wavelet}} function\\{}
The tracking is based on a reinterpretation of Arts (2014)\\{}
\end{Description}
%
\begin{Details}
Column 1: depth proxy record\\{}
Column 2: period tracked in the wavelet scalogram of the silicon/aluminium (XRF) record
\end{Details}
%
\begin{References}
M.C.M. Arts, 2014,
Magnetostratigrpahy and geochemical analysis of the early Miocene Bisciaro Formation
in the Contessa Valley (Northern Italy). Unpublished Bsc. thesis \\{}
\end{References}
\inputencoding{utf8}
\HeaderA{Bisciaro\_XRF}{XRF records of the Bisciaro Fm}{Bisciaro.Rul.XRF}
%
\begin{Description}
XRF proxy records from the early Miocene Bisciaro Formation
in the Contessa Valley (Northern Italy)
\end{Description}
%
\begin{Details}
Column 1: depth proxy record\\{}
Column 2-71: XRF proxy records
\end{Details}
%
\begin{References}
M.C.M. Arts, 2014,
Magnetostratigrpahy and geochemical analysis of the early Miocene Bisciaro Formation
in the Contessa Valley (Northern Italy). Unpublished Bsc. thesis \\{}
\end{References}
\inputencoding{utf8}
\HeaderA{completed\_series}{Complete the tracking of cycle in a wavelet spectra}{completed.Rul.series}
%
\begin{Description}
Use the traced series and the existing wavelet
spectra to complete the tracking of a cycle of the wavelet spectra.
The selected points using the \code{\LinkA{track\_period\_wavelet}{track.Rul.period.Rul.wavelet}} function form a incomplete line
unless every point is tracked. However clicking every individual point along a wavelet ridge is time
intensive and error prone.
To avoid errors and save time the \code{\LinkA{completed\_series}{completed.Rul.series}} function can be used to
complete the tracing of a cycle in a wavelet spectra.The \code{\LinkA{completed\_series}{completed.Rul.series}} function interpolates the
data points selected using the \code{\LinkA{track\_period\_wavelet}{track.Rul.period.Rul.wavelet}}. A a search a algorithm then looks up and replaces
the interpolated curve values with the values of the nearest spectral peak in the wavelet spectra.
\end{Description}
%
\begin{Usage}
\begin{verbatim}
completed_series(
  wavelet = NULL,
  tracked_curve = NULL,
  period_up = 1.2,
  period_down = 0.8,
  extrapolate = TRUE,
  genplot = FALSE,
  keep_editable = FALSE
)
\end{verbatim}
\end{Usage}
%
\begin{Arguments}
\begin{ldescription}
\item[\code{wavelet}] Wavelet object created using the \code{\LinkA{analyze\_wavelet}{analyze.Rul.wavelet}} function.

\item[\code{tracked\_curve}] Traced period result from the \code{track\_period\_wavelet} function.

\item[\code{period\_up}] The period\_up parameter is the factor with which the linear interpolated tracked\_curve
curve is multiplied by. This linear interpolated tracked\_curve multiplied by the period\_up factor is
the upper boundary which is used  for detecting the spectral peak nearest to the linear interpolated tracked\_curve
curve. If no spectral peak is detected within the specified boundary the interpolated
value is used instead. between spectral peaks \code{Default=1.2},

\item[\code{period\_down}] The period\_down parameter is the factor with which the linear interpolated tracked\_curve
curve is multiplied by. This linear interpolated tracked\_curve multiplied by the period\_down factor is
the lower boundary which is used  for detecting the spectral peak nearest to the linear interpolated tracked\_curve
curve. If no spectral peak is detected within the specified boundary the interpolated
value is used instead. between spectral peaks \code{Default=0.8},

\item[\code{extrapolate}] Extrapolate the completed curve when through parts where no spectral peaks could be traced
\code{Default=TRUE}.

\item[\code{genplot}] Generate a plot \code{Default=TRUE}. The red curve is the completed curve,
the black curve is the original curve.

\item[\code{keep\_editable}] Keep option to add extra features after plotting  \code{Default=FALSE}
\end{ldescription}
\end{Arguments}
%
\begin{Value}
Returns a matrix with 2 columns
The first column is the depth axis
The second column is the completed tracking of the period a cycle of the wavelet spectra
\end{Value}
%
\begin{Examples}
\begin{ExampleCode}

#Use the grey_track example points to complete the tracking of the
# precession cycle in the wavelet spectra of the grey scale data set
# of Zeeden et al., (2013).



grey_wt <-
 analyze_wavelet(
   data = grey,
   dj = 1/200,
   lowerPeriod = 0.02,
   upperPeriod = 256,
   verbose = FALSE,
   omega_nr = 8
 )

#The ~22kyr precession cycle is between 0.25 and 1m The grey_track data
#set is a pre-loaded uncompleted tracking of the precession cycle

#grey_track <- track_period_wavelet(
#astro_cycle = 22,
#wavelet = NULL,
#n.levels = 100,
#periodlab = "Period (meters)",
#x_lab = "depth (meters)"
#)



grey_track <- completed_series(
 wavelet = grey_wt,
 tracked_curve  = grey_track,
 period_up  = 1.25,
 period_down  = 0.75,
 extrapolate = TRUE,
genplot = FALSE,
keep_editable=FALSE
)


\end{ExampleCode}
\end{Examples}
\inputencoding{utf8}
\HeaderA{curve2sedrate}{Convert a tracked tracked to a sedimentation rate curve}{curve2sedrate}
%
\begin{Description}
Converts the period of a tracked cycle to a sedimentation rate curve by
assigning a duration (in kyr) to the period of a tracked cycle
\end{Description}
%
\begin{Usage}
\begin{verbatim}
curve2sedrate(tracked_cycle_curve = NULL, tracked_cycle_period = NULL)
\end{verbatim}
\end{Usage}
%
\begin{Arguments}
\begin{ldescription}
\item[\code{tracked\_cycle\_curve}] A tracked cycle  which is the result of using the \code{\LinkA{track\_period\_wavelet}{track.Rul.period.Rul.wavelet}} function \\{}
Any input (matrix or data frame) in which the first column is depth in meters and the second column is period in meters

\item[\code{tracked\_cycle\_period}] Period of the tracked cycle (in kyr).
\end{ldescription}
\end{Arguments}
%
\begin{Value}
The output is a matrix with 2 columns
The first column is depth
The second column sedimentation rate in cm/kyr
\end{Value}
%
\begin{Examples}
\begin{ExampleCode}

#Conversion of the period (in meters) of a 405 kyr eccentricity cycle tracked
#in a wavelet spectra by assigning a duration of  405 kyr to the tracked cycle.
# the example uses the magnetic susceptibility data set of Pas et al., (2018)
# perform the CWT
mag_wt <- analyze_wavelet(data = mag,
dj = 1/100,
lowerPeriod = 0.1,
upperPeriod = 254,
verbose = FALSE,
omega_nr = 10)

#Track the 405 kyr eccentricity cycle in a wavelet spectra

#mag_track <- track_period_wavelet(astro_cycle = 405,
#                                   wavelet=mag_wt,
#                                   n.levels = 100,
#                                   periodlab = "Period (metres)",
#                                   x_lab = "depth (metres)")

#Instead of tracking, the tracked solution data set \code{\link{mag_track_solution}} is used \cr
mag_track <- mag_track_solution

mag_track_complete <- completed_series(
  wavelet = mag_wt,
  tracked_curve = mag_track,
  period_up = 1.2,
  period_down = 0.8,
  extrapolate = TRUE,
  genplot = FALSE
)

# smooth the tracking of the 405 kyr eccentricity cycle
mag_track_complete <- loess_auto(time_series = mag_track_complete,
genplot = FALSE, print_span = FALSE)

#convert period in meters to sedrate in cm/kyr
mag_track_sedrate <- curve2sedrate(tracked_cycle_curve=mag_track_complete,
tracked_cycle_period=405)

\end{ExampleCode}
\end{Examples}
\inputencoding{utf8}
\HeaderA{curve2time}{Convert the tracked curve to a depth time space}{curve2time}
%
\begin{Description}
Converts the tracked curve to a depth time space.
\end{Description}
%
\begin{Usage}
\begin{verbatim}
curve2time(
  tracked_cycle_curve = NULL,
  tracked_cycle_period = NULL,
  genplot = FALSE,
  keep_editable = FALSE
)
\end{verbatim}
\end{Usage}
%
\begin{Arguments}
\begin{ldescription}
\item[\code{tracked\_cycle\_curve}] Curve of the cycle tracked using the
\code{\LinkA{track\_period\_wavelet}{track.Rul.period.Rul.wavelet}} function \\{}
Any input (matrix or data frame) in which the first column is depth in
meters and the second column is period in meters can be used.

\item[\code{tracked\_cycle\_period}] Period of the tracked curve in kyr.

\item[\code{genplot}] Generates a plot with depth vs time \code{Default=FALSE}.

\item[\code{keep\_editable}] Keep option to add extra features after plotting  \code{Default=FALSE}
\end{ldescription}
\end{Arguments}
%
\begin{Value}
The output is a matrix with 2 columns.
The first column is depth.
The second column sedimentation rate in cm/kyr.
If \code{genplot=TRUE} then a depth vs time plot will be plotted.
\end{Value}
%
\begin{Author}
Based on the \LinkA{sedrate2time}{sedrate2time}
function of the 'astrochron' R package
\end{Author}
%
\begin{References}
Routines for astrochronologic testing, astronomical time scale construction, and
time series analysis <doi:10.1016/j.earscirev.2018.11.015>
\end{References}
%
\begin{Examples}
\begin{ExampleCode}

#Convert a tracked curve to a depth time space. The examples uses the
#magnetic susceptibility data set of Pas et al., (2018).

#'# perform the CWT
mag_wt <- analyze_wavelet(data = mag,
dj = 1/100,
lowerPeriod = 0.1,
upperPeriod = 254,
verbose = FALSE,
omega_nr = 10)

#Track the 405 kyr eccentricity cycle in a wavelet spectra

#mag_track <- track_period_wavelet(astro_cycle = 405,
#                                   wavelet=mag_wt,
#                                   n.levels = 100,
#                                   periodlab = "Period (metres)",
#                                   x_lab = "depth (metres)")

#Instead of tracking, the tracked solution data set mag_track_solution is used
mag_track <- mag_track_solution

mag_track_complete <- completed_series(
  wavelet = mag_wt,
  tracked_curve = mag_track,
  period_up = 1.2,
  period_down = 0.8,
  extrapolate = TRUE,
  genplot = FALSE
)

# smooth the tracking of the 405 kyr eccentricity cycle
mag_track_complete <- loess_auto(time_series = mag_track_complete,
genplot = FALSE, print_span = FALSE)

#convert period in meters to sedrate depth vs time
mag_track_time<- curve2time(tracked_cycle_curve=mag_track_complete,
tracked_cycle_period=405,
genplot=FALSE,
keep_editable=FALSE)


\end{ExampleCode}
\end{Examples}
\inputencoding{utf8}
\HeaderA{curve2time\_unc}{Convert the re-tracked curve results to a depth time space with uncertainty}{curve2time.Rul.unc}
%
\begin{Description}
Converts the re-tracked curve results from
\code{\LinkA{retrack\_wt\_MC}{retrack.Rul.wt.Rul.MC}} function to a depth time space while also
taking into account the uncertainty of the tracked astronomical cycle
\end{Description}
%
\begin{Usage}
\begin{verbatim}
curve2time_unc(
  tracked_cycle_curve = NULL,
  tracked_cycle_period = NULL,
  tracked_cycle_period_unc = NULL,
  tracked_cycle_period_unc_dist = "n",
  n_simulations = NULL,
  output = 1
)
\end{verbatim}
\end{Usage}
%
\begin{Arguments}
\begin{ldescription}
\item[\code{tracked\_cycle\_curve}] Curve of the cycle tracked using the
\code{\LinkA{retrack\_wt\_MC}{retrack.Rul.wt.Rul.MC}} function \\{}
Any input (matrix or data frame) with 3 columns in which column 1 is the
x-axis, column 2 is the  mean tracked frequency (in cycles/metres) column 3
1 standard deviation

\item[\code{tracked\_cycle\_period}] Period of the tracked curve in kyr.

\item[\code{tracked\_cycle\_period\_unc}] uncertainty in the period of the tracked cycle

\item[\code{tracked\_cycle\_period\_unc\_dist}] distribution of the uncertainty of the
tracked cycle value need to be either "u" for uniform distribution or
"n" for normal distribution  \code{Default="n"}

\item[\code{n\_simulations}] number of time series to be modeled

\item[\code{output}] If output = 1 a matrix with the predicted ages given the input for each run
is given. If output = 2 a matrix with 6 columns is generated,
the first column is depth/height, the other columns are the quantile
(0.025,0.373,0.5,0.6827,0.975) age values of the runs
if output = 3 a matrix with 4 columns is generated with the first column
being depth/height, column 2 is the mean tracked duration (in kyrs) column 3
is mean duration + 1 standard deviation and column 4  is mean duration -  1
standard deviation
\end{ldescription}
\end{Arguments}
%
\begin{Value}
If output = 1 a matrix with the predicted ages given the input for each run
is given
If output = 2 a matrix with 6 columns is generated, the first column is
depth/height, the other columns are the quantile
(0.02275, 0.373, 0.5, 0.6827, 0.97725) age values of the runs
if output = 3 a matrix with 4 columns is generated with the first column
being depth/height, column 2 is the mean tracked duration (in kyrs) column 3
is mean duration + 1 standard deviation and column 4  is mean duration -  1
standard deviation
\end{Value}
%
\begin{Author}
Based on the \LinkA{sedrate2time}{sedrate2time}
function of the 'astrochron' R package
\end{Author}
%
\begin{References}
Routines for astrochronologic testing, astronomical time scale construction, and
time series analysis <doi:10.1016/j.earscirev.2018.11.015>
\end{References}
%
\begin{Examples}
\begin{ExampleCode}

# Re-track the 110kyr eccentricity cycle in the wavelet scalogram
# from the XRF record of the Bisciaro data set of Arts (2014) and then
# add generate and age model including uncertainty

Bisciaro_al <- Bisciaro_XRF[, c(1, 61)]
Bisciaro_al <- astrochron::sortNave(Bisciaro_al,verbose=FALSE,genplot=FALSE)
Bisciaro_al <- astrochron::linterp(Bisciaro_al, dt = 0.01,verbose=FALSE,genplot=FALSE)
Bisciaro_al <- Bisciaro_al[Bisciaro_al[, 1] > 2, ]

Bisciaro_al_wt <-
 analyze_wavelet(
   data = Bisciaro_al,
   dj = 1 /200 ,
   lowerPeriod = 0.01,
   upperPeriod = 50,
   verbose = FALSE,
   omega_nr = 8
 )

# Bisciaro_al_wt_track <-
#   track_period_wavelet(
#     astro_cycle = 110,
#     wavelet = Bisciaro_al_wt,
#     n.levels = 100,
#     periodlab = "Period (metres)",
#     x_lab = "depth (metres)"
#   )
#
# Bisciaro_al_wt_track <- completed_series(
#   wavelet = Bisciaro_al_wt,
#   tracked_curve = Bisciaro_al_wt_track,
#   period_up = 1.2,
#   period_down = 0.8,
#   extrapolate = TRUE,
#   genplot = FALSE,
#   keep_editable = FALSE
# )
#
# Bisciaro_al_wt_track <-
#   loess_auto(
#     time_series = Bisciaro_al_wt_track,
#     genplot = FALSE,
#     print_span = FALSE,
#     keep_editable = FALSE
#   )

Bisciaro_ca <- Bisciaro_XRF[, c(1, 55)]
Bisciaro_ca <- astrochron::sortNave(Bisciaro_ca,verbose=FALSE,genplot=FALSE)
Bisciaro_ca <- astrochron::linterp(Bisciaro_ca, dt = 0.01,verbose=FALSE,genplot=FALSE)
Bisciaro_ca <- Bisciaro_ca[Bisciaro_ca[, 1] > 2, ]

Bisciaro_ca_wt <-
 analyze_wavelet(
   data = Bisciaro_ca,
   dj = 1 /200 ,
   lowerPeriod = 0.01,
   upperPeriod = 50,
   verbose = FALSE,
   omega_nr = 8
 )

# Bisciaro_ca_wt_track <-
#   track_period_wavelet(
#     astro_cycle = 110,
#     wavelet = Bisciaro_ca_wt,
#     n.levels = 100,
#     periodlab = "Period (metres)",
#     x_lab = "depth (metres)"
#   )
#
# Bisciaro_ca_wt_track <- completed_series(
#   wavelet = Bisciaro_ca_wt,
#   tracked_curve = Bisciaro_ca_wt_track,
#   period_up = 1.2,
#   period_down = 0.8,
#   extrapolate = TRUE,
#   genplot = FALSE,
#   keep_editable = FALSE
# )
#
# Bisciaro_ca_wt_track <-
#   loess_auto(
#     time_series = Bisciaro_ca_wt_track,
#     genplot = FALSE,
#     print_span = FALSE,
#     keep_editable = FALSE)

Bisciaro_sial <- Bisciaro_XRF[,c(1,64)]
Bisciaro_sial <- astrochron::sortNave(Bisciaro_sial,verbose=FALSE,genplot=FALSE)
Bisciaro_sial <- astrochron::linterp(Bisciaro_sial, dt = 0.01,verbose=FALSE,genplot=FALSE)
Bisciaro_sial <- Bisciaro_sial[Bisciaro_sial[, 1] > 2, ]

Bisciaro_sial_wt <-
 analyze_wavelet(
   data = Bisciaro_sial,
   dj = 1 /200 ,
   lowerPeriod = 0.01,
   upperPeriod = 50,
   verbose = FALSE,
   omega_nr = 8
 )

# Bisciaro_sial_wt_track <-
#   track_period_wavelet(
#     astro_cycle = 110,
#     wavelet = Bisciaro_sial_wt,
#     n.levels = 100,
#     periodlab = "Period (metres)",
#     x_lab = "depth (metres)"
#   )
#
#
# Bisciaro_sial_wt_track <- completed_series(
#   wavelet = Bisciaro_sial_wt,
#   tracked_curve = Bisciaro_sial_wt_track,
#   period_up = 1.2,
#   period_down = 0.8,
#   extrapolate = TRUE,
#   genplot = FALSE,
#   keep_editable = FALSE
# )
#
# Bisciaro_sial_wt_track <-
#   loess_auto(
#     time_series = Bisciaro_sial_wt_track,
#     genplot = FALSE,
#     print_span = FALSE,
#     keep_editable = FALSE
#   )


Bisciaro_Mn <- Bisciaro_XRF[,c(1,46)]
Bisciaro_Mn <- astrochron::sortNave(Bisciaro_Mn,verbose=FALSE,genplot=FALSE)
Bisciaro_Mn <- astrochron::linterp(Bisciaro_Mn, dt = 0.01,verbose=FALSE,genplot=FALSE)
Bisciaro_Mn <- Bisciaro_Mn[Bisciaro_Mn[, 1] > 2, ]

Bisciaro_Mn_wt <-
 analyze_wavelet(
   data = Bisciaro_Mn,
   dj = 1 /200 ,
   lowerPeriod = 0.01,
   upperPeriod = 50,
   verbose = FALSE,
   omega_nr = 8
 )

# Bisciaro_Mn_wt_track <-
#   track_period_wavelet(
#     astro_cycle = 110,
#     wavelet = Bisciaro_Mn_wt,
#     n.levels = 100,
#     periodlab = "Period (metres)",
#     x_lab = "depth (metres)"
#   )
#
#
# Bisciaro_Mn_wt_track <- completed_series(
#   wavelet = Bisciaro_Mn_wt,
#   tracked_curve = Bisciaro_Mn_wt_track,
#   period_up = 1.2,
#   period_down = 0.8,
#   extrapolate = TRUE,
#   genplot = FALSE,
#   keep_editable = FALSE
# )
# Bisciaro_Mn_wt_track <-
#   loess_auto(
#     time_series = Bisciaro_Mn_wt_track,
#     genplot = FALSE,
#     print_span = FALSE,
#     keep_editable = FALSE
#   )

Bisciaro_Mg <- Bisciaro_XRF[,c(1,71)]
Bisciaro_Mg <- astrochron::sortNave(Bisciaro_Mg,verbose=FALSE,genplot=FALSE)
Bisciaro_Mg <- astrochron::linterp(Bisciaro_Mg, dt = 0.01,verbose=FALSE,genplot=FALSE)
Bisciaro_Mg <- Bisciaro_Mg[Bisciaro_Mg[, 1] > 2, ]

Bisciaro_Mg_wt <-
 analyze_wavelet(
   data = Bisciaro_Mg,
   dj = 1 /200 ,
   lowerPeriod = 0.01,
   upperPeriod = 50,
   verbose = FALSE,
   omega_nr = 8
 )

# Bisciaro_Mg_wt_track <-
#   track_period_wavelet(
#     astro_cycle = 110,
#     wavelet = Bisciaro_Mg_wt,
#     n.levels = 100,
#     periodlab = "Period (metres)",
#     x_lab = "depth (metres)"
#   )
#
#
# Bisciaro_Mg_wt_track <- completed_series(
#   wavelet = Bisciaro_Mg_wt,
#   tracked_curve = Bisciaro_Mg_wt_track,
#   period_up = 1.2,
#   period_down = 0.8,
#   extrapolate = TRUE,
#   genplot = FALSE,
#   keep_editable = FALSE
# )
#
# Bisciaro_Mg_wt_track <-
#   loess_auto(
#     time_series = Bisciaro_Mg_wt_track,
#     genplot = FALSE,
#     print_span = FALSE,
#     keep_editable = FALSE)




wt_list_bisc <- list(Bisciaro_al_wt,
               Bisciaro_ca_wt,
               Bisciaro_sial_wt,
               Bisciaro_Mn_wt,
               Bisciaro_Mg_wt)

#Instead of tracking, the tracked solution data sets Bisciaro_al_wt_track,
#Bisciaro_ca_wt_track, Bisciaro_sial_wt_track, Bisciaro_Mn_wt_track,
# Bisciaro_Mn_wt_track and Bisciaro_Mg_wt_track are used

data_track_bisc <- cbind(Bisciaro_al_wt_track[,2],
                     Bisciaro_ca_wt_track[,2],
                     Bisciaro_sial_wt_track[,2],
                     Bisciaro_Mn_wt_track[,2],
                     Bisciaro_Mg_wt_track[,2])

x_axis_bisc <- Bisciaro_al_wt_track[,1]


bisc_retrack <- retrack_wt_MC(wt_list = wt_list_bisc,
             data_track = data_track_bisc,
             x_axis = x_axis_bisc,
             nr_simulations = 20,
             seed_nr = 1337,
             verbose = FALSE,
             genplot = FALSE,
             keep_editable = FALSE,
             create_GIF = FALSE,
             plot_GIF = FALSE,
             width_plt =  600,
             height_plt = 450,
            period_up  =  1.5,
             period_down = 0.5,
             plot.COI = TRUE,
             n.levels = 100,
             palette_name = "rainbow",
             color_brewer = "grDevices",
             periodlab = "Period (metres)",
             x_lab = "depth (metres)",
             add_avg = FALSE,
             time_dir = TRUE,
             file_name = NULL,
             run_multicore = FALSE,
             output = 5,
             n_imgs = 50,
             plot_horizontal = TRUE,
             empty_folder = FALSE)

bisc_retrack_age_incl_unc <- curve2time_unc(tracked_cycle_curve = bisc_retrack,
tracked_cycle_period = 110,
tracked_cycle_period_unc = ((135-110)+(110-95))/2,
tracked_cycle_period_unc_dist = "n",
n_simulations = 20,
output = 1)



\end{ExampleCode}
\end{Examples}
\inputencoding{utf8}
\HeaderA{curve2time\_unc\_anchor}{Convert the re-tracked curve results to a depth time space with uncertainty}{curve2time.Rul.unc.Rul.anchor}
%
\begin{Description}
Converts the re-tracked curve results from
\code{\LinkA{retrack\_wt\_MC}{retrack.Rul.wt.Rul.MC}} function to a depth time space unsing anchor dates
while also taking into account the uncertainty of the tracked astronomical cycle
\end{Description}
%
\begin{Usage}
\begin{verbatim}
curve2time_unc_anchor(
  age_constraints = NULL,
  tracked_cycle_curve = NULL,
  tracked_cycle_period = NULL,
  tracked_cycle_period_unc = NULL,
  tracked_cycle_period_unc_dist = "n",
  n_simulations = 20,
  gap_constraints = NULL,
  proxy_data = NULL,
  cycles_check = NULL,
  uncer_cycles_check = NULL,
  max_runs = 1000,
  run_multicore = FALSE,
  verbose = FALSE,
  genplot = FALSE,
  keep_nr = 2,
  keep_all_time_curves = FALSE,
  dj = 1/200,
  lowerPeriod = 1,
  upperPeriod = 4600,
  omega_nr = 6,
  seed_nr = 1337,
  dir = TRUE
)
\end{verbatim}
\end{Usage}
%
\begin{Arguments}
\begin{ldescription}
\item[\code{age\_constraints}] age constraints for the modelling run
Input should be a data frame with 7 columns, the first columns are the ID names
the second column are the ages (usually in kyr) the third column is the uncertainty (usually in kyr) given as
the fourth column is the distribution which is either "n" for a normal distribution or "u" for a uniform
distribution. The fifth column is the location in the depth domain of the age constraint. the sixth column is the
location/thickness uncertainty of the age\_constraint in the depth domain. The seventh column is the uncertainty distribution of the
age\_constrain in the depth domain

\item[\code{tracked\_cycle\_curve}] Curve of the cycle tracked using the
\code{\LinkA{retrack\_wt\_MC}{retrack.Rul.wt.Rul.MC}} function \\{}
Any input (matrix or data frame) with 3 columns in which column 1 is the
x-axis, column 2 is the  mean tracked frequency (in cycles/metres) column 3
1 standard deviation

\item[\code{tracked\_cycle\_period}] Period of the tracked curve in kyr.

\item[\code{tracked\_cycle\_period\_unc}] uncertainty in the period of the tracked cycle

\item[\code{tracked\_cycle\_period\_unc\_dist}] distribution of the uncertainty of the
tracked cycle value need to be either "u" for uniform distribution or
"n" for normal distribution  \code{Default="n"}

\item[\code{n\_simulations}] number of time series to be modeled \code{Default=20}

\item[\code{gap\_constraints}] gap parameters for the modelling run
input should be a data frame with

\item[\code{max\_runs}] maximum runs before one of the  age constraints is dropped \code{Default=1000}

\item[\code{run\_multicore}] Run function using multiple cores \code{Default="FALSE"}

\item[\code{verbose}] Print text \code{Default=FALSE}.

\item[\code{genplot}] generate plot codeDefault=FALSE

\item[\code{keep\_nr}] minimal number of age constraints to be kept \code{Default=2}

\item[\code{keep\_all\_time\_curves}] weather to keep all the generated age curves
including the ones rejected from the modelling run codeDefault=FALSE
\end{ldescription}
\end{Arguments}
%
\begin{Value}
The output is a list of 3 or 4 elements
if keep\_all\_time\_curves is set to TRUE
then the list consist of the x-axis, all the fitted curves in a matrix format,
the astrochronologically fitted age of the anchor, all the generated depth time curves
if keep\_all\_time\_curves is set to TRUE then the list consists of the x-axis,
all the fitted curves in a matrix format and the astrochronologically fitted age of the anchor
If \code{genplot=TRUE} then 3 plots stacked on top of each other will be plotted.
Plot 1: the original data set.
Plot 2: the depth time plot.
Plot 3: the data set in the time domain.
\#'
\end{Value}
%
\begin{Author}
Part of the code is based on the \LinkA{sedrate2time}{sedrate2time}
function of the 'astrochron' R package
\end{Author}
%
\begin{References}
Routines for astrochronologic testing, astronomical time scale construction, and
time series analysis <doi:10.1016/j.earscirev.2018.11.015>

'@examples



Bisciaro\_al <- Bisciaro\_XRF[, c(1, 61)]
Bisciaro\_al <-
 astrochron::sortNave(Bisciaro\_al, verbose = FALSE, genplot = FALSE)
Bisciaro\_al <-
 astrochron::linterp(Bisciaro\_al,
                     dt = 0.01,
                     verbose = FALSE,
                     genplot = FALSE)
Bisciaro\_al <- Bisciaro\_al[Bisciaro\_al[, 1] > 2, ]

Bisciaro\_al\_wt <-
 analyze\_wavelet(
   data = Bisciaro\_al,
   dj = 1 / 200 ,
   lowerPeriod = 0.01,
   upperPeriod = 50,
   verbose = FALSE,
   omega\_nr = 8
 )
\# Bisciaro\_al\_wt\_track <-
\#   track\_period\_wavelet(
\#     astro\_cycle = 110,
\#     wavelet = Bisciaro\_al\_wt,
\#     n.levels = 100,
\#     periodlab = "Period (metres)",
\#     x\_lab = "depth (metres)"
\#   )
\#
\# Bisciaro\_al\_wt\_track <- completed\_series(
\#   wavelet = Bisciaro\_al\_wt,
\#   tracked\_curve = Bisciaro\_al\_wt\_track,
\#   period\_up = 1.2,
\#   period\_down = 0.8,
\#   extrapolate = TRUE,
\#   genplot = FALSE,
\#   keep\_editable = FALSE
\# )
\#
\# Bisciaro\_al\_wt\_track <-
\#   loess\_auto(
\#     time\_series = Bisciaro\_al\_wt\_track,
\#     genplot = FALSE,
\#     print\_span = FALSE,
\#     keep\_editable = FALSE
\#   )



Bisciaro\_ca <- Bisciaro\_XRF[, c(1, 55)]
Bisciaro\_ca <-
 astrochron::sortNave(Bisciaro\_ca, verbose = FALSE, genplot = FALSE)
Bisciaro\_ca <-
 astrochron::linterp(Bisciaro\_ca,
                     dt = 0.01,
                     verbose = FALSE,
                     genplot = FALSE)
Bisciaro\_ca <- Bisciaro\_ca[Bisciaro\_ca[, 1] > 2, ]

Bisciaro\_ca\_wt <-
 analyze\_wavelet(
   data = Bisciaro\_ca,
   dj = 1 / 200 ,
   lowerPeriod = 0.01,
   upperPeriod = 50,
   verbose = FALSE,
   omega\_nr = 8
 )


\#
\# Bisciaro\_ca\_wt\_track <-
\#   track\_period\_wavelet(
\#     astro\_cycle = 110,
\#     wavelet = Bisciaro\_ca\_wt,
\#     n.levels = 100,
\#     periodlab = "Period (metres)",
\#     x\_lab = "depth (metres)"
\#   )
\#
\# Bisciaro\_ca\_wt\_track <- completed\_series(
\#   wavelet = Bisciaro\_ca\_wt,
\#   tracked\_curve = Bisciaro\_ca\_wt\_track,
\#   period\_up = 1.2,
\#   period\_down = 0.8,
\#   extrapolate = TRUE,
\#   genplot = FALSE,
\#   keep\_editable = FALSE
\# )
\#
\# Bisciaro\_ca\_wt\_track <-
\#   loess\_auto(
\#     time\_series = Bisciaro\_ca\_wt\_track,
\#     genplot = FALSE,
\#     print\_span = FALSE,
\#     keep\_editable = FALSE
\#   )


Bisciaro\_sial <- Bisciaro\_XRF[, c(1, 64)]
Bisciaro\_sial <-
 astrochron::sortNave(Bisciaro\_sial, verbose = FALSE, genplot = FALSE)
Bisciaro\_sial <-
 astrochron::linterp(Bisciaro\_sial,
                     dt = 0.01,
                     verbose = FALSE,
                     genplot = FALSE)
Bisciaro\_sial <- Bisciaro\_sial[Bisciaro\_sial[, 1] > 2, ]

Bisciaro\_sial\_wt <-
 analyze\_wavelet(
   data = Bisciaro\_sial,
   dj = 1 / 200 ,
   lowerPeriod = 0.01,
   upperPeriod = 50,
   verbose = FALSE,
   omega\_nr = 8
 )

\#Bisciaro\_sial\_wt\_track <-
\#   track\_period\_wavelet(
\#     astro\_cycle = 110,
\#     wavelet = Bisciaro\_sial\_wt,
\#     n.levels = 100,
\#     periodlab = "Period (metres)",
\#     x\_lab = "depth (metres)"
\#   )
\#
\#
\# Bisciaro\_sial\_wt\_track <- completed\_series(
\#   wavelet = Bisciaro\_sial\_wt,
\#   tracked\_curve = Bisciaro\_sial\_wt\_track,
\#   period\_up = 1.2,
\#   period\_down = 0.8,
\#   extrapolate = TRUE,
\#   genplot = FALSE,
\#   keep\_editable = FALSE
\# )
\#
\# Bisciaro\_sial\_wt\_track <-
\#   loess\_auto(
\#     time\_series = Bisciaro\_sial\_wt\_track,
\#     genplot = FALSE,
\#     print\_span = FALSE,
\#     keep\_editable = FALSE
\#   )


Bisciaro\_Mn <- Bisciaro\_XRF[, c(1, 46)]
Bisciaro\_Mn <-
 astrochron::sortNave(Bisciaro\_Mn, verbose = FALSE, genplot = FALSE)
Bisciaro\_Mn <-
 astrochron::linterp(Bisciaro\_Mn,
                     dt = 0.01,
                     verbose = FALSE,
                     genplot = FALSE)
Bisciaro\_Mn <- Bisciaro\_Mn[Bisciaro\_Mn[, 1] > 2, ]

Bisciaro\_Mn\_wt <-
 analyze\_wavelet(
   data = Bisciaro\_Mn,
   dj = 1 / 200 ,
   lowerPeriod = 0.01,
   upperPeriod = 50,
   verbose = FALSE,
   omega\_nr = 8
 )

\# Bisciaro\_Mn\_wt\_track <-
\#   track\_period\_wavelet(
\#     astro\_cycle = 110,
\#     wavelet = Bisciaro\_Mn\_wt,
\#     n.levels = 100,
\#     periodlab = "Period (metres)",
\#     x\_lab = "depth (metres)"
\#   )
\#
\#
\# Bisciaro\_Mn\_wt\_track <- completed\_series(
\#   wavelet = Bisciaro\_Mn\_wt,
\#   tracked\_curve = Bisciaro\_Mn\_wt\_track,
\#   period\_up = 1.2,
\#   period\_down = 0.8,
\#   extrapolate = TRUE,
\#   genplot = FALSE,
\#   keep\_editable = FALSE
\# )
\# Bisciaro\_Mn\_wt\_track <-
\#   loess\_auto(
\#     time\_series = Bisciaro\_Mn\_wt\_track,
\#     genplot = FALSE,
\#     print\_span = FALSE,
\#     keep\_editable = FALSE
\#   )

Bisciaro\_Mg <- Bisciaro\_XRF[, c(1, 71)]
Bisciaro\_Mg <-
 astrochron::sortNave(Bisciaro\_Mg, verbose = FALSE, genplot = FALSE)
Bisciaro\_Mg <-
 astrochron::linterp(Bisciaro\_Mg,
                     dt = 0.01,
                     verbose = FALSE,
                     genplot = FALSE)
Bisciaro\_Mg <- Bisciaro\_Mg[Bisciaro\_Mg[, 1] > 2, ]

Bisciaro\_Mg\_wt <-
 analyze\_wavelet(
   data = Bisciaro\_Mg,
   dj = 1 / 200 ,
   lowerPeriod = 0.01,
   upperPeriod = 50,
   verbose = FALSE,
   omega\_nr = 8
 )

\# Bisciaro\_Mg\_wt\_track <-
\#   track\_period\_wavelet(
\#     astro\_cycle = 110,
\#     wavelet = Bisciaro\_Mg\_wt,
\#     n.levels = 100,
\#     periodlab = "Period (metres)",
\#     x\_lab = "depth (metres)"
\#   )
\#
\#
\# Bisciaro\_Mg\_wt\_track <- completed\_series(
\#   wavelet = Bisciaro\_Mg\_wt,
\#   tracked\_curve = Bisciaro\_Mg\_wt\_track,
\#   period\_up = 1.2,
\#   period\_down = 0.8,
\#   extrapolate = TRUE,
\#   genplot = FALSE,
\#   keep\_editable = FALSE
\# )
\#
\# Bisciaro\_Mg\_wt\_track <-
\#   loess\_auto(
\#     time\_series = Bisciaro\_Mg\_wt\_track,
\#     genplot = FALSE,
\#     print\_span = FALSE,
\#     keep\_editable = FALSE
\#   )




wt\_list\_bisc <- list(Bisciaro\_al\_wt,
                    Bisciaro\_ca\_wt,
                    Bisciaro\_sial\_wt,
                    Bisciaro\_Mn\_wt,
                    Bisciaro\_Mg\_wt)


data\_track\_bisc <- cbind(
 Bisciaro\_al\_wt\_track[, 2],
 Bisciaro\_ca\_wt\_track[, 2],
Bisciaro\_sial\_wt\_track[, 2],
 Bisciaro\_Mn\_wt\_track[, 2],
 Bisciaro\_Mg\_wt\_track[, 2]
)

x\_axis\_bisc <- Bisciaro\_al\_wt\_track[, 1]

bisc\_retrack <- retrack\_wt\_MC(
wt\_list = wt\_list\_bisc,
 data\_track = data\_track\_bisc,
 x\_axis = x\_axis\_bisc,
 nr\_simulations = 500,
 seed\_nr = 1337,
 verbose = TRUE,
 genplot = FALSE,
 keep\_editable = FALSE,
 create\_GIF = FALSE,
 plot\_GIF = FALSE,
 width\_plt =  600,
 height\_plt = 450,
 period\_up  =  1.5,
 period\_down = 0.5,
 plot.COI = TRUE,
 n.levels = 100,
 palette\_name = "rainbow",
 color\_brewer = "grDevices",
periodlab = "Period (metres)",
x\_lab = "depth (metres)",
add\_avg = FALSE,
time\_dir = TRUE,
file\_name = "TEST",
run\_multicore = TRUE,
output = 5,
 n\_imgs = 50,
 plot\_horizontal = TRUE,
 empty\_folder = FALSE
)

proxy\_list\_bisc <- list(Bisciaro\_alt,
                    Bisciaro\_ca,
                    Bisciaro\_sial,
                    Bisciaro\_Mn,
                    Bisciaro\_Mg)




id <- c("CCT18\_322", "CCT18\_315", "CCT18\_311")
ages <- c(20158, 20575, 20857)
ageSds <- c(28, 40, 34)
ages\_unc\_dist <- c("n", "n", "n")
position <- c(13.3, 7.25, 3.2)
anchor\_thick <- c(0.2, 0.1, 0.1)
anchor\_thick\_unc\_dist <- c("u", "u", "u")

ash\_Bisc <-
 as.data.frame(
   cbind(
     id,
     ages,
     ageSds,
     ages\_unc\_dist,
     position,
     anchor\_thick,
     anchor\_thick\_unc\_dist
   )
 )

gap\_dur = c(10, 20)
gap\_unc = c(3, 10)
gap\_depth = c(2.5, 9)
gap\_unc\_dist = c("n", "n")


gap\_constraints\_Bisc <-
 as.data.frame(cbind(gap\_dur, gap\_unc, gap\_depth, gap\_unc\_dist))

cycles\_checks <- c(110,40,22)
uncer\_cycles\_checks <- c(20,5,7)

curve2time\_unc\_anchor\_res <-
 curve2time\_unc\_anchor(
 age\_constraints =  ash\_Bisc,
  tracked\_cycle\_curve =  bisc\_retrack,
  tracked\_cycle\_period = 110,
   tracked\_cycle\_period\_unc = ((135 - 110) + (110 - 95)) / 2,
  tracked\_cycle\_period\_unc\_dist = "n",
   n\_simulations = 20,
   gap\_constraints = gap\_constraints\_Bisc,
   proxy\_data = proxy\_list\_bisc,
   cycles\_check = NULL,
   uncer\_cycles\_check = NULL,
   cycles\_check = cycles\_checks,
   uncer\_cycles\_check = uncer\_cycles\_checks,
  max\_runs = 1000,
   run\_multicore = FALSE,
   verbose = FALSE,
   genplot = FALSE,
   keep\_nr = 2,
   keep\_all\_time\_curves = FALSE,
   dj = 1/200,
   lowerPeriod =1,
   upperPeriod =2500,
   omega\_nr = 6,
   seed\_nr=1337,
   dir=TRUE
 )















\end{References}
\inputencoding{utf8}
\HeaderA{curve2tune}{Convert data from the depth to the time domain}{curve2tune}
%
\begin{Description}
Converts a data set from the depth to the time domain
using a tracked curve/cycle to depth domain an assigning a duration (in kyr) set
tracked curve/cycle.
\end{Description}
%
\begin{Usage}
\begin{verbatim}
curve2tune(
  data = NULL,
  tracked_cycle_curve = NULL,
  tracked_cycle_period = NULL,
  genplot = FALSE,
  keep_editable = FALSE
)
\end{verbatim}
\end{Usage}
%
\begin{Arguments}
\begin{ldescription}
\item[\code{data}] Data set (matrix with 2 columns 1st column depth 2nd column proxy value)
which was used as input for the \code{\LinkA{analyze\_wavelet}{analyze.Rul.wavelet}} function. \\{}
That result was then used to tracked a cycle using the \code{\LinkA{track\_period\_wavelet}{track.Rul.period.Rul.wavelet}} function

\item[\code{tracked\_cycle\_curve}] Tracking result of a cycle tracked using the
\code{\LinkA{track\_period\_wavelet}{track.Rul.period.Rul.wavelet}} function \\{}
Any input (matrix or data frame) in which the first column
is depth in meters and the second column is period in meters can be used.

\item[\code{tracked\_cycle\_period}] Period of the tracked curve (in kyr).

\item[\code{genplot}] If \code{genplot=TRUE} 3 plots stacked on top of each other will be plotted.
Plot 1: the original data set.
Plot 2: the depth time plot.
Plot 3: the data set in the time domain.

\item[\code{keep\_editable}] Keep option to add extra features after plotting  \code{Default=FALSE}
\end{ldescription}
\end{Arguments}
%
\begin{Value}
The output is a matrix with 2 columns.
The first column is time.
The second column sedimentation proxy value.

If \code{genplot=TRUE} then 3 plots stacked on top of each other will be plotted.
Plot 1: the original data set.
Plot 2: the depth time plot.
Plot 3: the data set in the time domain.
\end{Value}
%
\begin{Author}
Part of the code is based on the \LinkA{sedrate2time}{sedrate2time}
function of the 'astrochron' R package
\end{Author}
%
\begin{References}
Routines for astrochronologic testing, astronomical time scale construction, and
time series analysis <doi:10.1016/j.earscirev.2018.11.015>
\end{References}
%
\begin{Examples}
\begin{ExampleCode}

#The example uses the magnetic susceptibility data set of Pas et al., (2018).
# perform the CWT
mag_wt <- analyze_wavelet(data = mag,
dj = 1/100,
lowerPeriod = 0.1,
upperPeriod = 254,
verbose = FALSE,
omega_nr = 10)

#Track the 405 kyr eccentricity cycle in a wavelet spectra

#mag_track <- track_period_wavelet(astro_cycle = 405,
#                                   wavelet=mag_wt,
#                                   n.levels = 100,
#                                   periodlab = "Period (meters)",
#                                   x_lab = "depth (meters)")

#Instead of tracking, the tracked solution data set mag_track_solution is used
mag_track <- mag_track_solution

mag_track_complete <- completed_series(
  wavelet = mag_wt,
  tracked_curve = mag_track,
  period_up = 1.2,
  period_down = 0.8,
  extrapolate = TRUE,
  genplot = FALSE
)

# smooth the tracking of the 405 kyr eccentricity cycle
mag_track_complete <- loess_auto(time_series = mag_track_complete,
genplot = FALSE, print_span = FALSE)

mag_track_time<- curve2tune(data=mag,
                           tracked_cycle_curve=mag_track_complete,
                           tracked_cycle_period=405,
                           genplot = FALSE,
                           keep_editable=FALSE)

\end{ExampleCode}
\end{Examples}
\inputencoding{utf8}
\HeaderA{delpts\_tracked\_period\_wt}{Remove tracking points which were tracked in a wavelet spectra}{delpts.Rul.tracked.Rul.period.Rul.wt}
%
\begin{Description}
Interactively select points for deletion
With the  \code{\LinkA{track\_period\_wavelet}{track.Rul.period.Rul.wavelet}} function it is possible to track points in a wavelet spectra,
however errors can be made and as such it is possible to delete these points with the \code{\LinkA{delpts\_tracked\_period\_wt}{delpts.Rul.tracked.Rul.period.Rul.wt}} function.
This function allows one to select points for deletion.
\#'
\end{Description}
%
\begin{Usage}
\begin{verbatim}
delpts_tracked_period_wt(
  tracking_pts = NULL,
  wavelet = NULL,
  n.levels = 100,
  periodlab = "Period (metres)",
  x_lab = "depth (metres)",
  palette_name = "rainbow",
  color_brewer = "grDevices"
)
\end{verbatim}
\end{Usage}
%
\begin{Arguments}
\begin{ldescription}
\item[\code{tracking\_pts}] Points tracked using the \code{\LinkA{track\_period\_wavelet}{track.Rul.period.Rul.wavelet}} function.

\item[\code{wavelet}] Wavelet object created using the \code{\LinkA{analyze\_wavelet}{analyze.Rul.wavelet}} function.

\item[\code{n.levels}] Number of color levels \code{Default=100}.

\item[\code{periodlab}] label for the y-axis \code{Default="Period (metres)"}.

\item[\code{x\_lab}] label for the x-axis \code{Default="depth (metres)"}.

\item[\code{palette\_name}] Name of the color palette which is used for plotting.
The color palettes than can be chosen depends on which the R package is specified in
the color\_brewer parameter. The included R packages from which palettes can be chosen
from are; the 'RColorBrewer', 'grDevices', 'ColorRamps' and 'Viridis' R packages.
There are many options to choose from so please
read the documentation of these packages \code{Default=rainbow}.
The R package 'viridis' has the color palette options: “magma”, “plasma”,
“inferno”, “viridis”, “mako”, and “rocket”  and “turbo”
To see the color palette options of the The R pacakge 'RColorBrewer' run
the RColorBrewer::brewer.pal.info() function
The R package 'colorRamps' has the color palette options:"blue2green",
"blue2green2red", "blue2red",    "blue2yellow", "colorRamps",    "cyan2yellow",
"green2red", "magenta2green", "matlab.like", "matlab.like2" and    "ygobb"
The R package 'grDevices' has the built in  palette options:"rainbow",
"heat.colors", "terrain.colors","topo.colors" and "cm.colors"
To see even more color palette options of the The R pacakge 'grDevices' run
the grDevices::hcl.pals() function

\item[\code{color\_brewer}] Name of the R package from which the color palette is chosen from.
The included R packages from which palettes can be chosen
are; the RColorBrewer, grDevices, ColorRamps and Viridis R packages.
There are many options to choose from so please
read the documentation of these packages. "\code{Default=grDevices}
\end{ldescription}
\end{Arguments}
%
\begin{Value}
The results of the deletion of the tracking points is a matrix with 3 columns.
The first column is depth/time
The second column is the period of the tracked cycle
The third column is the sedimentation rate based on the duration (in time) of the tracked cycle
\end{Value}
%
\begin{Examples}
\begin{ExampleCode}

#Track the 405kyr eccentricity cycle in the magnetic susceptibility record
# of the Sullivan core of Pas et al., (2018)

mag_wt <- analyze_wavelet(data = mag,
dj = 1/100,
lowerPeriod = 0.1,
upperPeriod = 254,
verbose = FALSE,
omega_nr = 10)

#mag_track <- track_period_wavelet(astro_cycle = 405,
#                                   wavelet=mag_wt,
#                                   n.levels = 100,
#                                   periodlab = "Period (metres)",
#                                   x_lab = "depth (metres)"
#                                    palette_name ="rainbow",
#                                    color_brewer ="grDevices)

#load the mag_track_solution data set to get an example data set from which
#data points can be deleted


mag_track_corr <- delpts_tracked_period_wt(tracking_pts = mag_track_solution,
                                    wavelet = mag_wt,
                                    n.levels = 100,
                                    periodlab = "Period (metres)",
                                    x_lab = "depth (metres)",
                                   palette_name ="rainbow",
                                   color_brewer ="grDevices")


\end{ExampleCode}
\end{Examples}
\inputencoding{utf8}
\HeaderA{depth\_rank\_example}{An example depth rank series}{depth.Rul.rank.Rul.example}
%
\begin{Description}
The \code{\LinkA{depth\_rank\_example}{depth.Rul.rank.Rul.example}} example data set is a depth rank series which \\{}
can be used as input for the \code{\LinkA{lithlog\_disc}{lithlog.Rul.disc}} function which creates a \\{}
discritzed record which can then be used as input in the \code{\LinkA{analyze\_wavelet}{analyze.Rul.wavelet}}\\{}
function
\end{Description}
%
\begin{Details}
Column 1: depth (meters)\\{}
Column 2: depth rank \\{}
\end{Details}
\inputencoding{utf8}
\HeaderA{dur\_gaps}{calculate the duration of stratigraphic gaps using astronomical cycles}{dur.Rul.gaps}
%
\begin{Description}
calculate the duration of stratigraphic gaps using the duration
of stable astronomical cycles
\end{Description}
%
\begin{Usage}
\begin{verbatim}
dur_gaps(
  proxies = NULL,
  retracked_period_1 = NULL,
  retracked_period_2 = NULL,
  min_max = NULL,
  n_simulations = 10,
  tracked_cycle_period = NULL,
  tracked_cycle_period_unc = NULL,
  tracked_cycle_period_unc_dist = "u",
  pts = 5,
  dj = 1/200,
  lowerPeriod = 1,
  upperPeriod = 3200,
  period_max = NULL,
  period_min = NULL,
  missing_cycle_dur = NULL,
  n_cycles_missing = 1,
  missing_cycle_unc = NULL,
  missing_cycle_unc_dist = "u",
  seed_nr = 1337,
  run_multicore = FALSE
)
\end{verbatim}
\end{Usage}
%
\begin{Arguments}
\begin{ldescription}
\item[\code{proxies}] list of proxies which were used to create a astrochronological
age model and which are used to calculate the duration of the gap

\item[\code{retracked\_period\_1}] A matrix of 3 columns in which the first column
is depth/height.The second column is the period of the tracked cycle.
The thirds column is uncertainty given as 1 standard deviation for the
period of the tracked cycle. The gap to be modeled should be located
in between retracked\_period\_1 and retracked\_period\_2

\item[\code{retracked\_period\_2}] A matrix of 3 columns in which the first column
is depth/height.The second column is the period of the tracked cycle.
The thirds column is uncertainty given as 1 standard deviation for the
period of the tracked cycle. The gap to be modeled should be located
in between retracked\_period\_1 and retracked\_period\_2

\item[\code{min\_max}] list of "min" or "max" indicating whether time should be
estimated between minima or maxima for each proxy

\item[\code{n\_simulations}] number of gap duration to calculate

\item[\code{tracked\_cycle\_period}] period in time of the tracked cycle

\item[\code{tracked\_cycle\_period\_unc}] uncertainty in the period of the tracked cycle

\item[\code{tracked\_cycle\_period\_unc\_dist}] distribution of the uncertainty of the
tracked cycle value need to be either "u" for uniform distribution or
"n" for normal distribution  \code{Default="u"}

\item[\code{pts}] the pts parameter specifies how many points to the left/right up/down the peak detect algorithm goes in detecting
a peak. The peak detecting algorithm works by comparing the values left/right up/down of it, if the values are both higher or lower
then the value a peak. To deal with error produced by this algorithm the pts parameter can be changed which can
aid in peak detection. Usually increasing the pts parameter means more peak certainty, however it also means that minor peaks might not be
picked up by the algorithm \code{Default=5}\#'

\item[\code{dj}] Spacing between successive scales. The CWT analyses analyses the signal using successive periods
which increase by the power of 2 (e.g.2\textasciicircum{}0=1,2\textasciicircum{}1=2,2\textasciicircum{}2=4,2\textasciicircum{}3=8,2\textasciicircum{}4=16). To have more resolution
in-between these steps the dj parameter exists, the dj parameter specifies how many extra steps/spacing in-between
the power of 2 scaled CWT is added. The amount of steps is 1/x with a higher x indicating a smaller spacing.
Increasing the increases the computational time of the CWT \code{Default=1/200}.

\item[\code{lowerPeriod}] Lowest period to be analyzed \code{Default=2}.
The CWT analyses the signal starting from the lowerPeriod to the upperPeriod so the proper selection these
parameters allows to analyze the signal for a specific range of cycles.
scaling is done using power 2 so for the best plotting results select a value to the power or 2.

\item[\code{upperPeriod}] Upper period to be analyzed \code{Default=1024}.
The CWT analyses the signal starting from the lowerPeriod to the upperPeriod so the proper selection these
parameters allows to analyze the signal for a specific range of cycles.
scaling is done using power 2 so for the best plotting results select a value to the power or 2.

\item[\code{period\_max}] Maximum period (upper boundary) to be used to extract a cycle.

\item[\code{period\_min}] Minimum period (lower boundary) to be used to extract a cycle.

\item[\code{missing\_cycle\_dur}] duration of the missing cycles

\item[\code{n\_cycles\_missing}] number of missing cycles \code{Default=1}

\item[\code{missing\_cycle\_unc}] duration uncertainty of the missing cycle

\item[\code{missing\_cycle\_unc\_dist}] distribution of the uncertainty of the
tracked cycle value need to be either "u" for uniform distribution or
"n" for normal distribution  \code{Default="u"}

\item[\code{seed\_nr}] The seed number of the Monte-Carlo simulations.
\code{Default=1337}

\item[\code{run\_multicore}] Run function using multiple cores \code{Default="FALSE"}
\end{ldescription}
\end{Arguments}
%
\begin{Value}
a vector with all the calculated gap durations
\end{Value}
\inputencoding{utf8}
\HeaderA{dynamic\_extraction}{Extract a signal in between tracked boundaries in a wavelet scalogram}{dynamic.Rul.extraction}
%
\begin{Description}
Interactively select points in a wavelet scalogram to trace the upper and
lower period of an cycle.  The \code{\LinkA{dynamic\_extraction}{dynamic.Rul.extraction}} function plots a wavelet scalogram in which points peaks can selected
allowing one to track the lower and upper period of a cycle. First track the upper or lower period of the to
be extracted cycle and then track the other boundary. Tracking points can be selected in the Interactive interface and will be shown as white dots
connected by a black line. When one wants to deselect a point the white dots can be re-clicked/re-selected and will turn red which
indicates that the previously selected point is deselected. Deselecting points can be quite tricky.
After tracking the lower and upper boundaries of the cycle the \code{\LinkA{dynamic\_extraction}{dynamic.Rul.extraction}} function
will extract the signal inbetween the boundaries. the output can then used as input for the
\code{\LinkA{minimal\_tuning}{minimal.Rul.tuning}} function to create an age model.
\end{Description}
%
\begin{Usage}
\begin{verbatim}
dynamic_extraction(
  wavelet = NULL,
  n.levels = 100,
  add_peaks = FALSE,
  periodlab = "Period (metres)",
  x_lab = "depth (metres)",
  palette_name = "rainbow",
  color_brewer = "grDevices",
  plot_horizontal = TRUE,
  smooth = FALSE,
  add_mean = TRUE
)
\end{verbatim}
\end{Usage}
%
\begin{Arguments}
\begin{ldescription}
\item[\code{wavelet}] Wavelet object created using the \code{\LinkA{analyze\_wavelet}{analyze.Rul.wavelet}} function.

\item[\code{n.levels}] Number of color levels \code{Default=100}.

\item[\code{add\_peaks}] Setting which indicates whether spectral peaks should be
added to the tracking plot  \code{Default=FALSE}.

\item[\code{periodlab}] label for the y-axis \code{Default="Period (metres)"}.

\item[\code{x\_lab}] label for the x-axis \code{Default="depth (metres)"}.

\item[\code{palette\_name}] Name of the color palette which is used for plotting.
The color palettes than can be chosen depends on which the R package is specified in
the color\_brewer parameter. The included R packages from which palettes can be chosen
from are; the 'RColorBrewer', 'grDevices', 'ColorRamps' and 'Viridis' R packages.
There are many options to choose from so please
read the documentation of these packages \code{Default=rainbow}.
The R package 'viridis' has the color palette options: “magma”, “plasma”,
“inferno”, “viridis”, “mako”, and “rocket”  and “turbo”
To see the color palette options of the The R pacakge 'RColorBrewer' run
the RColorBrewer::brewer.pal.info() function
The R package 'colorRamps' has the color palette options:"blue2green",
"blue2green2red", "blue2red",    "blue2yellow", "colorRamps",    "cyan2yellow",
"green2red", "magenta2green", "matlab.like", "matlab.like2" and    "ygobb"
The R package 'grDevices' has the built in  palette options:"rainbow",
"heat.colors", "terrain.colors","topo.colors" and "cm.colors"
To see even more color palette options of the The R pacakge 'grDevices' run
the grDevices::hcl.pals() function

\item[\code{color\_brewer}] Name of the R package from which the color palette is chosen from.
The included R packages from which palettes can be chosen
are; the RColorBrewer, grDevices, ColorRamps and Viridis R packages.
There are many options to choose from so please
read the documentation of these packages. "\code{Default=grDevices}

\item[\code{plot\_horizontal}] plot the wavelet horizontal or vertical eg y axis is depth or y axis power  \code{Default=TRUE}

\item[\code{smooth}] smooth the tracked period using the "loess\_auto" function

\item[\code{add\_mean}] add the mean to the extracted signal
\end{ldescription}
\end{Arguments}
%
\begin{Value}
Results of the tracking of a cycle in the wavelet spectra is a matrix with 3 columns.
The first column is depth/time
The second column is the extracted tracked cycle
The third column is upper tracked period
The fourth column is lower tracked period
\end{Value}
%
\begin{Author}
The function is based/inspired on the \LinkA{traceFreq}{traceFreq}
function of the 'astrochron' R package
\end{Author}
%
\begin{References}
Routines for astrochronologic testing, astronomical time scale construction, and
time series analysis <doi:10.1016/j.earscirev.2018.11.015>
\end{References}
%
\begin{Examples}
\begin{ExampleCode}

#Track the 405kyr upper and lower periods of the eccentricity cycle in the
magnetic susceptibility record of the Sullivan core of Pas et al., (2018)

mag_wt <- analyze_wavelet(
 data = mag,
 dj = 1 / 100,
 lowerPeriod = 0.1,
 upperPeriod = 254,
 verbose = FALSE,
 omega_nr = 10
)

mag_ext <- dynamic_extraction(
 wavelet = mag_wt,
 n.levels = 100,
 add_peaks = FALSE,
 periodlab = "Period (metres)",
 x_lab = "depth (metres)",
 palette_name = "rainbow",
 color_brewer = "grDevices",
 plot_horizontal = TRUE,
 smooth = TRUE,
 add_mean = TRUE
)


\end{ExampleCode}
\end{Examples}
\inputencoding{utf8}
\HeaderA{extract\_amplitude}{Extract amplitude from a signal}{extract.Rul.amplitude}
%
\begin{Description}
Extracts the amplitude from a signal using the continuous wavelet transform using a Morlet wavelet.
The extraction of the amplitude is useful for cyclostratigraphic studies because the amplitude of
an astronomical cycle is modulated by higher order astronomical cycles.
\end{Description}
%
\begin{Usage}
\begin{verbatim}
extract_amplitude(
  signal = NULL,
  pts = 3,
  genplot = FALSE,
  remean = TRUE,
  ver_results = FALSE,
  keep_editable = FALSE
)
\end{verbatim}
\end{Usage}
%
\begin{Arguments}
\begin{ldescription}
\item[\code{signal}] Input signal from which the amplitude is extracted any signal in which the first column is
depth/time and the second column is the proxy record from which the amplitude is extracted

\item[\code{pts}] The pts parameter specifies how many points to the left/right up/down the peak detect algorithm goes in detecting
a peak. The peak detecting algorithm works by comparing the values left/right up/down of it, if the values are both higher or lower
then the value a peak. To deal with error produced by this algorithm the pts parameter can be changed which can
aid in peak detection. Usually increasing the pts parameter means more peak certainty, however it also means that minor peaks might not be
picked up by the algorithm \code{Default=3}

\item[\code{genplot}] If set to TRUE a plot with extracted amplitude will be displayed \code{Default=FALSE}.

\item[\code{remean}] Prior to analysis the mean is subtracted from the data set to re-mean set \code{Default=TRUE}.

\item[\code{ver\_results}] To verify the amplitude extraction is representative of the amplitude
extracted using the \code{\LinkA{extract\_amplitude}{extract.Rul.amplitude}} function the results can be compared to the amplitude extracted
using the \code{\LinkA{Hilbert\_transform}{Hilbert.Rul.transform}} if the mean difference is more then 5
whether the input contains a reliable enough signal with high a enough amplitude modulation to actually extract an amplitude from. \code{Default=FALSE}.

\item[\code{keep\_editable}] Keep option to add extra features after plotting  \code{Default=FALSE}
\end{ldescription}
\end{Arguments}
%
\begin{Value}
Returns a matrix with 2 columns.
The first column is depth/time.
The second column is the extracted amplitude
\end{Value}
%
\begin{Author}
Code based on the \LinkA{reconstruct}{reconstruct} function of the 'WaveletComp' R package
which is based on the wavelet 'MATLAB' code written by Christopher Torrence and Gibert P. Compo.
The assignment of the standard deviation of the uncertainty of the wavelet
is based on the work of Gabor (1946) and Russell et al., (2016)
\end{Author}
%
\begin{References}
Angi Roesch and Harald Schmidbauer (2018). WaveletComp: Computational
Wavelet Analysis. R package version 1.1.
\url{https://CRAN.R-project.org/package=WaveletComp}

Gouhier TC, Grinsted A, Simko V (2021). R package biwavelet: Conduct Univariate and Bivariate Wavelet Analyses. (Version 0.20.21),
\url{https://github.com/tgouhier/biwavelet}

Torrence, C., and G. P. Compo. 1998. A Practical Guide to Wavelet Analysis.
Bulletin of the American Meteorological Society 79:61-78.
\url{https://paos.colorado.edu/research/wavelets/bams_79_01_0061.pdf}

Morlet, Jean, Georges Arens, Eliane Fourgeau, and Dominique Glard.
"Wave propagation and sampling theory—Part I: Complex signal and scattering in multilayered media.
" Geophysics 47, no. 2 (1982): 203-221.
\url{https://pubs.geoscienceworld.org/geophysics/article/47/2/203/68601/Wave-propagation-and-sampling-theory-Part-I}

J. Morlet, G. Arens, E. Fourgeau, D. Giard;
Wave propagation and sampling theory; Part II, Sampling theory and complex waves.
Geophysics 1982 47 (2): 222–236. \url{https://pubs.geoscienceworld.org/geophysics/article/47/2/222/68604/Wave-propagation-and-sampling-theory-Part-II}
\end{References}
%
\begin{Examples}
\begin{ExampleCode}

#Extract amplitude of the 405 kyr eccentricity cycle from the the magnetic
# susceptibility data set of De pas et al., (2018)
#Perform the CWT on the magnetic susceptibility data set of Pas et al., (2018)

mag_wt <- analyze_wavelet(data = mag,
dj = 1/100,
lowerPeriod = 0.1,
upperPeriod = 254,
verbose = FALSE,
omega_nr = 10)

#Track the 405 kyr eccentricity cycle in a wavelet spectra

#mag_track <- track_period_wavelet(astro_cycle = 405,
#                                   wavelet=mag_wt,
#                                   n.levels = 100,
#                                   periodlab = "Period (metres)",
#                                   x_lab = "depth (metres)")

#Instead of tracking, the tracked solution data set mag_track_solution
#is used
mag_track <- mag_track_solution

mag_track_complete <- completed_series(
  wavelet = mag_wt,
  tracked_curve = mag_track,
  period_up = 1.2,
  period_down = 0.8,
  extrapolate = TRUE,
  genplot = FALSE
)

#Smooth the completed tracking of the 405 kyr eccentricity cycle in the wavelet spectra

mag_track_complete <- loess_auto(time_series = mag_track_complete,
genplot = FALSE, print_span = FALSE)




mag_405_ecc  <- extract_signal(
tracked_cycle_curve = mag_track_complete,
wavelet = mag_wt,
period_up = 1.2,
period_down = 0.8,
add_mean = TRUE,
tracked_cycle_period = 405,
extract_cycle = 405,
tune = FALSE,
plot_residual = FALSE
)

#extract the amplitude  of the 405 kyr eccentricity cycle
mag_ampl <- extract_amplitude(
signal = mag_405_ecc,
pts=3,
genplot = FALSE,
ver_results = FALSE,
keep_editable=FALSE)


\end{ExampleCode}
\end{Examples}
\inputencoding{utf8}
\HeaderA{extract\_power}{Extract power from a wavelet spectra}{extract.Rul.power}
%
\begin{Description}
Extracts the  spectral power from a wavelet spectra in the depth domain using a traced period
and boundaries surround the traced period.
The extraction of spectral is useful for cyclostratigraphic studies because the spectral power of an
astronomical cycle is modulated by higher order astronomical cycles.
The spectral power record from an astronomical cycle can thus be used as a proxy for
amplitude modulating cycles
The traced period result from the \code{\LinkA{track\_period\_wavelet}{track.Rul.period.Rul.wavelet}}
function with boundaries is used to extract spectral power in the depth domain from a wavelet spectra.
\end{Description}
%
\begin{Usage}
\begin{verbatim}
extract_power(
  completed_series = NULL,
  wavelet = NULL,
  period_up = 1.2,
  period_down = 0.8,
  tracked_cycle_period = NULL,
  extract_cycle_power = NULL
)
\end{verbatim}
\end{Usage}
%
\begin{Arguments}
\begin{ldescription}
\item[\code{completed\_series}] Traced period result from the \code{\LinkA{track\_period\_wavelet}{track.Rul.period.Rul.wavelet}}
function completed using the \code{\LinkA{completed\_series}{completed.Rul.series}}.
The input can be pre-smoothed using the the \code{\LinkA{loess\_auto}{loess.Rul.auto}} function.

\item[\code{wavelet}] Wavelet object created using the \code{\LinkA{analyze\_wavelet}{analyze.Rul.wavelet}} function.

\item[\code{period\_up}] Upper period as a factor of the to be extracted power \code{Default=1.2}.

\item[\code{period\_down}] Lower period as a factor of the to be extracted power \code{Default=0.8}.

\item[\code{tracked\_cycle\_period}] Period of the tracked cycle (make sure that
\code{tracked\_cycle\_period}) and \code{extract\_cycle\_power}) are of the same
unit (either depth or time domain).

\item[\code{extract\_cycle\_power}] Period of the cycle for which the power will be
extracted (make sure that \code{extract\_cycle\_power}) and
\code{tracked\_cycle\_period}) are of the same unit (either depth or time domain).
\end{ldescription}
\end{Arguments}
%
\begin{Value}
Returns a matrix with 3 columns.
The first column is depth/time.
The second column is extracted power.
The third column is extracted power/total power.
\end{Value}
%
\begin{Author}
Code based on the \LinkA{reconstruct}{reconstruct} function of the 'WaveletComp' R package
which is based on the wavelet 'MATLAB' code written by Christopher Torrence and Gibert P. Compo.
The assignment of the standard deviation of the uncertainty of the wavelet
is based on the work of Gabor (1946) and Russell et al., (2016)
The functionality of this function is is inspired by the
\LinkA{integratePower}{integratePower} function of the 'astrochron' R package.
\end{Author}
%
\begin{References}
Angi Roesch and Harald Schmidbauer (2018). WaveletComp: Computational
Wavelet Analysis. R package version 1.1.
\url{https://CRAN.R-project.org/package=WaveletComp}

Gouhier TC, Grinsted A, Simko V (2021). R package biwavelet: Conduct Univariate and Bivariate Wavelet Analyses. (Version 0.20.21),
\url{https://github.com/tgouhier/biwavelet}

Torrence, C., and G. P. Compo. 1998. A Practical Guide to Wavelet Analysis.
Bulletin of the American Meteorological Society 79:61-78.
\url{https://paos.colorado.edu/research/wavelets/bams_79_01_0061.pdf}

Routines for astrochronologic testing, astronomical time scale construction, and
time series analysis <doi:10.1016/j.earscirev.2018.11.015>
\end{References}
%
\begin{Examples}
\begin{ExampleCode}
#Extract the power of the 405 kyr eccentricity cycle from the the magnetic
# susceptibility data set of De pas et al., (2018)
#Perform the CWT on the magnetic susceptibility data set of Pas et al., (2018)

mag_wt <- analyze_wavelet(data = mag,
dj = 1/100,
lowerPeriod = 0.1,
upperPeriod = 254,
verbose = FALSE,
omega_nr = 10)

#Track the 405 kyr eccentricity cycle in a wavelet spectra

#mag_track <- track_period_wavelet(astro_cycle = 405,
#                                   wavelet=mag_wt,
#                                   n.levels = 100,
#                                   periodlab = "Period (metres)",
#                                   x_lab = "depth (metres)")

#Instead of tracking, the tracked solution data set mag_track_solution
#is used
mag_track <- mag_track_solution

mag_track_complete <- completed_series(
  wavelet = mag_wt,
  tracked_curve = mag_track,
  period_up = 1.2,
  period_down = 0.8,
  extrapolate = TRUE,
  genplot = FALSE
)

#Smooth the completed tracking of the 405 kyr eccentricity cycle in the wavelet spectra

mag_track_complete <- loess_auto(time_series = mag_track_complete,
genplot = FALSE, print_span = FALSE)

#extract the spectral power of the 405 kyr eccentricity cycle
mag_power <- extract_power(
completed_series = mag_track_complete,
wavelet = mag_wt,
period_up = 1.2,
period_down = 0.8,
tracked_cycle_period = 405,
extract_cycle_power = 405
)

\end{ExampleCode}
\end{Examples}
\inputencoding{utf8}
\HeaderA{extract\_power\_stable}{Extract power from a wavelet spectra by using a constant period/duration}{extract.Rul.power.Rul.stable}
%
\begin{Description}
Extract spectral power from the wavelet using a constant period/duration and
boundaries as selection criteria. The extraction of spectral is useful for cyclostratigraphic studies because the spectral power of an
astronomical cycle is modulated by higher order astronomical cycles.
The spectral power record from an astronomical cycle can thus be used as a proxy for
amplitude modulating cycles. The spectral power is extracted from a wavelet spectra
which was created using the \code{\LinkA{analyze\_wavelet}{analyze.Rul.wavelet}}
function for a given, \code{cycle}, \code{period\_up} and \code{period\_down}
\end{Description}
%
\begin{Usage}
\begin{verbatim}
extract_power_stable(
  wavelet = NULL,
  cycle = NULL,
  period_up = 1.2,
  period_down = 0.8
)
\end{verbatim}
\end{Usage}
%
\begin{Arguments}
\begin{ldescription}
\item[\code{wavelet}] Wavelet object created using the \code{\LinkA{analyze\_wavelet}{analyze.Rul.wavelet}} function.

\item[\code{cycle}] Period of cycle for which the power will be extracted from the record.

\item[\code{period\_up}] Species the upper period of the to be extracted power \code{Default=1.2}.

\item[\code{period\_down}] specifies the lower period of the to be extracted power \code{Default=0.8}.
\end{ldescription}
\end{Arguments}
%
\begin{Value}
Returns a matrix with 3 columns.
The first column is depth/time.
The second column is extracted power.
The third column is extracted power/total power.
\end{Value}
%
\begin{Author}
Code based on the \LinkA{reconstruct}{reconstruct} function of the 'WaveletComp' R package
which is based on the wavelet 'MATLAB' code written by Christopher Torrence and Gibert P. Compo (1998).
The functionality of this function is is inspired by the
\LinkA{integratePower}{integratePower} function of the 'astrochron' R package
\end{Author}
%
\begin{References}
Angi Roesch and Harald Schmidbauer (2018). WaveletComp: Computational
Wavelet Analysis. R package version 1.1.
\url{https://CRAN.R-project.org/package=WaveletComp}

Gouhier TC, Grinsted A, Simko V (2021). R package biwavelet: Conduct Univariate and Bivariate Wavelet Analyses. (Version 0.20.21),
\url{https://github.com/tgouhier/biwavelet}

Torrence, C., and G. P. Compo. 1998. A Practical Guide to Wavelet Analysis.
Bulletin of the American Meteorological Society 79:61-78.
\url{https://paos.colorado.edu/research/wavelets/bams_79_01_0061.pdf}

Routines for astrochronologic testing, astronomical time scale construction, and
time series analysis <doi:10.1016/j.earscirev.2018.11.015>
\end{References}
%
\begin{Examples}
\begin{ExampleCode}
#Extract the spectral power of the 210 yr de Vries cycle from the Total Solar
#Irradiance data set of Steinhilber et al., (2012).

TSI_wt <-
 analyze_wavelet(
   data = TSI,
   dj = 1/200,
   lowerPeriod = 16,
   upperPeriod = 8192,
   verbose = FALSE,
   omega_nr = 6
 )
TSI_wt_pwr_de_Vries_cycle <-  extract_power_stable(
 wavelet = TSI_wt,
 cycle = 210,
 period_up = 1.2,
 period_down = 0.8
)


\end{ExampleCode}
\end{Examples}
\inputencoding{utf8}
\HeaderA{extract\_signal}{Extract signal from a wavelet spectra using a traced period curve}{extract.Rul.signal}
%
\begin{Description}
Extract signal power from the wavelet in the depth domain using the traced period.
\end{Description}
%
\begin{Usage}
\begin{verbatim}
extract_signal(
  tracked_cycle_curve = NULL,
  wavelet = NULL,
  period_up = 1.2,
  period_down = 0.8,
  add_mean = TRUE,
  tracked_cycle_period = NULL,
  extract_cycle = NULL,
  tune = FALSE,
  plot_residual = FALSE
)
\end{verbatim}
\end{Usage}
%
\begin{Arguments}
\begin{ldescription}
\item[\code{tracked\_cycle\_curve}] Traced period result from the \code{track\_period\_wavelet}
function completed using the \code{completed\_series}.
The input can be pre-smoothed using the the \code{loess\_auto} function.

\item[\code{wavelet}] wavelet object created using the \code{\LinkA{analyze\_wavelet}{analyze.Rul.wavelet}} function.

\item[\code{period\_up}] Upper period as a factor of the to be extracted cycle \code{Default=1.2}.

\item[\code{period\_down}] Lower period as a factor of the to be extracted cycle \code{Default=0.8}.

\item[\code{add\_mean}] Add mean to the extracted cycle \code{Default=TRUE}.

\item[\code{tracked\_cycle\_period}] Period in time of the traced cycle.

\item[\code{extract\_cycle}] Period of the to be extracted cycle.

\item[\code{tune}] Convert record from the depth to the time domain using the traced period \code{Default=FALSE}.

\item[\code{plot\_residual}] Plot the residual signal after extraction of set cycle \code{Default=FALSE}.
\end{ldescription}
\end{Arguments}
%
\begin{Value}
Returns a matrix with 2 columns
The first column is depth/time
The second column is extracted signal
\end{Value}
%
\begin{Author}
Code based on the \LinkA{reconstruct}{reconstruct} function of the 'WaveletComp' R package
which is based on the wavelet 'MATLAB' code written by Christopher Torrence and Gibert P. Compo (1998).
\end{Author}
%
\begin{References}
Angi Roesch and Harald Schmidbauer (2018). WaveletComp: Computational
Wavelet Analysis. R package version 1.1.
\url{https://CRAN.R-project.org/package=WaveletComp}

Gouhier TC, Grinsted A, Simko V (2021). R package biwavelet: Conduct Univariate and Bivariate Wavelet Analyses. (Version 0.20.21),
\url{https://github.com/tgouhier/biwavelet}

Torrence, C., and G. P. Compo. 1998. A Practical Guide to Wavelet Analysis.
Bulletin of the American Meteorological Society 79:61-78.
\url{https://paos.colorado.edu/research/wavelets/bams_79_01_0061.pdf}
\end{References}
%
\begin{Examples}
\begin{ExampleCode}

#Extract the 405 kyr eccentricity cycle from the the magnetic susceptibility \cr
#record of the Sullivan core and use the Gabor uncertainty principle to define \cr
#the mathematical uncertainty of the analysis and use a factor of that standard \cr
#deviation to define boundaries.

#Perform the CWT
mag_wt <- analyze_wavelet(data = mag,
dj = 1/100,
lowerPeriod = 0.1,
upperPeriod = 254,
verbose = FALSE,
omega_nr = 10)

#Track the 405 kyr eccentricity cycle in a wavelet spectra

#mag_track <- track_period_wavelet(astro_cycle = 405,
#                                   wavelet=mag_wt,
#                                   n.levels = 100,
#                                   periodlab = "Period (metres)",
#                                   x_lab = "depth (metres)")

#Instead of tracking, the tracked solution data set \code{\link{mag_track_solution}} is used \cr
mag_track <- mag_track_solution


mag_track_complete <- completed_series(
  wavelet = mag_wt,
  tracked_curve = mag_track,
  period_up = 1.2,
  period_down = 0.8,
  extrapolate = TRUE,
  genplot = FALSE
)

# smooth the tracking of the 405 kyr eccentricity cycle
mag_track_complete <- loess_auto(time_series = mag_track_complete,
genplot = FALSE, print_span = FALSE)

# extract the 405 kyr eccentricity cycle from the wavelet spectrum and use the \cr
# tracked cycle curve and set factors of the extracted cycle as boundaries

mag_405_ecc  <- extract_signal(
tracked_cycle_curve = mag_track_complete,
wavelet = mag_wt,
period_up = 1.2,
period_down = 0.8,
add_mean = TRUE,
tracked_cycle_period = 405,
extract_cycle = 405,
tune = FALSE,
plot_residual = FALSE
)

\end{ExampleCode}
\end{Examples}
\inputencoding{utf8}
\HeaderA{extract\_signal\_stable}{Extract a signal/cycle from a wavelet spectra using a set period and boundaries}{extract.Rul.signal.Rul.stable}
%
\begin{Description}
Extracts a cycle from the wavelet object created using the \code{\LinkA{analyze\_wavelet}{analyze.Rul.wavelet}}
function using a fixed period and fixed period boundaries defined as factors of the original cycle
\end{Description}
%
\begin{Usage}
\begin{verbatim}
extract_signal_stable(
  wavelet = NULL,
  cycle = NULL,
  period_up = 1.2,
  period_down = 0.8,
  add_mean = TRUE,
  plot_residual = FALSE,
  keep_editable = FALSE
)
\end{verbatim}
\end{Usage}
%
\begin{Arguments}
\begin{ldescription}
\item[\code{wavelet}] Wavelet object created using the \code{\LinkA{analyze\_wavelet}{analyze.Rul.wavelet}} function.

\item[\code{cycle}] Period of the cycle which needs to be extracted.

\item[\code{period\_up}] Specifies the upper period as a factor of the to be extracted cycle \code{Default=1.2}.

\item[\code{period\_down}] Specifies the lower period as a factor of the to be extracted cycle \code{Default=0.8}.

\item[\code{add\_mean}] Add mean to the extracted cycle \code{Default=TRUE}.

\item[\code{plot\_residual}] plot the residual signal after extraction of set cycle \code{Default=FALSE}.

\item[\code{keep\_editable}] Keep option to add extra features after plotting  \code{Default=FALSE}
\end{ldescription}
\end{Arguments}
%
\begin{Value}
\#'Returns a matrix with 2 columns.
The first column is time/depth.
The second column is the extracted signal/cycle.
\end{Value}
%
\begin{Author}
Code based on the \LinkA{reconstruct}{reconstruct} function of the 'WaveletComp' R package
which is based on the wavelet 'MATLAB' code written by Christopher Torrence and Gibert P. Compo (1998).
\end{Author}
%
\begin{References}
Angi Roesch and Harald Schmidbauer (2018). WaveletComp: Computational
Wavelet Analysis. R package version 1.1.
\url{https://CRAN.R-project.org/package=WaveletComp}

Gouhier TC, Grinsted A, Simko V (2021). R package biwavelet: Conduct Univariate and Bivariate Wavelet Analyses. (Version 0.20.21),
\url{https://github.com/tgouhier/biwavelet}

Torrence, C., and G. P. Compo. 1998. A Practical Guide to Wavelet Analysis.
Bulletin of the American Meteorological Society 79:61-78.
\url{https://paos.colorado.edu/research/wavelets/bams_79_01_0061.pdf}
\end{References}
%
\begin{Examples}
\begin{ExampleCode}
#Example in which the ~210yr de Vries cycle is extracted from the Total Solar
#Irradiance data set of Steinhilber et al., (2012)

#Perform the CWT
TSI_wt <-
analyze_wavelet(
data = TSI,
dj = 1/200,
lowerPeriod = 16,
upperPeriod = 8192,
   verbose = FALSE,
   omega_nr = 6
 )

#Extract the 210 yr de Vries cycle from the wavelet spectra
de_Vries_cycle <- extract_signal_stable(wavelet=TSI_wt,
cycle=210,
period_up =1.25,
period_down = 0.75,
add_mean=TRUE,
plot_residual=FALSE,
keep_editable=FALSE)

\end{ExampleCode}
\end{Examples}
\inputencoding{utf8}
\HeaderA{extract\_signal\_stable\_V2}{Extract signal from a wavelet spectrum using a upper and lower period boundary}{extract.Rul.signal.Rul.stable.Rul.V2}
%
\begin{Description}
Extract a signal from the wavelet using a upper and lower period boundary
\end{Description}
%
\begin{Usage}
\begin{verbatim}
extract_signal_stable_V2(
  wavelet = NULL,
  period_max = NULL,
  period_min = NULL,
  add_mean = TRUE,
  plot_residual = FALSE,
  keep_editable = FALSE
)
\end{verbatim}
\end{Usage}
%
\begin{Arguments}
\begin{ldescription}
\item[\code{wavelet}] wavelet object created using the \code{\LinkA{analyze\_wavelet}{analyze.Rul.wavelet}} function.

\item[\code{period\_max}] Maximum period (upper boundary) to be used to extract a cycle.

\item[\code{period\_min}] Minimum period (lower boundary) to be used to extract a cycle.

\item[\code{add\_mean}] Add mean to the extracted cycle \code{Default=TRUE}.

\item[\code{plot\_residual}] Plot the signal from which the extracted cycle is subtracted \code{Default=FALSE}.

\item[\code{keep\_editable}] Keep option to add extra features after plotting  \code{Default=FALSE}
\end{ldescription}
\end{Arguments}
%
\begin{Value}
Signal extracted from the wavelet spectra.
Output is a matrix with the first column being depth/time
and the second column is the cycle extracted from the proxy record.
\end{Value}
%
\begin{Author}
Code based on the \LinkA{reconstruct}{reconstruct} function of the 'WaveletComp' R package
which is based on the wavelet 'MATLAB' code written by Christopher Torrence and Gibert P. Compo (1998).
\end{Author}
%
\begin{References}
Angi Roesch and Harald Schmidbauer (2018). WaveletComp: Computational
Wavelet Analysis. R package version 1.1.
\url{https://CRAN.R-project.org/package=WaveletComp}

Gouhier TC, Grinsted A, Simko V (2021). R package biwavelet: Conduct Univariate and Bivariate Wavelet Analyses. (Version 0.20.21),
\url{https://github.com/tgouhier/biwavelet}

Torrence, C., and G. P. Compo. 1998. A Practical Guide to Wavelet Analysis.
Bulletin of the American Meteorological Society 79:61-78.
\url{https://paos.colorado.edu/research/wavelets/bams_79_01_0061.pdf}
\end{References}
%
\begin{Examples}
\begin{ExampleCode}
#Example in which the ~210yr de Vries cycle is extracted from the Total Solar
# Irradiance data set of Steinhilber et al., (2012)

TSI_wt <-
analyze_wavelet(
data = TSI,
dj = 1/200,
lowerPeriod = 16,
upperPeriod = 8192,
   verbose = FALSE,
   omega_nr = 6
 )

de_Vries_cycle <- extract_signal_stable_V2(wavelet=TSI_wt,
period_max = 240,
period_min = 180,
add_mean=TRUE,
plot_residual=FALSE,
keep_editable=FALSE)


\end{ExampleCode}
\end{Examples}
\inputencoding{utf8}
\HeaderA{extract\_signal\_standard\_deviation}{Extract a signal using standard deviation}{extract.Rul.signal.Rul.standard.Rul.deviation}
%
\begin{Description}
Extract signal from a
wavelet spectra in the depth domain using a the standard deviation of the omega (number of cycles)
as boundaries. The uncertainty is based on the Gabor uncertainty principle applied to the
continuous wavelet transform using a Morlet wavelet. The calculated uncertainty is the underlying
analytical uncertainty which is the result of applying the Gabor uncertainty principle to the
continuous wavelet transform using a Morlet wavelet.
\end{Description}
%
\begin{Usage}
\begin{verbatim}
extract_signal_standard_deviation(
  wavelet = NULL,
  tracked_cycle_curve = NULL,
  multi = 1,
  extract_cycle = NULL,
  tracked_cycle_period = NULL,
  add_mean = TRUE,
  tune = FALSE,
  genplot_uncertainty_wt = FALSE,
  genplot_extracted = FALSE,
  keep_editable = FALSE,
  palette_name = "rainbow",
  color_brewer = "grDevices"
)
\end{verbatim}
\end{Usage}
%
\begin{Arguments}
\begin{ldescription}
\item[\code{wavelet}] Wavelet object created using the \code{\LinkA{analyze\_wavelet}{analyze.Rul.wavelet}} function.

\item[\code{tracked\_cycle\_curve}] Curve of the cycle tracked using the
\code{\LinkA{track\_period\_wavelet}{track.Rul.period.Rul.wavelet}} function. Any input (matrix or data frame)
in which the first column is depth or time and the second column is period should work.

\item[\code{multi}] multiple of the standard deviation to be used as boundaries for the cycle extraction
\code{Default=1}.

\item[\code{extract\_cycle}] Period of the cycle to be extracted.

\item[\code{tracked\_cycle\_period}] Period of the tracked cycle.

\item[\code{add\_mean}] Add mean to the extracted cycle \code{Default=TRUE}.

\item[\code{tune}] Tune data set using the \code{Default=tracked\_cycle\_curve} curve \code{Default=FALSE}.

\item[\code{genplot\_uncertainty\_wt}] Generate a wavelet spectra plot with the tracked curve and its
analytical uncertainty based the Gabor uncertainty principle applied
continuous wavelet transform using a Morlet wavelet on superimposed on top of it.
In the plot the red curve and blue curves are the upper and lower bounds
based on the \code{multi} parameter which x-times the standard deviation of uncertainty.
The black curve is the \code{Default=FALSE} curve.

\item[\code{genplot\_extracted}] Generates a plot with the data set and
the extracted cycle on top \code{Default=FALSE} of it.

\item[\code{keep\_editable}] Keep option to add extra features after plotting  \code{Default=FALSE}

\item[\code{palette\_name}] Name of the color palette which is used for plotting.
The color palettes than can be chosen depends on which the R package is specified in
the color\_brewer parameter. The included R packages from which palettes can be chosen
from are; the 'RColorBrewer', 'grDevices', 'ColorRamps' and 'Viridis' R packages.
There are many options to choose from so please
read the documentation of these packages \code{Default=rainbow}.
The R package 'viridis' has the color palette options: “magma”, “plasma”,
“inferno”, “viridis”, “mako”, and “rocket”  and “turbo”
To see the color palette options of the The R pacakge 'RColorBrewer' run
the RColorBrewer::brewer.pal.info() function
The R package 'colorRamps' has the color palette options:"blue2green",
"blue2green2red", "blue2red",    "blue2yellow", "colorRamps",    "cyan2yellow",
"green2red", "magenta2green", "matlab.like", "matlab.like2" and    "ygobb"
The R package 'grDevices' has the built in  palette options:"rainbow",
"heat.colors", "terrain.colors","topo.colors" and "cm.colors"
To see even more color palette options of the The R pacakge 'grDevices' run
the grDevices::hcl.pals() function

\item[\code{color\_brewer}] Name of the R package from which the color palette is chosen from.
The included R packages from which palettes can be chosen
are; the RColorBrewer, grDevices, ColorRamps and Viridis R packages.
There are many options to choose from so please
read the documentation of these packages. "\code{Default=grDevices}
\end{ldescription}
\end{Arguments}
%
\begin{Value}
Signal extracted from the wavelet spectra.
Output is a matrix with the first column being depth/time
and the second column is the astronomical cycle extracted from the proxy record

If \code{genplot\_uncertainty\_wt=TRUE} then a wavelet spectra will be plotted
with the uncertainty superimposed on top of it. In the plot the red curve and
blue curves are the upper and lower bounds
based on the \code{multi} parameter.The black curve is the \code{Default=tracked\_cycle\_curve} curve.
If \code{genplot\_extracted=TRUE} plot with the data set and
the extracted cycle on top of it will be plotted.
\end{Value}
%
\begin{Author}
Code based on the \LinkA{reconstruct}{reconstruct} function of the 'WaveletComp' R package
which is based on the wavelet 'MATLAB' code written by Christopher Torrence and Gibert P. Compo (1998).
The assignment of the standard deviation of the uncertainty of the wavelet
is based on the work of Gabor (1946) and Russell et al., (2016)
\end{Author}
%
\begin{References}
Angi Roesch and Harald Schmidbauer (2018). WaveletComp: Computational
Wavelet Analysis. R package version 1.1.
\url{https://CRAN.R-project.org/package=WaveletComp}

Gouhier TC, Grinsted A, Simko V (2021). R package biwavelet: Conduct Univariate and Bivariate Wavelet Analyses. (Version 0.20.21),
\url{https://github.com/tgouhier/biwavelet}

Torrence, C., and G. P. Compo. 1998. A Practical Guide to Wavelet Analysis.
Bulletin of the American Meteorological Society 79:61-78.
\url{https://paos.colorado.edu/research/wavelets/bams_79_01_0061.pdf}

Gabor, Dennis. "Theory of communication. Part 1: The analysis of information."
Journal of the Institution of Electrical Engineers-part III: radio and
communication engineering 93, no. 26 (1946): 429-441.\url{http://genesis.eecg.toronto.edu/gabor1946.pdf}

Russell, Brian, and Jiajun Han. "Jean Morlet and the continuous wavelet transform.
" CREWES Res. Rep 28 (2016): 115. \url{https://www.crewes.org/Documents/ResearchReports/2016/CRR201668.pdf}


Morlet, Jean, Georges Arens, Eliane Fourgeau, and Dominique Glard.
"Wave propagation and sampling theory—Part I: Complex signal and scattering in multilayered media.
" Geophysics 47, no. 2 (1982): 203-221.
\url{https://pubs.geoscienceworld.org/geophysics/article/47/2/203/68601/Wave-propagation-and-sampling-theory-Part-I}

J. Morlet, G. Arens, E. Fourgeau, D. Giard;
Wave propagation and sampling theory; Part II, Sampling theory and complex waves.
Geophysics 1982 47 (2): 222–236. \url{https://pubs.geoscienceworld.org/geophysics/article/47/2/222/68604/Wave-propagation-and-sampling-theory-Part-II}
\end{References}
%
\begin{Examples}
\begin{ExampleCode}

#Extract the 405 kyr eccentricity cycle from the magnetic susceptibility
#record of the Sullivan core of Pas et al., (2018) and use the Gabor
# uncertainty principle to define the mathematical uncertainty of the
# analysis and use a factor of that standard deviation to define
# boundaries

# perform the CWT
mag_wt <- analyze_wavelet(data = mag,
dj = 1/100,
lowerPeriod = 0.1,
upperPeriod = 254,
verbose = FALSE,
omega_nr = 10)

#Track the 405 kyr eccentricity cycle in a wavelet spectra

#mag_track <- track_period_wavelet(astro_cycle = 405,
#                                   wavelet=mag_wt,
#                                   n.levels = 100,
#                                   periodlab = "Period (metres)",
#                                   x_lab = "depth (metres)",
#                                   palette_name="rainbow",
#                                   color_brewer="grDevices")

#Instead of tracking, the tracked solution data set mag_track_solution is used
mag_track <- mag_track_solution

mag_track_complete <- completed_series(
  wavelet = mag_wt,
  tracked_curve = mag_track,
  period_up = 1.2,
  period_down = 0.8,
  extrapolate = TRUE,
  genplot = FALSE
)

# smooth the tracking of the 405 kyr eccentricity cycle
mag_track_complete <- loess_auto(time_series = mag_track_complete,
genplot = FALSE, print_span = FALSE)

# extract the 405 kyr eccentricity cycle from the wavelet spectrum and use
# the Gabor uncertainty principle to define the mathematical uncertainty of
# the analysis and use a multiple of the derived standard deviation to define boundaries

mag_405_ecc <- extract_signal_standard_deviation(
wavelet = mag_wt,
tracked_cycle_curve = mag_track_complete,
multi = 1,
extract_cycle = 405,
tracked_cycle_period = 405,
add_mean = TRUE,
tune = FALSE,
genplot_uncertainty_wt = FALSE,
genplot_extracted = FALSE,
keep_editable=FALSE,
palette_name="rainbow",
color_brewer="grDevices"
)

\end{ExampleCode}
\end{Examples}
\inputencoding{utf8}
\HeaderA{flmw}{Fit linear models to spectral peaks extracted from the wavelet spectra to astronomical cycles multiplied by sedimentation rate x}{flmw}
%
\begin{Description}
The \code{\LinkA{flmw}{flmw}} function is used calculate the linear correlation
for a list of astronomical cycles transformed using a range of sedimentation rates and then compared
to spectral peaks of a wavelet spectra
\end{Description}
%
\begin{Usage}
\begin{verbatim}
flmw(
  wavelet = NULL,
  sedrate_low = NULL,
  sedrate_high = NULL,
  spacing = NULL,
  cycles = c(NULL),
  x_lab = "depth",
  y_lab = "sedrate",
  run_random = FALSE,
  rand_simulations = 1000,
  run_multicore = FALSE,
  genplot = FALSE,
  palette_name = "rainbow",
  color_brewer = "grDevices",
  plot_res = 2,
  keep_editable = FALSE,
  verbose = FALSE
)
\end{verbatim}
\end{Usage}
%
\begin{Arguments}
\begin{ldescription}
\item[\code{wavelet}] Wavelet object created using the \code{\LinkA{analyze\_wavelet}{analyze.Rul.wavelet}} function

\item[\code{sedrate\_low}] Minimum sedimentation rate (cm/kyr)for which the sum of maximum spectral power is calculated for.

\item[\code{sedrate\_high}] Maximum sedimentation rate (cm/kyr) for which the sum of maximum spectral power is calculated  for.

\item[\code{spacing}] Spacing (cm/kyr) between sedimentation rates

\item[\code{cycles}] Astronomical cycles (in kyr) for which the combined sum of maximum spectral power is calculated for

\item[\code{x\_lab}] label for the y-axis \code{Default="depth"}

\item[\code{y\_lab}] label for the y-axis \code{Default="sedrate"}

\item[\code{run\_random}] run multiple simulation to calculate percentile against the 0 hypothesis

\item[\code{rand\_simulations}] nr of simulations to calculate percentile against the 0 hypothesis

\item[\code{run\_multicore}] run simulation using multiple cores \code{Default=FALSE}
the simulation is run at x-2 cores to allow the 2 remaining processes to run background processes

\item[\code{genplot}] Generate plot \code{Default="FALSE"}

\item[\code{palette\_name}] Name of the color palette which is used for plotting.
The color palettes than can be chosen depends on which the R package is specified in
the color\_brewer parameter. The included R packages from which palettes can be chosen
from are; the 'RColorBrewer', 'grDevices', 'ColorRamps' and 'Viridis' R packages.
There are many options to choose from so please
read the documentation of these packages \code{Default=rainbow}.
The R package 'viridis' has the color palette options: “magma”, “plasma”,
“inferno”, “viridis”, “mako”, and “rocket”  and “turbo”
To see the color palette options of the The R pacakge 'RColorBrewer' run
the RColorBrewer::brewer.pal.info() function
The R package 'colorRamps' has the color palette options:"blue2green",
"blue2green2red", "blue2red",    "blue2yellow", "colorRamps",    "cyan2yellow",
"green2red", "magenta2green", "matlab.like", "matlab.like2" and    "ygobb"
The R package 'grDevices' has the built in  palette options:"rainbow",
"heat.colors", "terrain.colors","topo.colors" and "cm.colors"
To see even more color palette options of the The R pacakge 'grDevices' run
the grDevices::hcl.pals() function

\item[\code{color\_brewer}] Name of the R package from which the color palette is chosen from.
The included R packages from which palettes can be chosen
are; the RColorBrewer, grDevices, ColorRamps and Viridis R packages.
There are many options to choose from so please
read the documentation of these packages. "\code{Default=grDevices}

\item[\code{plot\_res}] options 1-8 option 1: slope coefficient, option 2: r squared,
option 3: nr of components, option 4: difference to the  origin , option 5: slope coefficient percentile
option 6: r squared percentile, option 7: nr of components percentile,
option 8: difference to the origin percentile \code{Default=2}

\item[\code{keep\_editable}] Keep option to add extra features after plotting  \code{Default=FALSE}

\item[\code{verbose}] Print text \code{Default=FALSE}.
\end{ldescription}
\end{Arguments}
%
\begin{Value}
Returns a list which contains 10 elements
element 1: slope coefficient
element 2: r squared
element 3: nr of components
element 4: difference to the origin
element 5: slope coefficient percentile
element 6: r squared percentile
element 7: nr of components percentile,
element 8: difference to the origin percentile
element 9: y-axis values of the matrices which is sedimentation rate
element 10: x-axis values of the matrices which is depth
\end{Value}
%
\begin{Author}
Based on the \LinkA{eAsm}{eAsm} function of the 'astrochron' R package and the 'eCOCO' and 'COCO' function of the 'Acycle' software
\end{Author}
%
\begin{References}
Routines for astrochronologic testing, astronomical time scale construction, and
time series analysis <doi:10.1016/j.earscirev.2018.11.015>

Acycle: Time-series analysis software for paleoclimate research and education,
Mingsong Li, Linda Hinnov, Lee Kump,
Computers \& Geosciences,Volume 127,2019,Pages 12-22,ISSN 0098-3004,
<doi:10.1016/j.cageo.2019.02.011>

Tracking variable sedimentation rates and astronomical forcing in Phanerozoic paleoclimate proxy series with evolutionary correlation coefficients and hypothesis testing,
Mingsong Li, Lee R. Kump, Linda A. Hinnov, Michael E. Mann,
Earth and Planetary Science Letters,Volume 501,
T2018,Pages 165-179,ISSN 0012-821X,<doi:10.1016/j.epsl.2018.08.041>
\end{References}
%
\begin{Examples}
\begin{ExampleCode}

#estimate sedimentation rate for the magnetic susceptibility record
# of the Sullivan core of Pas et al., (2018).

mag_wt <- analyze_wavelet(data = mag,
dj = 1/100,
lowerPeriod = 0.1,
upperPeriod = 254,
verbose = FALSE,
omega_nr = 10)

sedrates <- flmw(wavelet = mag_wt,
    sedrate_low = 0.5,
    sedrate_high = 4,
    spacing = 0.05,
    cycles = c(2376,1600,1180,696,406,110),
    x_lab = "depth",
    y_lab = "sedrate",
    run_random = FALSE,
    rand_simulations = 50, # increase to get better constrainted resutls
    run_multicore = FALSE,
    genplot = FALSE,
    palette_name = "rainbow",
    color_brewer = "grDevices",
    plot_res = 2,
    keep_editable=FALSE,
    verbose=FALSE)


\end{ExampleCode}
\end{Examples}
\inputencoding{utf8}
\HeaderA{geo\_col}{Generate standard color codes for the Geological Time Scale}{geo.Rul.col}
%
\begin{Description}
Generates the R color code which corresponds its respective
geological subdivision
\end{Description}
%
\begin{Usage}
\begin{verbatim}
geo_col(name = NULL)
\end{verbatim}
\end{Usage}
%
\begin{Arguments}
\begin{ldescription}
\item[\code{name}] Name of the geologchronological subdivision
\end{ldescription}
\end{Arguments}
%
\begin{Value}
Returns the color code of the geological subdivision
\end{Value}
%
\begin{References}
Ogg, Gabi \& Ogg, James \& Gradstein, Felix. (2021).
Recommended color coding of stages - Appendix 1
from Geologic Time Scale 2020.
\end{References}
%
\begin{Examples}
\begin{ExampleCode}
#generate the Silurian part of the GTS
plot.new()
plot(
 x = c(0, 1),
 y = c(419.2, 443.8),
 col = "white",
 xlab = "",
 ylab = "Time (Ma)",
 xaxt = "n",
 xaxs = "i",
 yaxs = "i",
 ylim = rev(c(419, 444))
)            # Draw empty plot

polygon(
 x = c(0.66, 1, 1, 0.66),
 y = geo_loc("Rhuddanian"),
 col =geo_col("Rhuddanian")
)

text(
 0.85,geo_mid("Rhuddanian"),
 "Rhuddanian",
 cex = 1,
 col = "black",
 srt = 0
)

polygon(
 x = c(0.66, 1, 1, 0.66),
 y = geo_loc("Aeronian"),
 col =geo_col("Aeronian")
)

text(
 0.85,geo_mid("Aeronian"),
 "Aeronian",
 cex = 1,
 col = "black",
 srt = 0
)

polygon(
 x = c(0.66, 1, 1, 0.66),
 y = geo_loc("Telychian"),
 col =geo_col("Telychian")
)

text(
 0.85,geo_mid("Telychian"),
 "Telychian",
 cex = 1,
 col = "black",
 srt = 0
)

polygon(
 x = c(0.66, 1, 1, 0.66),
 y = geo_loc("Sheinwoodian"),
 col =geo_col("Sheinwoodian")
)

text(
 0.85,geo_mid("Sheinwoodian"),
 "Sheinwoodian",
 cex = 1,
 col = "black",
 srt = 0
)


polygon(
 x = c(0.66, 1, 1, 0.66),
 y = geo_loc("Homerian"),
 col =geo_col("Homerian")
)

text(
 0.85,geo_mid("Homerian"),
 "Homerian",
 cex = 1,
 col = "black",
 srt = 0
)


polygon(
 x = c(0.66, 1, 1, 0.66),
 y = geo_loc("Gorstian"),
 col =geo_col("Gorstian")
)

text(
 0.85,geo_mid("Gorstian"),
 "Gorstian",
 cex = 1,
 col = "black",
 srt = 0
)

polygon(
 x = c(0.66, 1, 1, 0.66),
 y = geo_loc("Ludfordian"),
 col =geo_col("Ludfordian")
)

text(
 0.85,geo_mid("Ludfordian"),
 "Ludfordian",
 cex = 1,
 col = "black",
 srt = 0
)

polygon(
 x = c(0.66, 1, 1, 0.66),
 y = geo_loc("Pridoli_Age"),
 col =geo_col("Pridoli_Age")
)



polygon(
 x = c(0.33, 0.66, 0.66, 0.33),
 y = geo_loc("Pridoli"),
 col =geo_col("Pridoli")
)

text(
 0.5,geo_mid("Pridoli"),
 "Pridoli",
 cex = 1,
 col = "black",
 srt = 0
)


polygon(
 x = c(0.33, 0.66, 0.66, 0.33),
 y = geo_loc("Ludlow"),
 col =geo_col("Ludlow")
)

text(
 0.5,geo_mid("Ludlow"),
 "Ludlow",
 cex = 1,
 col = "black",
 srt = 0
)

polygon(
 x = c(0.33, 0.66, 0.66, 0.33),
 y = geo_loc("Wenlock"),
 col =geo_col("Wenlock")
)

text(
 0.5,geo_mid("Wenlock"),
 "Wenlock",
 cex = 1,
 col = "black",
 srt = 0
)

polygon(
 x = c(0.33, 0.66, 0.66, 0.33),
 y = geo_loc("Llandovery"),
 col =geo_col("Llandovery")
)

text(
 0.5,geo_mid("Llandovery"),
 "Llandovery",
 cex = 1,
 col = "black",
 srt = 0
)

polygon(
 x = c(0, 0.33, 0.33, 0),
 y = geo_loc("Silurian"),
 col =geo_col("Silurian")
)

text(
 0.165,geo_mid("Silurian"),
 "Silurian",
 cex = 1,
 col = "black",
 srt = 0
)


\end{ExampleCode}
\end{Examples}
\inputencoding{utf8}
\HeaderA{geo\_loc}{Generates  ages for the boundaries of a geochronological subdivision}{geo.Rul.loc}
%
\begin{Description}
Generates  ages for the boundaries of a
geochronological subdivision which is based on the Geological Time Scale
\end{Description}
%
\begin{Usage}
\begin{verbatim}
geo_loc(name = NULL)
\end{verbatim}
\end{Usage}
%
\begin{Arguments}
\begin{ldescription}
\item[\code{name}] Name of the geologchronological subdivision
\end{ldescription}
\end{Arguments}
%
\begin{Value}
Returns the ages of the boundary of a geochronological subdivision
which can then be added to a polygon object
\end{Value}
%
\begin{References}
Ogg, Gabi \& Ogg, James \& Gradstein, Felix. (2021).
Recommended color coding of stages - Appendix 1
from Geologic Time Scale 2020.
\end{References}
%
\begin{Examples}
\begin{ExampleCode}
#generate the Silurian part of the GTS
plot.new()
plot(
 x = c(0, 1),
 y = c(419.2, 443.8),
 col = "white",
 xlab = "",
 ylab = "Time (Ma)",
 xaxt = "n",
 xaxs = "i",
 yaxs = "i",
 ylim = rev(c(419, 444))
)            # Draw empty plot

polygon(
 x = c(0.66, 1, 1, 0.66),
 y = geo_loc("Rhuddanian"),
 col =geo_col("Rhuddanian")
)

text(
 0.85,geo_mid("Rhuddanian"),
 "Rhuddanian",
 cex = 1,
 col = "black",
 srt = 0
)

polygon(
 x = c(0.66, 1, 1, 0.66),
 y = geo_loc("Aeronian"),
 col =geo_col("Aeronian")
)

text(
 0.85,geo_mid("Aeronian"),
 "Aeronian",
 cex = 1,
 col = "black",
 srt = 0
)

polygon(
 x = c(0.66, 1, 1, 0.66),
 y = geo_loc("Telychian"),
 col =geo_col("Telychian")
)

text(
 0.85,geo_mid("Telychian"),
 "Telychian",
 cex = 1,
 col = "black",
 srt = 0
)

polygon(
 x = c(0.66, 1, 1, 0.66),
 y = geo_loc("Sheinwoodian"),
 col =geo_col("Sheinwoodian")
)

text(
 0.85,geo_mid("Sheinwoodian"),
 "Sheinwoodian",
 cex = 1,
 col = "black",
 srt = 0
)


polygon(
 x = c(0.66, 1, 1, 0.66),
 y = geo_loc("Homerian"),
 col =geo_col("Homerian")
)

text(
 0.85,geo_mid("Homerian"),
 "Homerian",
 cex = 1,
 col = "black",
 srt = 0
)


polygon(
 x = c(0.66, 1, 1, 0.66),
 y = geo_loc("Gorstian"),
 col =geo_col("Gorstian")
)

text(
 0.85,geo_mid("Gorstian"),
 "Gorstian",
 cex = 1,
 col = "black",
 srt = 0
)

polygon(
 x = c(0.66, 1, 1, 0.66),
 y = geo_loc("Ludfordian"),
 col =geo_col("Ludfordian")
)

text(
 0.85,geo_mid("Ludfordian"),
 "Ludfordian",
 cex = 1,
 col = "black",
 srt = 0
)

polygon(
 x = c(0.66, 1, 1, 0.66),
 y = geo_loc("Pridoli_Age"),
 col =geo_col("Pridoli_Age")
)



polygon(
 x = c(0.33, 0.66, 0.66, 0.33),
 y = geo_loc("Pridoli"),
 col =geo_col("Pridoli")
)

text(
 0.5,geo_mid("Pridoli"),
 "Pridoli",
 cex = 1,
 col = "black",
 srt = 0
)


polygon(
 x = c(0.33, 0.66, 0.66, 0.33),
 y = geo_loc("Ludlow"),
 col =geo_col("Ludlow")
)

text(
 0.5,geo_mid("Ludlow"),
 "Ludlow",
 cex = 1,
 col = "black",
 srt = 0
)

polygon(
 x = c(0.33, 0.66, 0.66, 0.33),
 y = geo_loc("Wenlock"),
 col =geo_col("Wenlock")
)

text(
 0.5,geo_mid("Wenlock"),
 "Wenlock",
 cex = 1,
 col = "black",
 srt = 0
)

polygon(
 x = c(0.33, 0.66, 0.66, 0.33),
 y = geo_loc("Llandovery"),
 col =geo_col("Llandovery")
)

text(
 0.5,geo_mid("Llandovery"),
 "Llandovery",
 cex = 1,
 col = "black",
 srt = 0
)

polygon(
 x = c(0, 0.33, 0.33, 0),
 y = geo_loc("Silurian"),
 col =geo_col("Silurian")
)

text(
 0.165,geo_mid("Silurian"),
 "Silurian",
 cex = 1,
 col = "black",
 srt = 0
)


\end{ExampleCode}
\end{Examples}
\inputencoding{utf8}
\HeaderA{geo\_mid}{Generate the mean age of a geological subdivision}{geo.Rul.mid}
%
\begin{Description}
Generates the mean age of a
geological subdivision which is based on the Geological Time Scale
\end{Description}
%
\begin{Usage}
\begin{verbatim}
geo_mid(name = NULL)
\end{verbatim}
\end{Usage}
%
\begin{Arguments}
\begin{ldescription}
\item[\code{name}] Name of the geologchronological subdivision
\end{ldescription}
\end{Arguments}
%
\begin{Value}
Returns the mean age of the geochronological subdivision
\end{Value}
%
\begin{References}
Ogg, Gabi \& Ogg, James \& Gradstein, Felix. (2021).
Recommended color coding of stages - Appendix 1
from Geologic Time Scale 2020.
\end{References}
%
\begin{Examples}
\begin{ExampleCode}
#generate the Silurian part of the GTS
plot.new()
plot(
 x = c(0, 1),
 y = c(419.2, 443.8),
 col = "white",
 xlab = "",
 ylab = "Time (Ma)",
 xaxt = "n",
 xaxs = "i",
 yaxs = "i",
 ylim = rev(c(419, 444))
)            # Draw empty plot

polygon(
 x = c(0.66, 1, 1, 0.66),
 y = geo_loc("Rhuddanian"),
 col =geo_col("Rhuddanian")
)

text(
 0.85,geo_mid("Rhuddanian"),
 "Rhuddanian",
 cex = 1,
 col = "black",
 srt = 0
)

polygon(
 x = c(0.66, 1, 1, 0.66),
 y = geo_loc("Aeronian"),
 col =geo_col("Aeronian")
)

text(
 0.85,geo_mid("Aeronian"),
 "Aeronian",
 cex = 1,
 col = "black",
 srt = 0
)

polygon(
 x = c(0.66, 1, 1, 0.66),
 y = geo_loc("Telychian"),
 col =geo_col("Telychian")
)

text(
 0.85,geo_mid("Telychian"),
 "Telychian",
 cex = 1,
 col = "black",
 srt = 0
)

polygon(
 x = c(0.66, 1, 1, 0.66),
 y = geo_loc("Sheinwoodian"),
 col =geo_col("Sheinwoodian")
)

text(
 0.85,geo_mid("Sheinwoodian"),
 "Sheinwoodian",
 cex = 1,
 col = "black",
 srt = 0
)


polygon(
 x = c(0.66, 1, 1, 0.66),
 y = geo_loc("Homerian"),
 col =geo_col("Homerian")
)

text(
 0.85,geo_mid("Homerian"),
 "Homerian",
 cex = 1,
 col = "black",
 srt = 0
)


polygon(
 x = c(0.66, 1, 1, 0.66),
 y = geo_loc("Gorstian"),
 col =geo_col("Gorstian")
)

text(
 0.85,geo_mid("Gorstian"),
 "Gorstian",
 cex = 1,
 col = "black",
 srt = 0
)

polygon(
 x = c(0.66, 1, 1, 0.66),
 y = geo_loc("Ludfordian"),
 col =geo_col("Ludfordian")
)

text(
 0.85,geo_mid("Ludfordian"),
 "Ludfordian",
 cex = 1,
 col = "black",
 srt = 0
)

polygon(
 x = c(0.66, 1, 1, 0.66),
 y = geo_loc("Pridoli_Age"),
 col =geo_col("Pridoli_Age")
)



polygon(
 x = c(0.33, 0.66, 0.66, 0.33),
 y = geo_loc("Pridoli"),
 col =geo_col("Pridoli")
)

text(
 0.5,geo_mid("Pridoli"),
 "Pridoli",
 cex = 1,
 col = "black",
 srt = 0
)


polygon(
 x = c(0.33, 0.66, 0.66, 0.33),
 y = geo_loc("Ludlow"),
 col =geo_col("Ludlow")
)

text(
 0.5,geo_mid("Ludlow"),
 "Ludlow",
 cex = 1,
 col = "black",
 srt = 0
)

polygon(
 x = c(0.33, 0.66, 0.66, 0.33),
 y = geo_loc("Wenlock"),
 col =geo_col("Wenlock")
)

text(
 0.5,geo_mid("Wenlock"),
 "Wenlock",
 cex = 1,
 col = "black",
 srt = 0
)

polygon(
 x = c(0.33, 0.66, 0.66, 0.33),
 y = geo_loc("Llandovery"),
 col =geo_col("Llandovery")
)

text(
 0.5,geo_mid("Llandovery"),
 "Llandovery",
 cex = 1,
 col = "black",
 srt = 0
)

polygon(
 x = c(0, 0.33, 0.33, 0),
 y = geo_loc("Silurian"),
 col =geo_col("Silurian")
)

text(
 0.165,geo_mid("Silurian"),
 "Silurian",
 cex = 1,
 col = "black",
 srt = 0
)


\end{ExampleCode}
\end{Examples}
\inputencoding{utf8}
\HeaderA{grey}{Grey scale record IODP 926 of Zeeden et al., (2013)}{grey}
%
\begin{Description}
IODP 926 grey scale record of Zeeden et al., (2013) for the (154-174m) interval.
The (154-174m) interval spans the Miocene.
\end{Description}
%
\begin{Details}
Column 1: depth (meters)\\{}
Column 2: greyscale value\\{}
\end{Details}
%
\begin{References}
Christian Zeeden, Frederik Hilgen, Thomas Westerhold, Lucas Lourens, Ursula Röhl, Torsten Bickert,
Revised Miocene splice, astronomical tuning and calcareous plankton biochronology of ODP Site 926 between 5 and 14.4Ma,
Palaeogeography, Palaeoclimatology, Palaeoecology,Volume 369,2013,Pages 430-451,ISSN 0031-0182,
<doi:10.1016/j.palaeo.2012.11.009>
\end{References}
\inputencoding{utf8}
\HeaderA{grey\_track}{Tracking points of the precession (22 kyr cycle) IODP 926 grey scale (154-174m) record of Zeeden et al., (2013)}{grey.Rul.track}
%
\begin{Description}
Example data which consists of tracking points of the precession (22 kyr cycle) in the wavelet
scalogram of the IODP 926 grey scale (154-174m) record of Zeeden et al., (2013)
\end{Description}
%
\begin{Details}
Column 1: Depth (meters)\\{}
Column 2: period (meters)\\{}
\end{Details}
%
\begin{References}
Christian Zeeden, Frederik Hilgen, Thomas Westerhold, Lucas Lourens, Ursula Röhl, Torsten Bickert,
Revised Miocene splice, astronomical tuning and calcareous plankton biochronology of ODP Site 926 between 5 and 14.4Ma,
Palaeogeography, Palaeoclimatology, Palaeoecology,Volume 369,2013,Pages 430-451,ISSN 0031-0182,
<doi:10.1016/j.palaeo.2012.11.009>
\end{References}
\inputencoding{utf8}
\HeaderA{GTS\_info}{Information of the Geological timescale 2020}{GTS.Rul.info}
%
\begin{Description}
GTS\_info data set consists the information of the Geological
timescale 2020 including the color data of Ogg et al., (2021)
The ages, durations, uncertainties and colors of the Geological
timescale 2020 are included in the data set
\end{Description}
%
\begin{Details}
Column 1: name	\\{}
Column 2: type	\\{}
Column 1: top age\\{}
Column 1: top error\\{}
Column 1: bottom age\\{}
Column 1: bottom error\\{}
Column 1: Cyan value\\{}
Column 1: Magenta value\\{}
Column 1: Yellow value\\{}
Column 1: Key  value\\{}
Column 1: Red Value\\{}
Column 1: Green value\\{}
Column 1: Blue value	\\{}
Column 1: font style	\\{}
Column 1: font color\\{}
\end{Details}
%
\begin{References}
Ogg, Gabi \& Ogg, James \& Gradstein, Felix. (2021).
Recommended color coding of stages - Appendix 1
from Geologic Time Scale 2020.
\end{References}
\inputencoding{utf8}
\HeaderA{Hilbert\_transform}{Perform a Hilbert transform on a signal}{Hilbert.Rul.transform}
%
\begin{Description}
Extract the amplitude modulation using the Hilbert transform.
\end{Description}
%
\begin{Usage}
\begin{verbatim}
Hilbert_transform(data = NULL, demean = TRUE, nr_pad = 100)
\end{verbatim}
\end{Usage}
%
\begin{Arguments}
\begin{ldescription}
\item[\code{data}] Input is a time series with the first column being depth or time and the second column being a proxy.

\item[\code{demean}] Remove the mean from the time series.

\item[\code{nr\_pad}] nr of points added tot the top and bottom of the data set
to mitigate the edging effect of the Hilbert transform.
\end{ldescription}
\end{Arguments}
%
\begin{Value}
Returns a matrix with 2 columns.
The first column is depth/time.
The second column is the Hilbert transform of the signal.
\end{Value}
%
\begin{Author}
Based on the the \LinkA{inst.pulse}{inst.pulse} function of the 'DecomposeR' R package.
\end{Author}
%
\begin{References}
Wouters, S., Crucifix, M., Sinnesael, M., Da Silva, A.C., Zeeden, C., Zivanovic, M., Boulvain, F.,
Devleeschouwer, X., 2022, "A decomposition approach to cyclostratigraphic signal processing".
Earth-Science Reviews 225 (103894). <doi:10.1016/j.earscirev.2021.103894>

Huang, Norden E., Zhaohua Wu, Steven R. Long, Kenneth C. Arnold, Xianyao Chen, and Karin Blank. 2009.
"On Instantaneous Frequency". Advances in Adaptive Data Analysis 01 (02): 177–229. <doi:10.1142/S1793536909000096>
\end{References}
%
\begin{Examples}
\begin{ExampleCode}
#Example in which the Hilbert transform (eg. amplitude modulation) of the ~210yr
#de Vries cycle is extracted from the Total Solar Irradiance data set of
#Steinhilber et al., (2012)

#Perform the CWT
TSI_wt <-
analyze_wavelet(
data = TSI,
dj = 1/200,
lowerPeriod = 16,
upperPeriod = 8192,
   verbose = FALSE,
   omega_nr = 6
 )

#Extract the 210 yr de Vries cycle from the wavelet spectra
de_Vries_cycle <- extract_signal_stable(wavelet=TSI_wt,
cycle=210,
period_up =1.25,
period_down = 0.75,
add_mean=TRUE,
plot_residual=FALSE)

#Perform the Hilbert transform on the amplitude record of the 210 yr de Vries
# cycle which was extracted from the wavelet spectra

de_Vries_cycle_hilbert <- Hilbert_transform(data=de_Vries_cycle,demean=TRUE)

\end{ExampleCode}
\end{Examples}
\inputencoding{utf8}
\HeaderA{lithlog\_disc}{Discriticizes lithologs}{lithlog.Rul.disc}
%
\begin{Description}
Discriticizes lithologs to allow further time-series analysis first the
Greatest common divisor/highest common factor is calculated which is then used to discriticize
the litholog to an evenly sampled data series. The function is designed to place the boundary
at the original depth level of the bed boundaries. The Greatest common divisor/highest common factor can
be a very small number as such the discriticized data set can be large which impacts computational
performance later on therefore a linear interpolation option is added to downscale the data to allow
for computational efficiency later on. This is made to discriticize lithologs created using the
'StratigrapheR' package. as such the same data format for input is used.
eg. column 1 is bottom of the bed, column 2 is top of bed, column is depth rank/proxy value
\end{Description}
%
\begin{Usage}
\begin{verbatim}
lithlog_disc(
  litholog = NULL,
  subset_fact = 10,
  lin_interp = FALSE,
  dt = NULL,
  genplot = FALSE,
  x_lab = "rank",
  y_lab = "depth (m)",
  keep_editable = FALSE
)
\end{verbatim}
\end{Usage}
%
\begin{Arguments}
\begin{ldescription}
\item[\code{litholog}] litholog input matrix with 3 columns column 1 is bottom of the bed,
column 2 is top of bed, column is depth rank/proxy value

\item[\code{subset\_fact}] subset factor which is x times the greatest common divider \code{Default=10}.

\item[\code{lin\_interp}] Linear interpolation of the data set \code{Default=FALSE}

\item[\code{dt}] step size  \code{Default=NULL}.

\item[\code{genplot}] generate plot  \code{Default=FALSE}

\item[\code{x\_lab}] label for the y-axis \code{Default="rank"}

\item[\code{y\_lab}] label for the y-axis \code{Default="depth (m)"}

\item[\code{keep\_editable}] Keep option to add extra features after plotting  \code{Default=FALSE}
\end{ldescription}
\end{Arguments}
%
\begin{Value}
Returns a matrix with 2 columns, the first column is depth the second columns is the depth/rank proxy
If genplot is \code{Default=TRUE} then a plot of the discriticizes time series is plotted
\end{Value}
%
\begin{References}
Wouters, S., Da Silva, A.-C., Boulvain, F., and Devleeschouwer, X.. 2021.
StratigrapheR: Concepts for Litholog Generation in R.
The R Journal. <doi:10.32614/RJ-2021-039>
\end{References}
%
\begin{Examples}
\begin{ExampleCode}
# Convert depth rank record to a discrete proxy record to allow for further
# analysis in which discrete time series are needed
depth_rank_example_disc <- lithlog_disc(litholog = depth_rank_example,
           subset_fact = 10,
           genplot = FALSE,
           x_lab = "rank",
           y_lab = "depth (m)",
           keep_editable=FALSE)


\end{ExampleCode}
\end{Examples}
\inputencoding{utf8}
\HeaderA{loess\_auto}{Perform an automatically loess based smoothing of a time series}{loess.Rul.auto}
%
\begin{Description}
Perform an automatically loess based smoothing of a time series.
The local polynomial regression with automatic smoothing parameter selection is based on an
optimization using the 'aicc' bias-corrected 'AIC' criterion and the 'gcv' generalized cross-validation criterion.
\end{Description}
%
\begin{Usage}
\begin{verbatim}
loess_auto(
  time_series = NULL,
  genplot = FALSE,
  print_span = FALSE,
  keep_editable = FALSE
)
\end{verbatim}
\end{Usage}
%
\begin{Arguments}
\begin{ldescription}
\item[\code{time\_series}] Input is a time series with the first column being depth or time and the second column being a proxy

\item[\code{genplot}] Option to generate plot \code{Default=TRUE}. \\{}
The plot will consist of the original signal in blue, the smoothed plot is displayed
in black and the + and - 1 sd bounds of the smoothing are displayed in red.

\item[\code{print\_span}] Print span length as a fraction of the total length of the record.

\item[\code{keep\_editable}] Keep option to add extra features after plotting  \code{Default=FALSE}
\end{ldescription}
\end{Arguments}
%
\begin{Value}
A matrix with 3 columns.
The first column is depth/time.
The second column is the smoothed curve.
The third column is difference between the original curve and the smoothed curve.
\end{Value}
%
\begin{Author}
Based on the the \code{\LinkA{loess.as}{loess.as}} function of the 'fANCOVA' R package.
\end{Author}
%
\begin{References}
Cleveland, W. S. (1979) Robust locally weighted regression and smoothing scatter plots. Journal of the American Statistical Association. 74, 829–836. <doi:10.1080/01621459.1979.10481038>
Hurvich, C.M., Simonoff, J.S., and Tsai, C.L. (1998), Smoothing Parameter Selection in Nonparametric Regression Using an Improved Akaike Information Criterion. Journal of the Royal Statistical Society B. 60, 271–293 <doi:10.1111/1467-9868.00125>
Golub, G., Heath, M. and Wahba, G. (1979). Generalized cross validation as a method for choosing a good ridge parameter. Technometrics. 21, 215–224. <doi:10.2307/1268518>
\end{References}
%
\begin{Examples}
\begin{ExampleCode}

#'smooth the period curve of the 405 kyr eccentricity cycle extracted from
# the magnetic susceptibility data set of Pas et al., (2018)
#perform the CWT on the magnetic susceptibility data set of Pas et al., (2018)

mag_wt <- analyze_wavelet(data = mag,
dj = 1/100,
lowerPeriod = 0.1,
upperPeriod = 254,
verbose = FALSE,
omega_nr = 10)

#Track the 405 kyr eccentricity cycle in a wavelet spectra

#mag_track <- track_period_wavelet(astro_cycle = 405,
#                                   wavelet=mag_wt,
#                                   n.levels = 100,
#                                   periodlab = "Period (metres)",
#                                   x_lab = "depth (metres)")

#Instead of tracking, the tracked solution data set mag_track_solution is used
mag_track <- mag_track_solution

mag_track_complete <- completed_series(
  wavelet = mag_wt,
  tracked_curve = mag_track,
  period_up = 1.2,
  period_down = 0.8,
  extrapolate = TRUE,
  genplot = FALSE,
  keep_editable=FALSE
)

#Smooth the completed tracking of the 405 kyr eccentricity cycle as tracked in the wavelet spectra
mag_track_complete <- loess_auto(time_series = mag_track_complete,
genplot = FALSE, print_span = FALSE,keep_editable=FALSE)

\end{ExampleCode}
\end{Examples}
\inputencoding{utf8}
\HeaderA{mag}{Magnetic susceptibility data of the Sullivan core of Pas et al., (2018)}{mag}
%
\begin{Description}
The magnetic susceptibility data set consists
of the magnetic susceptibility measurements of Pas et al., (2018), which measured the magnetic
susceptibility on the Sullivan core which is of Famennian age.
\end{Description}
%
\begin{Details}
Column 1: depth value (meters depoth)\\{}
Column 2: magnetic susceptibility  value\\{}
\end{Details}
%
\begin{References}
Damien Pas, Linda Hinnov, James E. (Jed) Day, Kenneth Kodama, Matthias Sinnesael, Wei Liu,
Cyclostratigraphic calibration of the Famennian stage (Late Devonian, Illinois Basin, USA),
Earth and Planetary Science Letters,
Volume 488,
2018,
Pages 102-114,
ISSN 0012-821X,
<doi:1016/j.epsl.2018.02.010>
\end{References}
\inputencoding{utf8}
\HeaderA{mag\_track\_solution}{Period of the 405 kyr ecc cycle in the magnetic susceptibility record of the Sullivan core}{mag.Rul.track.Rul.solution}
%
\begin{Description}
Data points which give the period (in meters) of the 405 kyr eccentricity cycle tracked
in the wavelet scalogram of the magnetic susceptibility record of the Sullivan core \\{}
The period was tracked using the \code{\LinkA{track\_period\_wavelet}{track.Rul.period.Rul.wavelet}} function\\{}
The tracking is based on the original age model of Pas et al., (2018)\\{}
\end{Description}
%
\begin{Details}
Column 1: Depth (meters)\\{}
Column 2: tracked period of 405 kyr eccentricity cycle (meters)\\{}
\end{Details}
%
\begin{References}
Damien Pas, Linda Hinnov, James E. (Jed) Day, Kenneth Kodama, Matthias Sinnesael, Wei Liu,
Cyclostratigraphic calibration of the Famennian stage (Late Devonian, Illinois Basin, USA),
Earth and Planetary Science Letters,
Volume 488,
2018,
Pages 102-114,
ISSN 0012-821X,
<doi:10.1016/j.epsl.2018.02.010>
\end{References}
\inputencoding{utf8}
\HeaderA{max\_detect}{Detect and filter out all maxima in a signal}{max.Rul.detect}
%
\begin{Description}
The \code{\LinkA{max\_detect}{max.Rul.detect}} function is used
to detect and filter out local maxima in a sinusoidal signal.
\end{Description}
%
\begin{Usage}
\begin{verbatim}
max_detect(data = NULL, pts)
\end{verbatim}
\end{Usage}
%
\begin{Arguments}
\begin{ldescription}
\item[\code{data}] Matrix or data frame with the first column being depth or time
and the second column being a proxy

\item[\code{pts}] The pts parameter specifies how many points to the left/right up/down the peak detect algorithm goes in detecting
a peak. The peak detecting algorithm works by comparing the values left/right up/down of it, if the values are both higher or lower
then the value a peak. To deal with error produced by this algorithm the pts parameter can be changed which can
aid in peak detection. Usually increasing the pts parameter means more peak certainty, however it also means that minor peaks might not be
picked up by the algorithm \code{Default=3}
\end{ldescription}
\end{Arguments}
%
\begin{Value}
\#Returns a matrix with 2 columns
first column is depth/time
the second column are local maxima values
\end{Value}
%
\begin{Examples}
\begin{ExampleCode}
#Example in which the ~210yr de Vries cycle is extracted from the Total Solar
#Irradiance data set of Steinhilber et al., (2012)
#after which all maxima are extracted

TSI_wt <-
analyze_wavelet(
data = TSI,
dj = 1/200,
lowerPeriod = 16,
upperPeriod = 8192,
   verbose = FALSE,
   omega_nr = 6
 )

de_Vries_cycle <- extract_signal_stable(wavelet=TSI_wt,
cycle=210,
period_up =1.25,
period_down = 0.75,
add_mean=TRUE,
plot_residual=FALSE)


min_de_Vries_cycle <- min_detect(de_Vries_cycle)

\end{ExampleCode}
\end{Examples}
\inputencoding{utf8}
\HeaderA{minimal\_tuning}{Create an age model using minimal tuning}{minimal.Rul.tuning}
%
\begin{Description}
Create an age model using the minimal tuning technique.
This means that the distance between 2 peaks of an extracted cycle are set
to duration of the interpreted astronomical cycle
\end{Description}
%
\begin{Usage}
\begin{verbatim}
minimal_tuning(
  data = NULL,
  pts = 5,
  cycle = 405,
  tune_opt = "max",
  output = 0,
  genplot = FALSE,
  keep_editable = FALSE
)
\end{verbatim}
\end{Usage}
%
\begin{Arguments}
\begin{ldescription}
\item[\code{data}] Input is an cycle extracted filtered in the depth domain

\item[\code{pts}] The pts parameter specifies how many points to the left/right up/down the peak detect algorithm goes in detecting
a peak. The peak detecting algorithm works by comparing the values left/right up/down of it, if the values are both higher or lower
then the value a peak. To deal with error produced by this algorithm the pts parameter can be changed which can
aid in peak detection. Usually increasing the pts parameter means more peak certainty, however it also means that minor peaks might not be
picked up by the algorithm \code{Default=5}

\item[\code{cycle}] duration in kyr of the filtered/extracted cycle

\item[\code{tune\_opt}] tuning options "min", "max" and "minmax" use minima, maxima or both
of the cyclic signal to create the age model \code{Default="max"}

\item[\code{output}] \#'The output depends on the output setting
If output = 0 output is a matrix of with 4 columns being; depth,proxy,sedimentation rate and time
If output = 1 output is a matrix of with 2 columns being; depth and sedimentation rate
\#'If output = 2 output is a matrix of with 2 columns being; depth and time

\item[\code{genplot}] Keep option to add extra features after plotting  \code{Default=FALSE}

\item[\code{keep\_editable}] Keep option to add extra features after plotting  \code{Default=FALSE}
\end{ldescription}
\end{Arguments}
%
\begin{Value}
The output depends on the output setting
If output = 0 output is a matrix of with 4 columns being (depth,proxy,sedimentation rate and time)
If genplot = TRUE 4 plots are generated; depth vs proxy, depth vs sedimentation rate, depth vs time and time vs proxy
If output = 1 output is a matrix of with 2 columns being (depth and sedimentation rate )
If genplot = TRUE a plot of depth vs sedimentation rate is generated
If output = 2 output is a matrix of with 2 columns being (depth and time)
If genplot = TRUE a plot of depth vs time is generated
\end{Value}
%
\begin{Author}
Part of the code is based on the \LinkA{sedrate2time}{sedrate2time}
function of the 'astrochron' R package
\end{Author}
%
\begin{References}
Routines for astrochronologic testing, astronomical time scale construction, and
time series analysis <doi:10.1016/j.earscirev.2018.11.015>
\end{References}
%
\begin{Examples}
\begin{ExampleCode}

# Extract the 405kyr eccentricity cycle from the wavelet scalogram
# from the magnetic susceptibility record f the Sullivan core
# of Pas et al., (2018) and then create a age model using minimal tuning
# (e.g.) set the distance between peaks to 405 kyr

mag_wt <- analyze_wavelet(data = mag,
dj = 1/100,
lowerPeriod = 0.1,
upperPeriod = 254,
verbose = FALSE,
omega_nr = 10)


mag_405 <- extract_signal_stable_V2(
 wavelet = mag_wt,
 period_max = 4,
 period_min = 2,
 add_mean = FALSE,
 plot_residual = FALSE,
 keep_editable = FALSE
)

mag_405_min_tuning <- minimal_tuning(data = mag_405,
pts = 5,
cycle = 405,
tune_opt = "max",
output = 0,
genplot = FALSE,
keep_editable = FALSE)




\end{ExampleCode}
\end{Examples}
\inputencoding{utf8}
\HeaderA{min\_detect}{Detect and filter out all minima in a signal}{min.Rul.detect}
%
\begin{Description}
The \code{\LinkA{min\_detect}{min.Rul.detect}} function is used to detect and
filter out local minima in a sinusoidal signal
\end{Description}
%
\begin{Usage}
\begin{verbatim}
min_detect(data = NULL, pts = 3)
\end{verbatim}
\end{Usage}
%
\begin{Arguments}
\begin{ldescription}
\item[\code{data}] Matrix or data frame with first column being depth or time and
the second column being a proxy

\item[\code{pts}] the pts parameter specifies how many points to the left/right up/down the peak detect algorithm goes in detecting
a peak. The peak detecting algorithm works by comparing the values left/right up/down of it, if the values are both higher or lower
then the value a peak. To deal with error produced by this algorithm the pts parameter can be changed which can
aid in peak detection. Usually increasing the pts parameter means more peak certainty, however it also means that minor peaks might not be
picked up by the algorithm \code{Default=3}
\end{ldescription}
\end{Arguments}
%
\begin{Value}
\#Returns a matrix with 2 columns
first column is depth/time
the second column are local minima values
\end{Value}
%
\begin{Examples}
\begin{ExampleCode}
#Example in which the ~210yr de Vries cycle is extracted from the Total Solar
#Irradiance data set of Steinhilber et al., (2012)
#after which all minima are extracted

TSI_wt <-
analyze_wavelet(
data = TSI,
dj = 1/200,
lowerPeriod = 16,
upperPeriod = 8192,
   verbose = FALSE,
   omega_nr = 6
 )

de_Vries_cycle <- extract_signal_stable(wavelet=TSI_wt,
cycle=210,
period_up =1.25,
period_down = 0.75,
add_mean=TRUE,
plot_residual=FALSE)


min_de_Vries_cycle <- min_detect(de_Vries_cycle)

\end{ExampleCode}
\end{Examples}
\inputencoding{utf8}
\HeaderA{model\_red\_noise\_wt}{Models average spectral power based curves based on a red-noise signal generated using the characteristics of an input signal.}{model.Rul.red.Rul.noise.Rul.wt}
%
\begin{Description}
The \code{\LinkA{model\_red\_noise\_wt}{model.Rul.red.Rul.noise.Rul.wt}} function is used to generate
average spectral power curves based on and input signal and set wavelet settings.
\end{Description}
%
\begin{Usage}
\begin{verbatim}
model_red_noise_wt(
  wavelet = NULL,
  n_simulations = NULL,
  run_multicore = FALSE,
  verbose = FALSE
)
\end{verbatim}
\end{Usage}
%
\begin{Arguments}
\begin{ldescription}
\item[\code{wavelet}] Wavelet object created using the \code{\LinkA{analyze\_wavelet}{analyze.Rul.wavelet}} function.

\item[\code{n\_simulations}] Number of red noise simulations.

\item[\code{run\_multicore}] run simulation using multiple cores \code{Default=FALSE}
the simulation is run at x-2 cores to allow the 2 remaining processes to run background processes.

\item[\code{verbose}] Print text \code{Default=FALSE}.
\end{ldescription}
\end{Arguments}
%
\begin{Value}
Returns a matrix in which each column represents the average spectral
power resulting from a red-noise run.
\end{Value}
%
\begin{Author}
Code based on the \LinkA{analyze.wavelet}{analyze.wavelet} function of the 'WaveletComp' R package
and \LinkA{wt}{wt} function of the 'biwavelet' R package which are based on the
wavelet 'MATLAB' code written by Christopher Torrence and Gibert P. Compo (1998).
\end{Author}
%
\begin{References}
Angi Roesch and Harald Schmidbauer (2018). WaveletComp: Computational
Wavelet Analysis. R package version 1.1.
\url{https://CRAN.R-project.org/package=WaveletComp}

Gouhier TC, Grinsted A, Simko V (2021). R package biwavelet: Conduct Univariate and Bivariate Wavelet Analyses. (Version 0.20.21),
\url{https://github.com/tgouhier/biwavelet}

Torrence, C., and G. P. Compo. 1998. A Practical Guide to Wavelet Analysis.
Bulletin of the American Meteorological Society 79:61-78.
\url{https://paos.colorado.edu/research/wavelets/bams_79_01_0061.pdf}

Morlet, Jean, Georges Arens, Eliane Fourgeau, and Dominique Glard.
"Wave propagation and sampling theory—Part I: Complex signal and scattering in multilayered media.
" Geophysics 47, no. 2 (1982): 203-221.
\url{https://pubs.geoscienceworld.org/geophysics/article/47/2/203/68601/Wave-propagation-and-sampling-theory-Part-I}

J. Morlet, G. Arens, E. Fourgeau, D. Giard;
Wave propagation and sampling theory; Part II, Sampling theory and complex waves.
Geophysics 1982 47 (2): 222–236. \url{https://pubs.geoscienceworld.org/geophysics/article/47/2/222/68604/Wave-propagation-and-sampling-theory-Part-II}
\end{References}
%
\begin{Examples}
\begin{ExampleCode}

#'#generate average spectral power curves based on red noise curves which are
# based on the magnetic susceptibility record of the Sullivan core of Pas et al., (2018)

mag_wt <- analyze_wavelet(data = mag,
dj = 1/100,
lowerPeriod = 0.1,
upperPeriod = 254,
verbose = FALSE,
omega_nr = 10)

#increase n_simulations to better define the red noise spectral power curve
mag_wt_red_noise <- model_red_noise_wt(wavelet=mag_wt,
n_simulations=10, # increase number for better constrained results
run_multicore=FALSE,
verbose=FALSE)



\end{ExampleCode}
\end{Examples}
\inputencoding{utf8}
\HeaderA{percentile\_from\_red\_noise}{Calculate average spectral power from red noise curves for a given percentile}{percentile.Rul.from.Rul.red.Rul.noise}
%
\begin{Description}
The \code{\LinkA{percentile\_from\_red\_noise}{percentile.Rul.from.Rul.red.Rul.noise}} function is
used to generate and average spectral power curve based on
a set percentile based. To generate the percentile curve the results of
the \code{\LinkA{model\_red\_noise\_wt}{model.Rul.red.Rul.noise.Rul.wt}} function are used.
\end{Description}
%
\begin{Usage}
\begin{verbatim}
percentile_from_red_noise(red_noise = NULL, wavelet = NULL, percentile = NULL)
\end{verbatim}
\end{Usage}
%
\begin{Arguments}
\begin{ldescription}
\item[\code{red\_noise}] Red noise curves generated using the \code{\LinkA{model\_red\_noise\_wt}{model.Rul.red.Rul.noise.Rul.wt}} function.

\item[\code{wavelet}] Wavelet object created using the \code{\LinkA{analyze\_wavelet}{analyze.Rul.wavelet}} function.

\item[\code{percentile}] Percentile value (0-1).
\end{ldescription}
\end{Arguments}
%
\begin{Value}
Returns a matrix with 2 columns.\\{}
The first column is the period (m). \\{}
The second column is the spectral power at percentile x based on \\{}
the red noise modelling runs. \\{}
\end{Value}
%
\begin{Examples}
\begin{ExampleCode}

#'#generate average spectral power curves based on red noise curves which are
# based on the magnetic susceptibility record of the Sullivan core of Pas et al., (2018)

mag_wt <- analyze_wavelet(data = mag,
dj = 1/100,
lowerPeriod = 0.1,
upperPeriod = 254,
verbose = FALSE,
omega_nr = 10)

#increase n_simulations to better define the red noise spectral power curve
mag_wt_red_noise <- model_red_noise_wt(wavelet=mag_wt,
n_simulations=10, # Increase number for a better constrained result
run_multicore=FALSE,
verbose=FALSE)


prob_curve <- percentile_from_red_noise(
red_noise = mag_wt_red_noise,
wavelet = mag_wt,
percentile = 0.9)



\end{ExampleCode}
\end{Examples}
\inputencoding{utf8}
\HeaderA{plot\_astro\_anchor}{Plot proxy record anchored to an astronomical solution}{plot.Rul.astro.Rul.anchor}
%
\begin{Description}
Plot the results of the anchoring  the extracted signal to an astronomical solution using
which was conducted using the \code{\LinkA{astro\_anchor}{astro.Rul.anchor}}
\end{Description}
%
\begin{Usage}
\begin{verbatim}
plot_astro_anchor(
  astro_solution = NULL,
  proxy_signal = NULL,
  anchor_points = NULL,
  time_dir = TRUE,
  keep_editable = FALSE
)
\end{verbatim}
\end{Usage}
%
\begin{Arguments}
\begin{ldescription}
\item[\code{astro\_solution}] Input is an astronomical solution with with the the proxy record was be anchored to,
the input should be a matrix or data frame with the first column being
age and the second column should be a insolation/angle/value

\item[\code{proxy\_signal}] Input is the proxy data set which will which was
anchored to an astronomical solution, the input should be a matrix or
data frame with the first column being  depth/time and the second column should be a proxy value.

\item[\code{anchor\_points}] Anchor points generated using the \code{\LinkA{astro\_anchor}{astro.Rul.anchor}} function

\item[\code{time\_dir}] The direction of the proxy record which was assumed during anchoring if time increases with increasing depth/time values
(e.g. bore hole data which gets older with increasing depth ) then time\_dir should be set to TRUE
if time decreases with depth/time values (eg stratospheric logs where 0m is the bottom of the section)
then time\_dir should be set to FALSE \code{time\_dir=TRUE}

\item[\code{keep\_editable}] Keep option to add extra features after plotting  \code{Default=FALSE}
\end{ldescription}
\end{Arguments}
%
\begin{Value}
The output is a set of 2 plots connected by lines
The top plot is the proxy record with anchor points on top of it
The bottom plot is the astronomical solution
The lines connect the anchor points
\end{Value}
%
\begin{Examples}
\begin{ExampleCode}

# Use the grey_track example tracking points to anchor the grey scale data set
# of Zeeden et al., (2013) to the p-0.5t la2004 solution

grey_wt <-
 analyze_wavelet(
   data = grey,
   dj = 1/200,
   lowerPeriod = 0.02,
   upperPeriod = 256,
   verbose = FALSE,
   omega_nr = 8
 )

#Use the pretracked grey_track curve which traced the precession cycle
grey_track <- completed_series(
 wavelet = grey_wt,
 tracked_curve  = grey_track,
 period_up  = 1.25,
 period_down  = 0.75,
 extrapolate = TRUE,
genplot = FALSE
)
# Extract precession, obliquity and eccentricity to create a synthetic insolation curve

grey_prec <- extract_signal(
tracked_cycle_curve = grey_track[,c(1,2)],
wavelet = grey_wt,
period_up = 1.2,
period_down = 0.8,
add_mean = FALSE,
tracked_cycle_period = 22,
extract_cycle = 22,
tune = FALSE,
plot_residual = FALSE
)

grey_obl <- extract_signal(
 tracked_cycle_curve = grey_track[,c(1,2)],
 wavelet = grey_wt,
 period_up = 1.2,
 period_down = 0.8,
 add_mean = FALSE,
 tracked_cycle_period = 22,
 extract_cycle = 110,
 tune = FALSE,
 plot_residual = FALSE
)

grey_ecc <- extract_signal(
 tracked_cycle_curve = grey_track[,c(1,2)],
 wavelet = grey_wt,
 period_up = 1.25,
 period_down = 0.75,
 add_mean = FALSE,
 tracked_cycle_period = 22,
 extract_cycle = 40.8,
 tune = FALSE,
 plot_residual = FALSE
)

insolation_extract <- cbind(grey_ecc[,1],grey_prec[,2]+grey_obl[,2]+grey_ecc[,2]+mean(grey[,2]))
insolation_extract <- as.data.frame(insolation_extract)
insolation_extract_mins <- min_detect(insolation_extract,pts=3)

#use the astrosignal_example to tune to which is an \cr
# ETP solution (p-0.5t la2004 solution).

astrosignal_example <- na.omit(astrosignal_example)
astrosignal_example[,2] <- -1*astrosignal_example[,2]
astrosignal <- as.data.frame(astrosignal_example)

#anchor the synthetic insolation curve extracted from the
# grey scale record to the insolation curve.
#use the anchor_points_grey data set to plot the
#result of using the astro_anchor function

#anchor_points_grey <- astro_anchor(
#astro_solution = astrosignal,
#proxy_signal = insolation_extract,
#proxy_min_or_max = "min",
#clip_astrosolution = FALSE,
#astrosolution_min_or_max = "min",
#clip_high = NULL,
#clip_low = NULL,
#extract_astrosolution  = FALSE,
#astro_period_up  = NULL,
#astro_period_down  = NULL,
#astro_period_cycle  = NULL,
#extract_proxy_signal  = FALSE,
#proxy_period_up  = NULL,
#proxy_period_down  = NULL,
#proxy_period_cycle  = NULL,
#pts=3,
#verbose=FALSE,
#genplot=FALSE # set verbose to TRUE to allow for anchoring using text feedback commands
#)


plot_astro_anchor(astro_solution = astrosignal,
proxy_signal = insolation_extract,
anchor_points = anchor_points_grey,
time_dir = FALSE,
keep_editable = FALSE)



\end{ExampleCode}
\end{Examples}
\inputencoding{utf8}
\HeaderA{plot\_avg\_wavelet}{Plot the average spectral power of a wavelet spectra}{plot.Rul.avg.Rul.wavelet}
%
\begin{Description}
Plot the average spectral power of a wavelet spectra using the results of
the \code{\LinkA{analyze\_wavelet}{analyze.Rul.wavelet}} function.
\end{Description}
%
\begin{Usage}
\begin{verbatim}
plot_avg_wavelet(
  wavelet = NULL,
  y_lab = "Power",
  x_lab = "period (metres)",
  keep_editable = FALSE
)
\end{verbatim}
\end{Usage}
%
\begin{Arguments}
\begin{ldescription}
\item[\code{wavelet}] Wavelet object created using the \code{\LinkA{analyze\_wavelet}{analyze.Rul.wavelet}} function.

\item[\code{y\_lab}] Label for the y-axis \code{Default="Power"}.

\item[\code{x\_lab}] Label for the x-axis \code{Default="depth (metres)"}.

\item[\code{keep\_editable}] Keep option to add extra features after plotting  \code{Default=FALSE}
\end{ldescription}
\end{Arguments}
%
\begin{Value}
The output is a plot of the average spectral power of a wavelet spectra
\end{Value}
%
\begin{Examples}
\begin{ExampleCode}

#Example 1. Plot the average spectral power of the wavelet spectra of
# the Total Solar Irradiance data set of Steinhilber et al., (2012)
TSI_wt <-
 analyze_wavelet(
   data = TSI,
   dj = 1/200,
   lowerPeriod = 16,
   upperPeriod = 8192,
   verbose = FALSE,
   omega_nr = 6
 )

plot_avg_wavelet(wavelet=TSI_wt,
                 y_lab= "power",
                 x_lab="period (years)",
                 keep_editable=FALSE)


#Example 2. Plot the average spectral power of the wavelet spectra of \cr
# the magnetic susceptibility data set of Pas et al., (2018)
mag_wt <-
analyze_wavelet(
data = mag,
dj = 1/100,
lowerPeriod = 0.1,
upperPeriod = 254,
verbose = FALSE,
omega_nr = 10
)
plot_avg_wavelet(wavelet=mag_wt,
                 y_lab= "power",
                 x_lab="period (metres)",
                 keep_editable=FALSE)



#Example 3. Plot the average spectral power of the wavelet spectra of
#the greyscale data set of Zeeden et al., (2013)
grey_wt <-
 analyze_wavelet(
   data = grey,
   dj = 1/200,
   lowerPeriod = 0.02,
   upperPeriod = 256,
   verbose = FALSE,
   omega_nr = 8
 )

plot_avg_wavelet(wavelet=grey_wt,
                 y_lab= "power",
                 x_lab="period (metres)",
                 keep_editable=FALSE)


\end{ExampleCode}
\end{Examples}
\inputencoding{utf8}
\HeaderA{plot\_sed\_model}{Plot sedimentation modelling results}{plot.Rul.sed.Rul.model}
%
\begin{Description}
The \code{\LinkA{plot\_sed\_model}{plot.Rul.sed.Rul.model}} function is used plot/re-plot the results from the
\code{\LinkA{flmw}{flmw}} and  \code{\LinkA{sum\_power\_sedrate}{sum.Rul.power.Rul.sedrate}} functions
\end{Description}
%
\begin{Usage}
\begin{verbatim}
plot_sed_model(
  model_results = NULL,
  plot_res = 1,
  x_lab = "depth (m)",
  y_lab = "sed rate cm/kyr",
  keep_editable = FALSE,
  palette_name = "rainbow",
  color_brewer = "grDevices"
)
\end{verbatim}
\end{Usage}
%
\begin{Arguments}
\begin{ldescription}
\item[\code{model\_results}] Wavelet object created using the \code{\LinkA{analyze\_wavelet}{analyze.Rul.wavelet}} function

\item[\code{plot\_res}] Numbers to be used as input form the \code{\LinkA{flmw}{flmw}}output
options 1-8 option 1: slope coefficient, option 2: r squared,
option 3: nr of components, option 4: difference to origin, option 5: slope coefficient percentile
option 6: r squared percentile, option 7: nr of components percentile,
option 8: difference to origin percentile. If the output of the  \code{\LinkA{sum\_power\_sedrate}{sum.Rul.power.Rul.sedrate}} function is used
then input should be option 1: sum max power option 2: nr of components

\item[\code{x\_lab}] Label for x-axis \code{Default="depth (m)"}

\item[\code{y\_lab}] Label for y-axis \code{Default=""sed rate cm/kyr""}

\item[\code{keep\_editable}] Keep option to add extra features after plotting  \code{Default=FALSE}

\item[\code{palette\_name}] Name of the color palette which is used for plotting.
The color palettes than can be chosen depends on which the R package is specified in
the color\_brewer parameter. The included R packages from which palettes can be chosen
from are; the 'RColorBrewer', 'grDevices', 'ColorRamps' and 'Viridis' R packages.
There are many options to choose from so please
read the documentation of these packages \code{Default=rainbow}.
The R package 'viridis' has the color palette options: “magma”, “plasma”,
“inferno”, “viridis”, “mako”, and “rocket”  and “turbo”
To see the color palette options of the The R pacakge 'RColorBrewer' run
the RColorBrewer::brewer.pal.info() function
The R package 'colorRamps' has the color palette options:"blue2green",
"blue2green2red", "blue2red",    "blue2yellow", "colorRamps",    "cyan2yellow",
"green2red", "magenta2green", "matlab.like", "matlab.like2" and    "ygobb"
The R package 'grDevices' has the built in  palette options:"rainbow",
"heat.colors", "terrain.colors","topo.colors" and "cm.colors"
To see even more color palette options of the The R pacakge 'grDevices' run
the grDevices::hcl.pals() function

\item[\code{color\_brewer}] Name of the R package from which the color palette is chosen from.
The included R packages from which palettes can be chosen
are; the RColorBrewer, grDevices, ColorRamps and Viridis R packages.
There are many options to choose from so please
read the documentation of these packages. "\code{Default=grDevices}
\end{ldescription}
\end{Arguments}
%
\begin{Value}
Returns a plot of sedimentation rates vs depth and a value which was generated using
the \code{\LinkA{flmw}{flmw}} or \code{\LinkA{sum\_power\_sedrate}{sum.Rul.power.Rul.sedrate}} functions
\end{Value}
%
\begin{Examples}
\begin{ExampleCode}

#estimate sedimentation rate for the the magnetic susceptibility record
# of the Sullivan core of Pas et al., (2018).

mag_wt <- analyze_wavelet(data = mag,
dj = 1/100,
lowerPeriod = 0.1,
upperPeriod = 254,
verbose = FALSE,
omega_nr = 10)

#increase n_simulations to better define the red noise spectral power curve
mag_wt_red_noise <- model_red_noise_wt(wavelet=mag_wt,
n_simulations=10, # increase for a better constrained result
run_multicore=FALSE,
verbose=FALSE)

sedrates <- sum_power_sedrate(red_noise=mag_wt_red_noise,
wavelet=mag_wt,
percentile=0.75,
sedrate_low = 0.5,
sedrate_high = 4,
spacing = 0.05,
cycles = c(2376,1600,1180,696,406,110),
x_lab="depth",
y_lab="sedrate",
run_multicore=FALSE,
genplot = FALSE,
palette_name = "rainbow",
color_brewer= "grDevices",
verbose=FALSE)

plot_sed_model(model_results=sedrates,
plot_res=1,
x_lab = "depth (m)",
y_lab = "sed rate cm/kyr",
keep_editable=FALSE,
palette_name = "rainbow",
color_brewer= "grDevices")



\end{ExampleCode}
\end{Examples}
\inputencoding{utf8}
\HeaderA{plot\_wavelet}{Plots a wavelet power spectra}{plot.Rul.wavelet}
%
\begin{Description}
Plot wavelet spectra using the outcome of the \code{\LinkA{analyze\_wavelet}{analyze.Rul.wavelet}} function.
\end{Description}
%
\begin{Usage}
\begin{verbatim}
plot_wavelet(
  wavelet = NULL,
  lowerPeriod = NULL,
  upperPeriod = NULL,
  n.levels = 100,
  palette_name = "rainbow",
  color_brewer = "grDevices",
  useRaster = TRUE,
  periodlab = "Period (metres)",
  x_lab = "depth (metres)",
  keep_editable = FALSE,
  dev_new = TRUE,
  plot_dir = TRUE,
  add_lines = NULL,
  add_points = NULL,
  add_abline_h = NULL,
  add_abline_v = NULL,
  add_MTM_peaks = FALSE,
  add_data = TRUE,
  add_avg = FALSE,
  add_pval = FALSE,
  pval_abline = c(0.1, 0.05),
  pval_cutoff = c(0.1),
  add_MTM = FALSE,
  demean_mtm = TRUE,
  detrend_mtm = TRUE,
  padfac_mtm = 5,
  tbw_mtm = 3,
  plot_horizontal = TRUE
)
\end{verbatim}
\end{Usage}
%
\begin{Arguments}
\begin{ldescription}
\item[\code{wavelet}] wavelet object created using the \code{\LinkA{analyze\_wavelet}{analyze.Rul.wavelet}} function.

\item[\code{lowerPeriod}] Lowest period value which will be plotted

\item[\code{upperPeriod}] Highest period value which will be plotted

\item[\code{n.levels}] Number of color levels \code{Default=100}.

\item[\code{palette\_name}] Name of the color palette which is used for plotting.
The color palettes than can be chosen depends on which the R package is specified in
the color\_brewer parameter. The included R packages from which palettes can be chosen
from are; the 'RColorBrewer', 'grDevices', 'ColorRamps' and 'Viridis' R packages.
There are many options to choose from so please
read the documentation of these packages \code{Default=rainbow}.
The R package 'viridis' has the color palette options: “magma”, “plasma”,
“inferno”, “viridis”, “mako”, and “rocket”  and “turbo”
To see the color palette options of the The R pacakge 'RColorBrewer' run
the RColorBrewer::brewer.pal.info() function
The R package 'colorRamps' has the color palette options:"blue2green",
"blue2green2red", "blue2red",    "blue2yellow", "colorRamps",    "cyan2yellow",
"green2red", "magenta2green", "matlab.like", "matlab.like2" and    "ygobb"
The R package 'grDevices' has the built in  palette options:"rainbow",
"heat.colors", "terrain.colors","topo.colors" and "cm.colors"
To see even more color palette options of the The R pacakge 'grDevices' run
the grDevices::hcl.pals() function

\item[\code{color\_brewer}] Name of the R package from which the color palette is chosen from.
The included R packages from which palettes can be chosen
are; the RColorBrewer, grDevices, ColorRamps and Viridis R packages.
There are many options to choose from so please
read the documentation of these packages. "\code{Default=grDevices}

\item[\code{useRaster}] Plot as a raster or vector image \code{Default=TRUE}.
WARNING plotting as a vector image is computationally intensive.

\item[\code{periodlab}] Label for the y-axis \code{Default="Period (metres)"}.

\item[\code{x\_lab}] Label for the x-axis \code{Default="depth (metres)"}.

\item[\code{keep\_editable}] Keep option to add extra features after plotting  \code{Default=FALSE}

\item[\code{dev\_new}] Opens a new plotting window to plot the plot, this guarantees a "nice" looking plot however when plotting in an R markdown
document the plot might not plot  \code{Default=TRUE}

\item[\code{plot\_dir}] The direction of the proxy record which is assumed for tuning if time increases with increasing depth/time values
(e.g. bore hole data which gets older with increasing depth ) then plot\_dir should be set to TRUE
if time decreases with depth/time values (eg stratospheric logs where 0m is the bottom of the section)
then plot\_dir should be set to FALSE \code{plot\_dir=TRUE}

\item[\code{add\_lines}] Add  lines to the wavelet plot input should be matrix with first axis being depth/time the columns after that
should be period values  \code{Default=NULL}

\item[\code{add\_points}] Add points to the wavelet plot input should be matrix with first axis being depth/time and columns after that
should be period values \code{Default=NULL}

\item[\code{add\_abline\_h}] Add horizontal lines to the plot. Specify the lines as a vector e.g. c(2,3,5,6)  \code{Default=NULL}

\item[\code{add\_abline\_v}] Add vertical lines to the plot. Specify the lines as a vector e.g. c(2,3,5,6)  \code{Default=NULL}

\item[\code{add\_MTM\_peaks}] Add the MTM peak periods as horizontal lines \code{Default=FALSE}

\item[\code{add\_data}] Plot the data on top of the wavelet \code{Default=TRUE}

\item[\code{add\_avg}] Plot the average wavelet spectral power to the side of the wavelet \code{Default=FALSE}
\#'@param add\_pval add an transparent overlay on the wavelet scalogram based on the p-value and add the p-value curve to the
average spectral power curve. The p-value is based on a Monte Carlo simulation of the \code{\LinkA{analyze\_wavelet}{analyze.Rul.wavelet}} function.
The p-value is based on Monte Carlo modelling runs on surrogate data generated based on autocorrelated noise (red noise)
the calculated using a windowed (the window is half the size of the data set) temporal autocorrelation
and on shuffling the data set resulting in a random data sets which has similar spectral characteristics
to the original data set.The shuffling of the data set creates white noise which ensures that high amplitude high frequency/short
period cycles do not result in statistical significant peaks. The part of the data generated using the  autocorrelated noise (red noise)
based on the windowed  (the window is half the size of the data set) temporal autocorrelation represent a spectral signature similar to
to that of the original data. The original data might include spectral peaks which are the result of astronomical
forcing. The result is that the spectral power profile is biased towards rejecting the 0-hypothesis (e.g. no astronomical forcing).
By combining the shuffling of the data set with autocorrelated noise a surrogate data set is created which rejects
high amplitude high frequency/short period cycles and a reduced biased towards towards rejecting the 0-hypothesis if the data was
solely the result of autocorrelated noise. \code{Default=FALSE}

\item[\code{pval\_abline}] add ab-lines to the average spectral power plot which indicate certain p-values \code{Default=c(0.1,0.5)}

\item[\code{pval\_cutoff}] cutoff p-value to be used in the transparent overlay of the wavelet scalogram \code{Default=c(0.1)}

\item[\code{add\_MTM}] Add the MTM  plot next to the wavelet plot \code{Default=FALSE}

\item[\code{demean\_mtm}] Remove mean from data before conducting the MTM analysis \code{Default=TRUE}

\item[\code{detrend\_mtm}] Remove mean from data before conducting the MTM analysis \code{Default=TRUE}

\item[\code{padfac\_mtm}] Pad factor for the MTM analysis \code{Default=5}

\item[\code{tbw\_mtm}] time bandwidth product of the MTM analysis  \code{Default=3}

\item[\code{plot\_horizontal}] plot the wavelet horizontal or vertical eg y axis is depth or y axis power \code{Default=TRUE}
\end{ldescription}
\end{Arguments}
%
\begin{Value}
The output is a plot of a wavelet spectra.
if add\_MTM\_peaks = TRUE then the output of the MTM analysis will given as matrix
\end{Value}
%
\begin{Author}
Code based on the \LinkA{analyze.wavelet}{analyze.wavelet} and \LinkA{wt.image}{wt.image} functions of the 'WaveletComp' R package
and \LinkA{wt}{wt} function of the 'biwavelet' R package which are based on the
wavelet MATLAB code written by Christopher Torrence and Gibert P. Compo (1998).
The MTM analysis is from the astrochron R package of Meyers et al., (2012)
\end{Author}
%
\begin{References}
Angi Roesch and Harald Schmidbauer (2018). WaveletComp: Computational
Wavelet Analysis. R package version 1.1.
\url{https://CRAN.R-project.org/package=WaveletComp}

Gouhier TC, Grinsted A, Simko V (2021). R package biwavelet: Conduct Univariate and Bivariate Wavelet Analyses. (Version 0.20.21),
\url{https://github.com/tgouhier/biwavelet}

Torrence, C., and G. P. Compo. 1998. A Practical Guide to Wavelet Analysis.
Bulletin of the American Meteorological Society 79:61-78.
\url{https://paos.colorado.edu/research/wavelets/bams_79_01_0061.pdf}

Morlet, Jean, Georges Arens, Eliane Fourgeau, and Dominique Glard.
"Wave propagation and sampling theory—Part I: Complex signal and scattering in multilayered media.
" Geophysics 47, no. 2 (1982): 203-221.
\url{https://pubs.geoscienceworld.org/geophysics/article/47/2/203/68601/Wave-propagation-and-sampling-theory-Part-I}

J. Morlet, G. Arens, E. Fourgeau, D. Giard;
Wave propagation and sampling theory; Part II, Sampling theory and complex waves.
Geophysics 1982 47 (2): 222–236. \url{https://pubs.geoscienceworld.org/geophysics/article/47/2/222/68604/Wave-propagation-and-sampling-theory-Part-II}

S.R. Meyers, 2012, Seeing Red in Cyclic Stratigraphy: Spectral Noise Estimation for
Astrochronology: Paleoceanography, 27, PA3228, <doi:10.1029/2012PA002307>
\end{References}
%
\begin{Examples}
\begin{ExampleCode}

#Example 1. A plot of a wavelet spectra using the Total Solar Irradiance
# data set of Steinhilber et al., (2012)

TSI_wt <-
 analyze_wavelet(
   data = TSI,
   dj = 1/200,
   lowerPeriod = 16,
   upperPeriod = 8192,
   verbose = FALSE,
   omega_nr = 6
 )

plot_wavelet(
 wavelet = TSI_wt,
 lowerPeriod = NULL,
 upperPeriod = NULL,
 n.levels = 100,
 palette_name = "rainbow",
color_brewer= "grDevices",
 useRaster = TRUE,
 periodlab = "Period (metres)",
 x_lab = "depth (metres)",
 keep_editable = FALSE,
 dev_new=TRUE,
 plot_dir = TRUE,
 add_lines = NULL,
 add_points= NULL,
 add_abline_h = NULL,
 add_abline_v = NULL,
 add_MTM_peaks = FALSE,
 add_data = TRUE,
 add_avg = TRUE,
 add_pval = FALSE,
 pval_abline = c(0.1,0.05),
 pval_cutoff = c(0.1),
 add_MTM = FALSE,
 siglvl = 0.95,
 demean_mtm = TRUE,
 detrend_mtm = TRUE,
 padfac_mtm = 5,
 tbw_mtm = 3,
 plot_horizontal=TRUE)

#Example 2. A plot of a wavelet spectra using the magnetic susceptibility
#data set of Pas et al., (2018)
mag_wt <-
analyze_wavelet(
data = mag,
dj = 1/100,
lowerPeriod = 0.1,
upperPeriod = 254,
verbose = FALSE,
omega_nr = 10
)

plot_wavelet(
wavelet = mag_wt,
lowerPeriod = NULL,
upperPeriod = NULL,
n.levels = 100,
palette_name = "rainbow",
color_brewer= "grDevices",
useRaster = TRUE,
periodlab = "Period (metres)",
x_lab = "depth (metres)",
keep_editable = FALSE,
dev_new=TRUE,
plot_dir = TRUE,
add_lines= NULL,
add_points= NULL,
add_abline_h = NULL,
add_abline_v = NULL,
add_MTM_peaks = FALSE,
add_data = TRUE,
add_avg = TRUE,
add_pval = FALSE,
pval_abline = c(0.1,0.05),
pval_cutoff = c(0.1),
add_MTM = FALSE,
siglvl = 0.95,
demean_mtm = TRUE,
detrend_mtm = TRUE,
padfac_mtm = 5,
tbw_mtm = 3,
plot_horizontal=TRUE)


#Example 3. A plot of a wavelet spectra using the greyscale
# data set of Zeeden et al., (2013)
grey_wt <-
 analyze_wavelet(
   data = grey,
   dj = 1/200,
   lowerPeriod = 0.02,
   upperPeriod = 256,
   verbose = FALSE,
   omega_nr = 8
 )

plot_wavelet(
wavelet = grey_wt,
lowerPeriod = NULL,
upperPeriod = NULL,
n.levels = 100,
palette_name = "rainbow",
color_brewer= "grDevices",
useRaster = TRUE,
periodlab = "Period (metres)",
x_lab = "depth (metres)",
keep_editable = FALSE,
dev_new=TRUE,
plot_dir = TRUE,
add_lines = NULL,
add_points= NULL,
add_abline_h = NULL,
add_abline_v = NULL,
add_MTM_peaks = FALSE,
add_data = TRUE,
add_avg = TRUE,
add_pval = FALSE,
pval_abline = c(0.1,0.05),
pval_cutoff = c(0.1),
add_MTM = FALSE,
siglvl = 0.95,
demean_mtm = TRUE,
detrend_mtm = TRUE,
padfac_mtm = 5,
tbw_mtm = 3,
plot_horizontal=TRUE)



\end{ExampleCode}
\end{Examples}
\inputencoding{utf8}
\HeaderA{plot\_win\_fft}{Plot windowed fft based spectral analysis results}{plot.Rul.win.Rul.fft}
%
\begin{Description}
The \code{\LinkA{plot\_win\_fft}{plot.Rul.win.Rul.fft}} function allows for the (re)plotting of the results of the \code{\LinkA{win\_fft}{win.Rul.fft}}
\end{Description}
%
\begin{Usage}
\begin{verbatim}
plot_win_fft(
  win_fft = NULL,
  x_lab = c("depth (m)"),
  y_lab = c("frequency cycle/metre"),
  plot_res = 1,
  perc_vis = 0,
  freq_max = NULL,
  freq_min = NULL,
  keep_editable = FALSE,
  palette_name = "rainbow",
  color_brewer = "grDevices",
  plot_horizontal = TRUE,
  dev_new = TRUE
)
\end{verbatim}
\end{Usage}
%
\begin{Arguments}
\begin{ldescription}
\item[\code{win\_fft}] list which is the results of the \code{\LinkA{win\_fft}{win.Rul.fft}}

\item[\code{x\_lab}] label for the y-axis \code{Default="depth"}

\item[\code{y\_lab}] label for the y-axis \code{Default="sedrate"}

\item[\code{plot\_res}] plot 1 of 8 options option 1: Amplitude matrix,
option 2: Power matrix,
option 3: Phase matrix,
option 4: AR1\_CL matrix,
option 5: AR1\_Fit matrix ,
option 6: AR1\_90\_power matrix,
option 7: AR1\_95\_power matrix,
option 8: AR1\_99\_power matrix, \code{Default=1}

\item[\code{perc\_vis}] Cutoff percentile when plotting \code{Default=0}

\item[\code{freq\_max}] Maximum frequency to plot

\item[\code{freq\_min}] Minimum frequency to plot

\item[\code{keep\_editable}] Keep option to add extra features after plotting  \code{Default=FALSE}

\item[\code{palette\_name}] Name of the color palette which is used for plotting.
The color palettes than can be chosen depends on which the R package is specified in
the color\_brewer parameter. The included R packages from which palettes can be chosen
from are; the 'RColorBrewer', 'grDevices', 'ColorRamps' and 'Viridis' R packages.
There are many options to choose from so please
read the documentation of these packages \code{Default=rainbow}.
The R package 'viridis' has the color palette options: “magma”, “plasma”,
“inferno”, “viridis”, “mako”, and “rocket”  and “turbo”
To see the color palette options of the The R pacakge 'RColorBrewer' run
the RColorBrewer::brewer.pal.info() function
The R package 'colorRamps' has the color palette options:"blue2green",
"blue2green2red", "blue2red",    "blue2yellow", "colorRamps",    "cyan2yellow",
"green2red", "magenta2green", "matlab.like", "matlab.like2" and    "ygobb"
The R package 'grDevices' has the built in  palette options:"rainbow",
"heat.colors", "terrain.colors","topo.colors" and "cm.colors"
To see even more color palette options of the The R pacakge 'grDevices' run
the grDevices::hcl.pals() function

\item[\code{color\_brewer}] Name of the R package from which the color palette is chosen from.
The included R packages from which palettes can be chosen
are; the RColorBrewer, grDevices, ColorRamps and Viridis R packages.
There are many options to choose from so please
read the documentation of these packages. "\code{Default=grDevices}

\item[\code{plot\_horizontal}] plot the wavelet horizontal or vertical eg y axis is depth or y axis power  \code{Default=TRUE}

\item[\code{dev\_new}] Opens a new plotting window to plot the plot, this guarantees  a "nice" looking plot however when plotting in an R markdown
document the plot might not plot  \code{Default=TRUE}
\end{ldescription}
\end{Arguments}
%
\begin{Value}
Returns a plot of, which plot 1 of 8 options,
option 1: Amplitude matrix
option 2: Power matrix
option 3: Phase matrix
option 4: AR1\_CL matrix
option 5: AR1\_Fit matrix
option 6: AR1\_90\_power matrix
option 7: AR1\_95\_power matrix
option 8: AR1\_99\_power matrix
\end{Value}
%
\begin{Examples}
\begin{ExampleCode}

#Conduct a windowed fft on the magnetic susceptibility record \cr
# of the Sullivan core of Pas et al., (2018).

mag_win_fft <- win_fft(data= mag,
                   padfac = 5,
                   window_size = 12.5,
                   run_multicore = FALSE,
                   genplot = FALSE,
                   palette_name = "rainbow",
                   color_brewer="grDevices",
                   x_lab = c("depth (m)"),
                   y_lab = c("frequency cycle/meter"),
                   plot_res = 1,
                   perc_vis = 0.5,
                   freq_max = 5,
                   freq_min = 0.001,
                   keep_editable=FALSE,
                   verbose=FALSE)

# Plot the amplitude spectra
plot_win_fft(win_fft= mag_win_fft,
x_lab = c("depth (m)"),
y_lab = c("frequency cycle/meter"),
plot_res = 1,
perc_vis = 0.5,
freq_max = 5,
freq_min = 0.001,
keep_editable=FALSE,
palette_name = "rainbow",
color_brewer="grDevices",
plot_horizontal=TRUE,
dev_new=TRUE)



\end{ExampleCode}
\end{Examples}
\inputencoding{utf8}
\HeaderA{retrack\_wt\_MC}{Re-track cycles using a Monte-Carlo simulation}{retrack.Rul.wt.Rul.MC}
%
\begin{Description}
When analyzing multi-proxy records an age-model can be created
for each proxy. These age-models can be in general agreement but might also
indicate conflicting deposition rates. Picking one age-model out of the all
multi-proxy age-models and stating that, that age-model is the best overlooks
the information contained within the other proxies and hence a degree of error
remains the age-model exists.To combine the multiple age-models all the age
models can be averaged out and the uncertainty can be calculated by means
of the standard deviation. The result is an age-model which takes into
account all the age-models from the proxy records. The averaged out age-model
does not take into account any small user errors during  the creation of the
individual age-models nor does the averaging take into account the differences
between the age-models and how the initial age-model of a certain proxy might
be off in certain intervals. the link[WaverideR]retrack\_wt\_MC mitigates
these problems by re-tracking periods of astronomical cycles in the wavelet
spectra. The re-tracking is based on the information provided by the
age-models constructed from the different proxy records.
First a synthetic tracked curve is created by adding up fractions (0-1) of
the tracked periods of the different proxy records. This synthetic curve is
then used to re-track the period/spectral peaks of an astronomical cycle
in a randomly select wavelet scalogram. This process is repeated x times.
The result x tracked curves which take into account all the original
age-models. From the re-tracked curves one can calculate the mean period and
the standard deviation. The resulting  standard deviation is a good
indicator of the quality of the imprint of of astronomical cycles in the
proxy records. A small standard deviation indicates that given the input
of the different tracked cycles similar periods keep on being tracked
indicating the an astronomical is well recorded in the proxy records and as
such the age-model is very reliable in set interval. A high standard
deviation on the other hand means that the tracking results in vastly
different periods of the tracked astronomical cycle, as such the quality of
the imprint of the astronomical cycle proxy records is poor and hence the
age-model is less-reliable in this interval.
\end{Description}
%
\begin{Usage}
\begin{verbatim}
retrack_wt_MC(
  wt_list = NULL,
  data_track = NULL,
  x_axis = NULL,
  nr_simulations = 50,
  seed_nr = 1337,
  verbose = FALSE,
  genplot = FALSE,
  keep_editable = FALSE,
  create_GIF = FALSE,
  plot_GIF = FALSE,
  width_plt = 600,
  height_plt = 450,
  period_up = 1.5,
  period_down = 0.5,
  plot.COI = TRUE,
  n.levels = 100,
  palette_name = "rainbow",
  color_brewer = "grDevices",
  periodlab = "Period (metres)",
  x_lab = "depth (metres)",
  add_avg = FALSE,
  time_dir = TRUE,
  file_name = NULL,
  run_multicore = FALSE,
  output = 1,
  n_imgs = 50,
  plot_horizontal = TRUE,
  empty_folder = FALSE
)
\end{verbatim}
\end{Usage}
%
\begin{Arguments}
\begin{ldescription}
\item[\code{wt\_list}] a list containing all the wavelet objects created using the
link[WaverideR]analyze\_wavelet wavelet function
To create a list use the link[base]list function

\item[\code{data\_track}] a matrix containing all the tracked period values.
To create the matrix use the link[base]cbind function and only add the
tracked period values so do not add the depth axis. When combining the
tracked period values make sure that all curves have a similar depth
spacing.

\item[\code{x\_axis}] The x-axis of the tracked period values

\item[\code{nr\_simulations}] The number of Monte-Carlo simulations which are to be
conducted\code{Default=50}

\item[\code{seed\_nr}] The seed number of the Monte-Carlo simulations.
\code{Default=1337}

\item[\code{verbose}] Print text when running the function  \code{Default=FALSE}

\item[\code{genplot}] Plot a plot with the mean period and + and - standard
deviation \code{Default=FALSE}

\item[\code{keep\_editable}] Keep option to add extra features after plotting
\code{Default=FALSE}

\item[\code{create\_GIF}] Create a GIF with the re-tracked lines in the wavelet
scalograms \code{Default=FALSE}

\item[\code{plot\_GIF}] Plot a GIF with the re-tracked lines in the wavelet
scalograms\code{Default=FALSE}

\item[\code{width\_plt}] width of the re-tracked plot \code{Default=600}

\item[\code{height\_plt}] width of the re-tracked plot \code{Default=450}

\item[\code{period\_up}] The period\_up parameter is the factor with which the linear interpolated tracked\_curve
curve is multiplied by. This linear interpolated tracked\_curve multiplied by the period\_up factor is
the upper boundary which is used  for detecting the spectral peak nearest to the linear interpolated tracked\_curve
curve. If no spectral peak is detected within the specified boundary the interpolated
value is used instead. between spectral peaks \code{Default=1.5},

\item[\code{period\_down}] The period\_down parameter is the factor with which the linear interpolated tracked\_curve
curve is multiplied by. This linear interpolated tracked\_curve multiplied by the period\_down factor is
the lower boundary which is used  for detecting the spectral peak nearest to the linear interpolated tracked\_curve
curve. If no spectral peak is detected within the specified boundary the interpolated
value is used instead. between spectral peaks \code{Default=0.5},

\item[\code{plot.COI}] Option to plot the cone of influence \code{Default=TRUE}.

\item[\code{n.levels}] Number of color levels \code{Default=100}.

\item[\code{palette\_name}] Name of the color palette which is used for plotting.
The color palettes than can be chosen depends on which the R package is specified in
the color\_brewer parameter. The included R packages from which palettes can be chosen
from are; the 'RColorBrewer', 'grDevices', 'ColorRamps' and 'Viridis' R packages.
There are many options to choose from so please
read the documentation of these packages \code{Default=rainbow}.
The R package 'viridis' has the color palette options: “magma”, “plasma”,
“inferno”, “viridis”, “mako”, and “rocket”  and “turbo”
To see the color palette options of the The R pacakge 'RColorBrewer' run
the RColorBrewer::brewer.pal.info() function
The R package 'colorRamps' has the color palette options:"blue2green",
"blue2green2red", "blue2red",    "blue2yellow", "colorRamps",    "cyan2yellow",
"green2red", "magenta2green", "matlab.like", "matlab.like2" and    "ygobb"
The R package 'grDevices' has the built in  palette options:"rainbow",
"heat.colors", "terrain.colors","topo.colors" and "cm.colors"
To see even more color palette options of the The R pacakge 'grDevices' run
the grDevices::hcl.pals() function

\item[\code{color\_brewer}] Name of the R package from which the color palette is chosen from.
The included R packages from which palettes can be chosen
are; the RColorBrewer, grDevices, ColorRamps and Viridis R packages.
There are many options to choose from so please
read the documentation of these packages. "\code{Default=grDevices}

\item[\code{periodlab}] Label for the y-axis \code{Default="Period (metres)"}.

\item[\code{x\_lab}] Label for the x-axis \code{Default="depth (metres)"}.

\item[\code{add\_avg}] Plot the average wavelet spectral power to the side of the wavelet \code{Default=FALSE}

\item[\code{time\_dir}] The direction of the proxy record which is assumed for tuning if time increases with increasing depth/time values
(e.g. bore hole data which gets older with increasing depth ) then time\_dir should be set to TRUE
if time decreases with depth/time values (eg stratospheric logs where 0m is the bottom of the section)
then time\_dir should be set to FALSE \code{time\_dir=TRUE}

\item[\code{file\_name}] Name of the images created using this function. Each file
gets a number added to it which corresponds to which number of simulation it was
the files are saved in a folder with a similar name created in the current
directory

\item[\code{run\_multicore}] Run function using multiple cores \code{Default="FALSE"}

\item[\code{output}] \#'If output = 1, output is a list which contain 3 objects.
object 1 is a matrix with the x-axis and the mean tracked frequency and
standard deviation. \#'object 2 is a matrix with all the tracked periods.
Object 3 is a GIF in which \#'all the tracked periods are plotted.
If output = 2, output is a list which contain 2 objects. object 1 is a matrix
with the x-axis and the mean tracked frequency and standard deviation.
object 2 is a matrix with all the tracked periods.
If output = 3, output is a list which contain 2 objects. object 1 is a matrix
with the x-axis and the mean tracked frequency and standard deviation.
Object 2 is a GIF in which
all the tracked periods are plotted.
If output = 4, output is a list which contain 3 objects. Object 1 is a matrix
with all the tracked periods. Object 2 is a GIF in which  all the tracked
periods are plotted.
If output = 4 output is a list which contain 3 objects. Object 1 is a matrix
with all the tracked periods. Object 2 is a GIF in which  all the tracked
periods are plotted.
If output = 5 a matrix with the x-axis and the mean tracked frequency and
standard deviation is returned.
If output = 6, a matrix with all the tracked periods is returned.
If output = 7, a GIF in which all the tracked periods are plotted is returned.
\code{Default=1}

\item[\code{n\_imgs}] Number images used in creating the GIF a high number of images
is computationally intensive and will create a large sized GIF \code{Default=50}

\item[\code{plot\_horizontal}] plot the wavelet horizontal or vertical eg y axis
is depth or y axis power \code{Default=TRUE}

\item[\code{empty\_folder}] Empty the folder in which the images created using
this function are saved \code{Default=FALSE}
\end{ldescription}
\end{Arguments}
%
\begin{Value}
The output depends on the output setting
If genplot = TRUE a plot will be generated in which the mean period and
standard deviation is plotted
if plot\_GIF = TRUE a GIF with n number of n\_imgs will be plotted in which the
retraced curve is plotted in a wavelet scalogram
If output = 1, output is a list which contain 3 objects. object 1 is a matrix
with the x-axis and the mean tracked frequency and standard deviation.
object 2 is a matrix with all the tracked periods. Object 3 is a GIF in which
all the tracked periods are plotted.
If output = 2, output is a list which contain 2 objects. object 1 is a matrix
with the x-axis and the mean tracked frequency and standard deviation.
object 2 is a matrix with all the tracked periods.
If output = 3, output is a list which contain 2 objects. object 1 is a matrix
with the x-axis and the mean tracked frequency and standard deviation.
Object 2 is a GIF in which
all the tracked periods are plotted.
If output = 4, output is a list which contain 3 objects. Object 1 is a matrix
with all the tracked periods. Object 2 is a GIF in which  all the tracked
periods are plotted.
If output = 4 output is a list which contain 3 objects. Object 1 is a matrix
with all the tracked periods. Object 2 is a GIF in which  all the tracked
periods are plotted.
If output = 5 a matrix with the x-axis and the mean tracked period and
standard deviation is returned.
If output = 6, a matrix with all the tracked periods is returned.
If output = 7, a GIF in which all the tracked periods are plotted is returned
\end{Value}
%
\begin{Examples}
\begin{ExampleCode}

# Re-track the 110kyr eccentricity cycle in the wavelet scalogram
# from the XRF record of the Bisciaro data set of Arts (2014)

Bisciaro_al <- Bisciaro_XRF[, c(1, 61)]
Bisciaro_al <- astrochron::sortNave(Bisciaro_al,verbose=FALSE,genplot=FALSE)
Bisciaro_al <- astrochron::linterp(Bisciaro_al, dt = 0.01,verbose=FALSE,genplot=FALSE)
Bisciaro_al <- Bisciaro_al[Bisciaro_al[, 1] > 2, ]

Bisciaro_al_wt <-
 analyze_wavelet(
   data = Bisciaro_al,
   dj = 1 /200 ,
   lowerPeriod = 0.01,
   upperPeriod = 50,
   verbose = FALSE,
   omega_nr = 8
 )

# Bisciaro_al_wt_track <-
#   track_period_wavelet(
#     astro_cycle = 110,
#     wavelet = Bisciaro_al_wt,
#     n.levels = 100,
#     periodlab = "Period (metres)",
#     x_lab = "depth (metres)"
#   )
#
# Bisciaro_al_wt_track <- completed_series(
#   wavelet = Bisciaro_al_wt,
#   tracked_curve = Bisciaro_al_wt_track,
#   period_up = 1.2,
#   period_down = 0.8,
#   extrapolate = TRUE,
#   genplot = FALSE,
#   keep_editable = FALSE
# )
#
# Bisciaro_al_wt_track <-
#   loess_auto(
#     time_series = Bisciaro_al_wt_track,
#     genplot = FALSE,
#     print_span = FALSE,
#     keep_editable = FALSE
#   )

Bisciaro_ca <- Bisciaro_XRF[, c(1, 55)]
Bisciaro_ca <- astrochron::sortNave(Bisciaro_ca,verbose=FALSE,genplot=FALSE)
Bisciaro_ca <- astrochron::linterp(Bisciaro_ca, dt = 0.01,verbose=FALSE,genplot=FALSE)
Bisciaro_ca <- Bisciaro_ca[Bisciaro_ca[, 1] > 2, ]

Bisciaro_ca_wt <-
 analyze_wavelet(
   data = Bisciaro_ca,
   dj = 1 /200 ,
   lowerPeriod = 0.01,
   upperPeriod = 50,
   verbose = FALSE,
   omega_nr = 8
 )

# Bisciaro_ca_wt_track <-
#   track_period_wavelet(
#     astro_cycle = 110,
#     wavelet = Bisciaro_ca_wt,
#     n.levels = 100,
#     periodlab = "Period (metres)",
#     x_lab = "depth (metres)"
#   )
#
# Bisciaro_ca_wt_track <- completed_series(
#   wavelet = Bisciaro_ca_wt,
#   tracked_curve = Bisciaro_ca_wt_track,
#   period_up = 1.2,
#   period_down = 0.8,
#   extrapolate = TRUE,
#   genplot = FALSE,
#   keep_editable = FALSE
# )
#
# Bisciaro_ca_wt_track <-
#   loess_auto(
#     time_series = Bisciaro_ca_wt_track,
#     genplot = FALSE,
#     print_span = FALSE,
#     keep_editable = FALSE)

Bisciaro_sial <- Bisciaro_XRF[,c(1,64)]
Bisciaro_sial <- astrochron::sortNave(Bisciaro_sial,verbose=FALSE,genplot=FALSE)
Bisciaro_sial <- astrochron::linterp(Bisciaro_sial, dt = 0.01,verbose=FALSE,genplot=FALSE)
Bisciaro_sial <- Bisciaro_sial[Bisciaro_sial[, 1] > 2, ]

Bisciaro_sial_wt <-
 analyze_wavelet(
   data = Bisciaro_sial,
   dj = 1 /200 ,
   lowerPeriod = 0.01,
   upperPeriod = 50,
   verbose = FALSE,
   omega_nr = 8
 )

# Bisciaro_sial_wt_track <-
#   track_period_wavelet(
#     astro_cycle = 110,
#     wavelet = Bisciaro_sial_wt,
#     n.levels = 100,
#     periodlab = "Period (metres)",
#     x_lab = "depth (metres)"
#   )
#
#
# Bisciaro_sial_wt_track <- completed_series(
#   wavelet = Bisciaro_sial_wt,
#   tracked_curve = Bisciaro_sial_wt_track,
#   period_up = 1.2,
#   period_down = 0.8,
#   extrapolate = TRUE,
#   genplot = FALSE,
#   keep_editable = FALSE
# )
#
# Bisciaro_sial_wt_track <-
#   loess_auto(
#     time_series = Bisciaro_sial_wt_track,
#     genplot = FALSE,
#     print_span = FALSE,
#     keep_editable = FALSE
#   )


Bisciaro_Mn <- Bisciaro_XRF[,c(1,46)]
Bisciaro_Mn <- astrochron::sortNave(Bisciaro_Mn,verbose=FALSE,genplot=FALSE)
Bisciaro_Mn <- astrochron::linterp(Bisciaro_Mn, dt = 0.01,verbose=FALSE,genplot=FALSE)
Bisciaro_Mn <- Bisciaro_Mn[Bisciaro_Mn[, 1] > 2, ]

Bisciaro_Mn_wt <-
 analyze_wavelet(
   data = Bisciaro_Mn,
   dj = 1 /200 ,
   lowerPeriod = 0.01,
   upperPeriod = 50,
   verbose = FALSE,
   omega_nr = 8
 )

# Bisciaro_Mn_wt_track <-
#   track_period_wavelet(
#     astro_cycle = 110,
#     wavelet = Bisciaro_Mn_wt,
#     n.levels = 100,
#     periodlab = "Period (metres)",
#     x_lab = "depth (metres)"
#   )
#
#
# Bisciaro_Mn_wt_track <- completed_series(
#   wavelet = Bisciaro_Mn_wt,
#   tracked_curve = Bisciaro_Mn_wt_track,
#   period_up = 1.2,
#   period_down = 0.8,
#   extrapolate = TRUE,
#   genplot = FALSE,
#   keep_editable = FALSE
# )
# Bisciaro_Mn_wt_track <-
#   loess_auto(
#     time_series = Bisciaro_Mn_wt_track,
#     genplot = FALSE,
#     print_span = FALSE,
#     keep_editable = FALSE
#   )

Bisciaro_Mg <- Bisciaro_XRF[,c(1,71)]
Bisciaro_Mg <- astrochron::sortNave(Bisciaro_Mg,verbose=FALSE,genplot=FALSE)
Bisciaro_Mg <- astrochron::linterp(Bisciaro_Mg, dt = 0.01,verbose=FALSE,genplot=FALSE)
Bisciaro_Mg <- Bisciaro_Mg[Bisciaro_Mg[, 1] > 2, ]

Bisciaro_Mg_wt <-
 analyze_wavelet(
   data = Bisciaro_Mg,
   dj = 1 /200 ,
   lowerPeriod = 0.01,
   upperPeriod = 50,
   verbose = FALSE,
   omega_nr = 8
 )

# Bisciaro_Mg_wt_track <-
#   track_period_wavelet(
#     astro_cycle = 110,
#     wavelet = Bisciaro_Mg_wt,
#     n.levels = 100,
#     periodlab = "Period (metres)",
#     x_lab = "depth (metres)"
#   )
#
#
# Bisciaro_Mg_wt_track <- completed_series(
#   wavelet = Bisciaro_Mg_wt,
#   tracked_curve = Bisciaro_Mg_wt_track,
#   period_up = 1.2,
#   period_down = 0.8,
#   extrapolate = TRUE,
#   genplot = FALSE,
#   keep_editable = FALSE
# )
#
# Bisciaro_Mg_wt_track <-
#   loess_auto(
#     time_series = Bisciaro_Mg_wt_track,
#     genplot = FALSE,
#     print_span = FALSE,
#     keep_editable = FALSE)




wt_list_bisc <- list(Bisciaro_al_wt,
               Bisciaro_ca_wt,
               Bisciaro_sial_wt,
               Bisciaro_Mn_wt,
               Bisciaro_Mg_wt)

#Instead of tracking, the tracked solution data sets Bisciaro_al_wt_track,
#Bisciaro_ca_wt_track, Bisciaro_sial_wt_track, Bisciaro_Mn_wt_track,
# Bisciaro_Mn_wt_track and Bisciaro_Mg_wt_track are used

data_track_bisc <- cbind(Bisciaro_al_wt_track[,2],
                     Bisciaro_ca_wt_track[,2],
                     Bisciaro_sial_wt_track[,2],
                     Bisciaro_Mn_wt_track[,2],
                     Bisciaro_Mg_wt_track[,2])

x_axis_bisc <- Bisciaro_al_wt_track[,1]


bisc_retrack <- retrack_wt_MC(wt_list = wt_list_bisc,
             data_track = data_track_bisc,
             x_axis = x_axis_bisc,
             nr_simulations = 20,
             seed_nr = 1337,
             verbose = FALSE,
             genplot = FALSE,
             keep_editable = FALSE,
             create_GIF = FALSE,
             plot_GIF = FALSE,
             width_plt =  600,
             height_plt = 450,
            period_up  =  1.5,
             period_down = 0.5,
             plot.COI = TRUE,
             n.levels = 100,
             palette_name = "rainbow",
             color_brewer = "grDevices",
             periodlab = "Period (metres)",
             x_lab = "depth (metres)",
             add_avg = FALSE,
             time_dir = TRUE,
             file_name = NULL,
             run_multicore = FALSE,
             output = 1,
             n_imgs = 50,
             plot_horizontal = TRUE,
             empty_folder = FALSE)




\end{ExampleCode}
\end{Examples}
\inputencoding{utf8}
\HeaderA{sedrate2tune}{Use a sedimentation curve to convert data to the time domain}{sedrate2tune}
%
\begin{Description}
Convert a proxy record from the depth to time domain using
a sedimentation rate curve
\end{Description}
%
\begin{Usage}
\begin{verbatim}
sedrate2tune(
  data = NULL,
  sed_curve = NULL,
  genplot = FALSE,
  keep_editable = FALSE
)
\end{verbatim}
\end{Usage}
%
\begin{Arguments}
\begin{ldescription}
\item[\code{data}] Input should be a matrix of 2 columns with first column being depth and the second column
is a proxy value

\item[\code{sed\_curve}] Input should be a matrix of 2 columns with first column being depth and the second column
is the sedimentation rate is cm/kyr

\item[\code{genplot}] Generates a plot of the proxy record in  the time domain \code{Default=FALSE}.

\item[\code{keep\_editable}] Keep option to add extra features after plotting  \code{Default=FALSE}
\end{ldescription}
\end{Arguments}
%
\begin{Value}
The output is a matrix with 2 columns.
The first column is time
The second column is the proxy value
If \code{genplot=TRUE} then a time vs proxy value plot will be plotted.
\end{Value}
%
\begin{Author}
Part of the code is based on the \LinkA{sedrate2time}{sedrate2time}
function of the 'astrochron' R package
\end{Author}
%
\begin{References}
Routines for astrochronologic testing, astronomical time scale construction, and
time series analysis <doi:10.1016/j.earscirev.2018.11.015>
\end{References}
%
\begin{Examples}
\begin{ExampleCode}

# Extract the 405kyr eccentricity cycle from the wavelet scalogram
# from the magnetic susceptibility record of the Sullivan core
# of Pas et al., (2018) and then create a age model using minimal tuning
# (e.g.) set the distance between peaks to 405 kyr. The age model
# (sedimentation rate curve) is then used to convert the data
# from the depth to the time domain

mag_wt <- analyze_wavelet(data = mag,
dj = 1/100,
lowerPeriod = 0.1,
upperPeriod = 254,
verbose = FALSE,
omega_nr = 10)


mag_405 <- extract_signal_stable_V2(
 wavelet = mag_wt,
 period_max = 4,
 period_min = 2,
 add_mean = TRUE,
 plot_residual = FALSE,
 keep_editable = FALSE
)

mag_405_min_tuning <- minimal_tuning(data = mag_405,
pts = 5,
cycle = 405,
tune_opt = "max",
output = 1,
genplot = FALSE,
keep_editable = FALSE)

mag_time <- sedrate2tune(
data=mag,
sed_curve=mag_405_min_tuning,
genplot=FALSE,
keep_editable=FALSE)


\end{ExampleCode}
\end{Examples}
\inputencoding{utf8}
\HeaderA{sum\_power\_sedrate}{Calculate sum of maximum spectral power for sedimentation rates for a wavelet spectra}{sum.Rul.power.Rul.sedrate}
%
\begin{Description}
The \code{\LinkA{sum\_power\_sedrate}{sum.Rul.power.Rul.sedrate}} function is used calculate the sum of
maximum spectral power for a list of astronomical cycles from a wavelet spectra.
The data is first normalized using the average spectral power curves
for a given percentile based on results of the \code{\LinkA{model\_red\_noise\_wt}{model.Rul.red.Rul.noise.Rul.wt}} function
\end{Description}
%
\begin{Usage}
\begin{verbatim}
sum_power_sedrate(
  red_noise = NULL,
  wavelet = NULL,
  percentile = NULL,
  sedrate_low = NULL,
  sedrate_high = NULL,
  spacing = NULL,
  cycles = c(NULL),
  x_lab = "depth",
  y_lab = "sedrate",
  run_multicore = FALSE,
  genplot = FALSE,
  plot_res = 1,
  keep_editable = FALSE,
  palette_name = "rainbow",
  color_brewer = "grDevices",
  verbose = FALSE
)
\end{verbatim}
\end{Usage}
%
\begin{Arguments}
\begin{ldescription}
\item[\code{red\_noise}] Red noise curves generated using the \code{\LinkA{model\_red\_noise\_wt}{model.Rul.red.Rul.noise.Rul.wt}} function

\item[\code{wavelet}] Wavelet object created using the \code{\LinkA{analyze\_wavelet}{analyze.Rul.wavelet}} function

\item[\code{percentile}] Percentile value (0-1) of the rednoise runs which is used to normalize the data for.
To account for the distribution/distortion of the spectral power distribution based on the analytical technique and
random red-noise the data is normalized against a percentile based red-noise curve which is the results of the
'\code{\LinkA{model\_red\_noise\_wt}{model.Rul.red.Rul.noise.Rul.wt}} modelling runs.

\item[\code{sedrate\_low}] Minimum sedimentation rate (cm/kyr)for which the sum of maximum spectral power is calculated for.

\item[\code{sedrate\_high}] Maximum sedimentation rate (cm/kyr) for which the sum of maximum spectral power is calculated  for.

\item[\code{spacing}] Spacing (cm/kyr) between sedimentation rates

\item[\code{cycles}] Astronomical cycles (in kyr) for which the combined sum of maximum spectral power is calculated for

\item[\code{x\_lab}] label for the y-axis \code{Default="depth"}

\item[\code{y\_lab}] label for the y-axis \code{Default="sedrate"}

\item[\code{run\_multicore}] run simulation using multiple cores \code{Default=FALSE}
the simulation is run at x-2 cores to allow the 2 remaining processes to run background processes

\item[\code{genplot}] Generate plot \code{Default="FALSE"}

\item[\code{plot\_res}] plot options are 1: sum max power or 2: nr of components \code{Default=2}

\item[\code{keep\_editable}] Keep option to add extra features after plotting  \code{Default=FALSE}

\item[\code{palette\_name}] Name of the color palette which is used for plotting.
The color palettes than can be chosen depends on which the R package is specified in
the color\_brewer parameter. The included R packages from which palettes can be chosen
from are; the 'RColorBrewer', 'grDevices', 'ColorRamps' and 'Viridis' R packages.
There are many options to choose from so please
read the documentation of these packages \code{Default=rainbow}.
The R package 'viridis' has the color palette options: “magma”, “plasma”,
“inferno”, “viridis”, “mako”, and “rocket”  and “turbo”
To see the color palette options of the The R pacakge 'RColorBrewer' run
the RColorBrewer::brewer.pal.info() function
The R package 'colorRamps' has the color palette options:"blue2green",
"blue2green2red", "blue2red",    "blue2yellow", "colorRamps",    "cyan2yellow",
"green2red", "magenta2green", "matlab.like", "matlab.like2" and    "ygobb"
The R package 'grDevices' has the built in  palette options:"rainbow",
"heat.colors", "terrain.colors","topo.colors" and "cm.colors"
To see even more color palette options of the The R pacakge 'grDevices' run
the grDevices::hcl.pals() function

\item[\code{color\_brewer}] Name of the R package from which the color palette is chosen from.
The included R packages from which palettes can be chosen
are; the RColorBrewer, grDevices, ColorRamps and Viridis R packages.
There are many options to choose from so please
read the documentation of these packages. "\code{Default=grDevices}

\item[\code{verbose}] Print text \code{Default=FALSE}.
\end{ldescription}
\end{Arguments}
%
\begin{Value}
Returns a list which contains 4 elements
element 1: sum of maximum spectral power
element 2: number of cycles used in the sum of maximum spectral power
element 3: y-axis values of the matrices which is sedimentation rate
element 4: x-axis values of the matrices which is depth

If \code{Default="TRUE"} a plot is created with 3 subplots.
Subplot 1 is plot in which the the sum of maximum spectral power for a
given sedimentation rate or nr of cycles is plotted for each depth given depth.
Subplot 2 is a plot in which the average sum of maximum spectral power is plotted fro each sedimentation
Subplot 3 is a color scale for subplot 1.
\end{Value}
%
\begin{Author}
Based on the \LinkA{asm}{asm} and \LinkA{eAsm}{eAsm}
functions of the 'astrochron' R package
and the 'eCOCO' and 'COCO' functions of the 'Acycle' software
\end{Author}
%
\begin{References}
Routines for astrochronologic testing, astronomical time scale construction, and
time series analysis <doi:10.1016/j.earscirev.2018.11.015>

Acycle: Time-series analysis software for paleoclimate research and education,
Mingsong Li, Linda Hinnov, Lee Kump,
Computers \& Geosciences,Volume 127,2019,Pages 12-22,ISSN 0098-3004,
<doi:10.1016/j.cageo.2019.02.011>

Tracking variable sedimentation rates and astronomical forcing in Phanerozoic paleoclimate proxy series with evolutionary correlation coefficients and hypothesis testing,
Mingsong Li, Lee R. Kump, Linda A. Hinnov, Michael E. Mann,
Earth and Planetary Science Letters,Volume 501,
T2018,Pages 165-179,ISSN 0012-821X,<doi:10.1016/j.epsl.2018.08.041>
\end{References}
%
\begin{Examples}
\begin{ExampleCode}

#estimate sedimentation rate for the the magnetic susceptibility record
# of the Sullivan core of Pas et al., (2018).

mag_wt <- analyze_wavelet(data = mag,
dj = 1/100,
lowerPeriod = 0.1,
upperPeriod = 254,
verbose = FALSE,
omega_nr = 10)

#increase n_simulations to better define the red noise spectral power curve
mag_wt_red_noise <- model_red_noise_wt(wavelet=mag_wt,
n_simulations=10,
run_multicore=FALSE,
verbose=FALSE)

sedrates <- sum_power_sedrate(red_noise=mag_wt_red_noise,
wavelet=mag_wt,
percentile=0.75,
sedrate_low = 0.5,
sedrate_high = 4,
spacing = 0.05,
cycles = c(2376,1600,1180,696,406,110),
x_lab="depth",
y_lab="sedrate",
run_multicore=FALSE,
genplot = FALSE,
plot_res=1,
keep_editable=FALSE,
palette_name = "rainbow",
color_brewer="grDevices",
verbose=FALSE)



\end{ExampleCode}
\end{Examples}
\inputencoding{utf8}
\HeaderA{track\_period\_wavelet}{Track the period of a cycle in a wavelet spectra}{track.Rul.period.Rul.wavelet}
%
\begin{Description}
Interactively select points in a wavelet spectra to trace a period in a wavelet spectra.
The \code{\LinkA{track\_period\_wavelet}{track.Rul.period.Rul.wavelet}} function plots a wavelet spectra in which spectral peaks can selected
allowing one to track a ridge hence one can track the a cycle with a changing period.
Tracking points can be selected in the Interactive interface and will be shown as white dots
when one wants to deselect a point the white dots can be re-clicked/re-selected and will turn red which
indicates that the previously selected point is deselected. Deselecting points can be quite tricky
due to the close spacing of  points and such the \code{\LinkA{delpts\_tracked\_period\_wt}{delpts.Rul.tracked.Rul.period.Rul.wt}} can be used to
delete points were previously selected using the \code{\LinkA{track\_period\_wavelet}{track.Rul.period.Rul.wavelet}} function.
\end{Description}
%
\begin{Usage}
\begin{verbatim}
track_period_wavelet(
  astro_cycle = 405,
  wavelet = NULL,
  n.levels = 100,
  track_peaks = TRUE,
  periodlab = "Period (metres)",
  x_lab = "depth (metres)",
  palette_name = "rainbow",
  color_brewer = "grDevices",
  plot_horizontal = TRUE
)
\end{verbatim}
\end{Usage}
%
\begin{Arguments}
\begin{ldescription}
\item[\code{astro\_cycle}] Duration (in kyr) of the cycle which traced.

\item[\code{wavelet}] Wavelet object created using the \code{\LinkA{analyze\_wavelet}{analyze.Rul.wavelet}} function.

\item[\code{n.levels}] Number of color levels \code{Default=100}.

\item[\code{track\_peaks}] Setting which indicates whether tracking is restricted
to spectral peaks (track\_peaks=TRUE) or whether any point within the wavelet
spectra can be selected (track\_peaks=FALSE) \code{Default=TRUE}.

\item[\code{periodlab}] label for the y-axis \code{Default="Period (metres)"}.

\item[\code{x\_lab}] label for the x-axis \code{Default="depth (metres)"}.

\item[\code{palette\_name}] Name of the color palette which is used for plotting.
The color palettes than can be chosen depends on which the R package is specified in
the color\_brewer parameter. The included R packages from which palettes can be chosen
from are; the 'RColorBrewer', 'grDevices', 'ColorRamps' and 'Viridis' R packages.
There are many options to choose from so please
read the documentation of these packages \code{Default=rainbow}.
The R package 'viridis' has the color palette options: “magma”, “plasma”,
“inferno”, “viridis”, “mako”, and “rocket”  and “turbo”
To see the color palette options of the The R pacakge 'RColorBrewer' run
the RColorBrewer::brewer.pal.info() function
The R package 'colorRamps' has the color palette options:"blue2green",
"blue2green2red", "blue2red",    "blue2yellow", "colorRamps",    "cyan2yellow",
"green2red", "magenta2green", "matlab.like", "matlab.like2" and    "ygobb"
The R package 'grDevices' has the built in  palette options:"rainbow",
"heat.colors", "terrain.colors","topo.colors" and "cm.colors"
To see even more color palette options of the The R pacakge 'grDevices' run
the grDevices::hcl.pals() function

\item[\code{color\_brewer}] Name of the R package from which the color palette is chosen from.
The included R packages from which palettes can be chosen
are; the RColorBrewer, grDevices, ColorRamps and Viridis R packages.
There are many options to choose from so please
read the documentation of these packages. "\code{Default=grDevices}

\item[\code{plot\_horizontal}] plot the wavelet horizontal or vertical eg y axis is depth or y axis power  \code{Default=TRUE}
\end{ldescription}
\end{Arguments}
%
\begin{Value}
Results of the tracking of a cycle in the wavelet spectra is a matrix with 3 columns.
The first column is depth/time
The second column is the period of the tracked cycle
The third column is the sedimentation rate based on the duration (in time) of the tracked cycle
\end{Value}
%
\begin{Author}
The function is based/inspired on the \LinkA{traceFreq}{traceFreq}
function of the 'astrochron' R package
\end{Author}
%
\begin{References}
Routines for astrochronologic testing, astronomical time scale construction, and
time series analysis <doi:10.1016/j.earscirev.2018.11.015>
\end{References}
%
\begin{Examples}
\begin{ExampleCode}

#Track the 405kyr eccentricity cycle in the magnetic susceptibility record
# of the Sullivan core of Pas et al., (2018)

mag_wt <- analyze_wavelet(data = mag,
dj = 1/100,
lowerPeriod = 0.1,
upperPeriod = 254,
verbose = FALSE,
omega_nr = 10)

mag_track <- track_period_wavelet(astro_cycle = 405,
                                  wavelet=mag_wt,
                                  n.levels = 100,
                                  track_peaks=TRUE,
                                  periodlab = "Period (metres)",
                                  x_lab = "depth (metres)",
                                 palette_name = "rainbow",
                                 color_brewer="grDevices",
                                 plot_horizontal=TRUE)


\end{ExampleCode}
\end{Examples}
\inputencoding{utf8}
\HeaderA{TSI}{Total solar irradiation data (0-9400ka) of steinhilber et al., (2012)}{TSI}
%
\begin{Description}
The Total solar irradiation data set consists of the TSI values of Steinhilber et al., (2012)
\end{Description}
%
\begin{Details}
Column 1: Age (kyr)\\{}
Column 2: Total solar Irradiation (TSI)\\{}
\end{Details}
%
\begin{References}
Steinhilber, Friedhelm \& Abreu, Jacksiel \& Beer, Juerg \& Brunner,
Irene \& Christl, Marcus \& Fischer, Hubertus \& Heikkilä, U. \& Kubik,
Peter \& Mann, Mathias \& Mccracken, K. \& Miller, Heinrich \& Miyahara,
Hiroko \& Oerter, Hans \& Wilhelms, Frank. (2012).
9,400 Years of cosmic radiation and solar activity from ice cores and tree rings.
Proceedings of the National Academy of Sciences of the United States of America.
109. 5967-71. 10.1073/pnas.1118965109.
<doi:10.1073/pnas.1118965109>
\end{References}
\inputencoding{utf8}
\HeaderA{wavelet\_uncertainty}{Calculate the uncertainty associated with the wavelet analysis based on the Gabor uncertainty principle}{wavelet.Rul.uncertainty}
%
\begin{Description}
The \code{\LinkA{wavelet\_uncertainty}{wavelet.Rul.uncertainty}} function is used to calculate uncertainties associated
with the wavelet analysis based on the Gabor uncertainty principle applied to the
continuous wavelet transform using a Morlet wavelet. The calculated uncertainty is the underlying
analytical uncertainty which is the result of applying the Gabor uncertainty principle to the
continuous wavelet transform using a Morlet wavelet.
\end{Description}
%
\begin{Usage}
\begin{verbatim}
wavelet_uncertainty(
  tracked_cycle = NULL,
  period_of_tracked_cycle = NULL,
  wavelet = NULL,
  multi = 1,
  verbose = FALSE,
  genplot_time = FALSE,
  genplot_uncertainty = FALSE,
  genplot_uncertainty_wt = FALSE,
  keep_editable = FALSE,
  palette_name = "rainbow",
  color_brewer = "grDevices"
)
\end{verbatim}
\end{Usage}
%
\begin{Arguments}
\begin{ldescription}
\item[\code{tracked\_cycle}] Curve of the cycle tracked using the \code{\LinkA{track\_period\_wavelet}{track.Rul.period.Rul.wavelet}} function
Any input (matrix or data frame) in which the first column is depth or
time and the second column is period should work

\item[\code{period\_of\_tracked\_cycle}] period of the tracked curve (in kyr).

\item[\code{wavelet}] wavelet object created using the \code{\LinkA{analyze\_wavelet}{analyze.Rul.wavelet}} function.

\item[\code{multi}] multiple of the standard deviation to be used for defining uncertainty \code{Default=1}.

\item[\code{verbose}] Print text \code{Default=FALSE}.

\item[\code{genplot\_time}] plot time curves with a upper and lower uncertainty based on Gabor uncertainty principle applied to the
continuous wavelet transform using a Morlet wavelet, which uses which uses the omega number (number
of cycles in the wavelet) at one standard deviation to define the analytical uncertainty \code{Default=TRUE}

\item[\code{genplot\_uncertainty}] Plot period curves with upper and lower uncertainty based on Gabor uncertainty principle applied to the
continuous wavelet transform using a Morlet wavelet, which uses which uses the omega number
(number of cycles in the wavelet) to define uncertainty at one standard deviation \code{Default=TRUE}

\item[\code{genplot\_uncertainty\_wt}] generate a wavelet plot with the uncertainty based on Gabor uncertainty
principle applied to the continuous wavelet transform using a Morlet wavelet superimposed on top of
original wavelet plot. The red curve is period of the tracked curve plus the analytical uncertainty.
The blue curve is period of the tracked curve min the analytical uncertainty.
The  black curve is the curve tracked using the '\code{Default=tracked\_cycle\_curve} function \code{Default=TRUE}

\item[\code{keep\_editable}] Keep option to add extra features after plotting  \code{Default=FALSE}

\item[\code{palette\_name}] Name of the color palette which is used for plotting.
The color palettes than can be chosen depends on which the R package is specified in
the color\_brewer parameter. The included R packages from which palettes can be chosen
from are; the 'RColorBrewer', 'grDevices', 'ColorRamps' and 'Viridis' R packages.
There are many options to choose from so please
read the documentation of these packages \code{Default=rainbow}.
The R package 'viridis' has the color palette options: “magma”, “plasma”,
“inferno”, “viridis”, “mako”, and “rocket”  and “turbo”
To see the color palette options of the The R pacakge 'RColorBrewer' run
the RColorBrewer::brewer.pal.info() function
The R package 'colorRamps' has the color palette options:"blue2green",
"blue2green2red", "blue2red",    "blue2yellow", "colorRamps",    "cyan2yellow",
"green2red", "magenta2green", "matlab.like", "matlab.like2" and    "ygobb"
The R package 'grDevices' has the built in  palette options:"rainbow",
"heat.colors", "terrain.colors","topo.colors" and "cm.colors"
To see even more color palette options of the The R pacakge 'grDevices' run
the grDevices::hcl.pals() function

\item[\code{color\_brewer}] Name of the R package from which the color palette is chosen from.
The included R packages from which palettes can be chosen
are; the RColorBrewer, grDevices, ColorRamps and Viridis R packages.
There are many options to choose from so please
read the documentation of these packages. "\code{Default=grDevices}
\end{ldescription}
\end{Arguments}
%
\begin{Value}
Results pertaining to the uncertainty calculated based on the Gabor uncertainty principle.\\{}
If the genplot\_time is TRUE then a depth time plot will be plotted with 3 lines, the mean age,age plus
x times the standard deviation and age minus x times the standard deviation . \\{}
If the genplot\_uncertainty is TRUE then a curve will be plotted with the mean period, the tracked period plus
x times the standard deviation and the tracked period minus x times the standard deviation. \\{}
If the genplot\_uncertainty\_wt is TRUE a wavelet spectra will be plotted with the tracked period, the tracked period plus
x times the standard deviation,the tracked period minus x times the standard deviation and the area in between will be shaded in grey.\\{}

Returns a matrix with 8 columns.\\{}
The first column is called "depth" eg. depth \\{}
The second column is "period" of the originally tracked period. \\{}
The third column is "frequency" of the originally tracked period. \\{}
The fourth column "uncertainty in frequency FWHM" is the uncertainty in frequency based on the Gabor uncertainty principle defined as
(FWHM) full width at half maximum. \\{}
The fifth column "uncertainty in frequency x\_times SD" is the uncertainty in frequency based on the Gabor uncertainty principle defined as
times x standard deviations. \\{}
The sixth column "time mean" is the mean time based on the tracked period. \\{}
The seventh column "time plus x\_times sd" is the time based on the tracked period plus x times the standard deviation. \\{}
The eight column "time min x\_times sd" is the time based on the tracked period min x times the standard deviation. \\{}
\end{Value}
%
\begin{Author}
Code based on the \LinkA{analyze.wavelet}{analyze.wavelet} function of the 'WaveletComp' R package
and \LinkA{wt}{wt} function of the 'biwavelet' R package which are based on the
wavelet 'MATLAB' code written by Christopher Torrence and Gibert P. Compo (1998).
The assignment of the standard deviation of the uncertainty of the wavelet
is based on the work of Gabor (1946) and Russell et al., (2016)
\end{Author}
%
\begin{References}
Angi Roesch and Harald Schmidbauer (2018). WaveletComp: Computational
Wavelet Analysis. R package version 1.1.
\url{https://CRAN.R-project.org/package=WaveletComp}

Gouhier TC, Grinsted A, Simko V (2021). R package biwavelet: Conduct Univariate and Bivariate Wavelet Analyses. (Version 0.20.21),
\url{https://github.com/tgouhier/biwavelet}

Torrence, C., and G. P. Compo. 1998. A Practical Guide to Wavelet Analysis.
Bulletin of the American Meteorological Society 79:61-78.
\url{https://paos.colorado.edu/research/wavelets/bams_79_01_0061.pdf}

Gabor, Dennis. "Theory of communication. Part 1: The analysis of information."
Journal of the Institution of Electrical Engineers-part III: radio and
communication engineering 93, no. 26 (1946): 429-441.\url{http://genesis.eecg.toronto.edu/gabor1946.pdf}

Russell, Brian, and Jiajun Han. "Jean Morlet and the continuous wavelet transform.
" CREWES Res. Rep 28 (2016): 115. \url{https://www.crewes.org/Documents/ResearchReports/2016/CRR201668.pdf}


Morlet, Jean, Georges Arens, Eliane Fourgeau, and Dominique Glard.
"Wave propagation and sampling theory—Part I: Complex signal and scattering in multilayered media.
" Geophysics 47, no. 2 (1982): 203-221.
\url{https://pubs.geoscienceworld.org/geophysics/article/47/2/203/68601/Wave-propagation-and-sampling-theory-Part-I}

J. Morlet, G. Arens, E. Fourgeau, D. Giard;
Wave propagation and sampling theory; Part II, Sampling theory and complex waves.
Geophysics 1982 47 (2): 222–236. \url{https://pubs.geoscienceworld.org/geophysics/article/47/2/222/68604/Wave-propagation-and-sampling-theory-Part-II}
\end{References}
%
\begin{Examples}
\begin{ExampleCode}

#calculate the Gabor uncertainty derived mathematical uncertainty of the
#magnetic susceptibility record of the Sullivan core of Pas et al., (2018)

mag_wt <- analyze_wavelet(data = mag,
dj = 1/100,
lowerPeriod = 0.1,
upperPeriod = 254,
verbose = FALSE,
omega_nr = 10)

#Track the 405 kyr eccentricity cycle in a wavelet spectra

#mag_track <- track_period_wavelet(astro_cycle = 405,
#                                   wavelet=mag_wt,
#                                   n.levels = 100,
#                                   periodlab = "Period (metres)",
#                                   x_lab = "depth (metres)",
#                                palette_name="rainbow",
#                                color_brewer= "grDevices")

#Instead of tracking, the tracked solution data set mag_track_solution is used
mag_track <- mag_track_solution

mag_track_complete <- completed_series(
  wavelet = mag_wt,
  tracked_curve = mag_track,
  period_up = 1.2,
  period_down = 0.8,
  extrapolate = FALSE,
  genplot = FALSE,
  keep_editable=FALSE
)

mag_track_complete <- loess_auto(time_series = mag_track_complete,
 genplot = FALSE, print_span = FALSE,keep_editable=FALSE)

uncertainty <- wavelet_uncertainty(
  tracked_cycle = mag_track_complete,
  period_of_tracked_cycle = 405,
  wavelet = mag_wt,
  multi=1,
  verbose = FALSE,
  genplot_time = FALSE,
  genplot_uncertainty = FALSE,
  genplot_uncertainty_wt = FALSE,
  keep_editable=FALSE,
  palette_name="rainbow",
  color_brewer= "grDevices"
)

\end{ExampleCode}
\end{Examples}
\inputencoding{utf8}
\HeaderA{WaverideR}{Extracting Signals from Wavelet Spectra}{WaverideR}
%
\begin{Description}
The continuous wavelet transform enables the observation of
transient/non-stationary cyclicity in time-series. The goal of
cyclostratigraphic studies is to define frequency/period in the
depth/time domain. By conducting the continuous wavelet transform
on cyclostratigraphic data series one can observe and extract cyclic
signals/signatures from signals. These results can then be visualized
and interpreted enabling one to identify/interpret cyclicity in the
geological record, which can be used to construct astrochronological
age-models and identify and interpret cyclicity in past and present
climate systems.
\end{Description}
%
\begin{Details}
Package: 'WaverideR'

Type: R package

Version: 0.3.2 (begin of 2023)

License: GPL (= 2)
\end{Details}
%
\begin{Note}
If you want to use this package for publication or research
purposes, please cite:

Arts, M.C.M (2023).
WaverideR: Extracting Signals from Wavelet Spectra.
https://CRAN.R-project.org/package=WaverideR
\end{Note}
%
\begin{Author}
Michiel Arts

Maintainer: Michiel Arts \email{michiel.arts@stratigraphy.eu}
\end{Author}
%
\begin{References}
The 'WaverideR' package builds upon existing literature and existing codebase.
The following list of articles is relevant for the 'WaverideR' R package and
its functions. Individual articles are also cited in the descriptions of
function when relative for set function. The articles in the list below can
be grouped in four subjects: (1) Cyclostratigraphic data analysis, (2)
example data sets, (3) the (continuous) wavelet transform and (4)
astronomical solutions). For each of these categories the
relevance of set articles will be explained in the framework of
the 'WaverideR' R package.

\# 1. Cyclostratigraphic data analysis

Stephen R. Meyers, Cyclostratigraphy and the problem of astrochronologic
testing, Earth-Science Reviews,Volume 190,2019,Pages 190-223,ISSN 0012-8252
\Rhref{https://doi.org/10.1016/j.earscirev.2018.11.015}{doi:10.1016\slash{}j.earscirev.2018.11.015} \\{}
The 'astrochron' R package is the most extensive R package with regards to
cyclostratigraphic analysis. As such many of the functionalities of the
'WaverideR' R package are \#' inspired/based on the 'astrochron' R package.
The major difference between \#' the 'astrochron' R package and the 'WaverideR'
package is that the \#' astrochron' R package relies on the Fast Fourier Transform whereas

S.R. Meyers, 2012, Seeing Red in Cyclic Stratigraphy: Spectral Noise
Estimation for Astrochronology: Paleoceanography, 27, PA3228,
\Rhref{https://doi.org/10.1029/2012PA002307}{doi:10.1029\slash{}2012PA002307} \\{}
The article of Meyers (2012) explains how the (Multitaper method) MTM technique
implemented into The 'astrochron' R package The MTM method can be used
to assign confidence levels to spectral peaks and distinguish spectral peaks
from harmonic spectral peaks.

Acycle: Time-series analysis software for paleoclimate research and education,
Mingsong Li, Linda Hinnov, Lee Kump, Computers \& Geosciences,Volume 127,2019,
Pages 12-22,ISSN 0098-3004, \Rhref{https://doi.org/10.1016/j.cageo.2019.02.011}{doi:10.1016\slash{}j.cageo.2019.02.011} \\{}
The 'Acycle' software package is a 'Matlab' based program, which is used for
cyclostratigraphic studies. Acycle relies mostly on the Fast Fourier Transform.
The 'Coco' and 'eCoco' functions from Acycle formed the inspiration for the
\LinkA{flmw}{flmw} \LinkA{sum\_power\_sedrate}{sum.Rul.power.Rul.sedrate} functions of the
‘Waverider’ R package.

Tracking variable sedimentation rates and astronomical forcing in Phanerozoic
paleoclimate proxy series with evolutionary correlation coefficients and
hypothesis testing, Mingsong Li, Lee R. Kump, Linda A. Hinnov, Michael
E. Mann, Earth and Planetary Science Letters,Volume 501, 2018,Pages 165-179,
ISSN 0012-821X, \Rhref{https://doi.org/10.1016/j.epsl.2018.08.041}{doi:10.1016\slash{}j.epsl.2018.08.041} \\{}
Li et al., (2019) introduces the Coco and eCoco functions of the Acycle
software package. the 'Coco' and 'eCoco' function of the 'Acycle' software
are able to estimate the sedimentation rate based on spectral
characteristics of astronomical cycles. The 'Coco' and 'eCoco' function and
form the inspiration for the  \LinkA{flmw}{flmw} and  \LinkA{sum\_power\_sedrate}{sum.Rul.power.Rul.sedrate}
functions of the 'WaverideR' Package.


Wouters, S., Crucifix, M., Sinnesael, M., Da Silva, A.C., Zeeden, C.,
Zivanovic, M., Boulvain, F., Devleeschouwer, X., 2022,
"A decomposition approach to cyclostratigraphic signal processing".
Earth-Science Reviews 225 (103894).\Rhref{https://doi.org/10.1016/j.earscirev.2021.103894}{doi:10.1016\slash{}j.earscirev.2021.103894} \\{}
Wouters et al., (2022) introduces the Empirical Mode Decomposition (EMD) as
part of the 'DecomposeR' R package. EMD is a non-Fast Fourier Transform
based spectral analysis technique. The Hilbert transform function
\LinkA{inst.pulse}{inst.pulse} of this package is used in WaverideR
functions \LinkA{extract\_amplitude}{extract.Rul.amplitude} and \LinkA{Hilbert\_transform}{Hilbert.Rul.transform}.

Wouters, S., Da Silva, A.-C., Boulvain, F., and Devleeschouwer, X.. 2021.
StratigrapheR: Concepts for Litholog Generation in R. The R Journal.
\Rhref{https://doi.org/10.32614/RJ-2021-039}{doi:10.32614\slash{}RJ\-2021\-039} \\{}
Wouters et al., (2021) introduces the \LinkA{StratigrapheR}{StratigrapheR} R
package. This package contains functions which format, process, and plot lithologs.
The litholog format of Wouters et al., (2021) is used as the standardized input
format to convert lithologs to a time series format using the \LinkA{lithlog\_disc}{lithlog.Rul.disc}
function. The time series can then be analysed for the imprint of cycles.

\#'Huang, Norden E., Zhaohua Wu, Steven R. Long, Kenneth C. Arnold, Xianyao Chen,
and Karin Blank. 2009. "On Instantaneous Frequency". Advances in Adaptive
Data Analysis 01 (02): 177–229. \Rhref{https://doi.org/10.1142/S1793536909000096}{doi:10.1142\slash{}S1793536909000096} \\{}
The Hilbert transform function \LinkA{inst.pulse}{inst.pulse} of the 'DecomposeR'
R package is based on the work of Huang et al., (2009).

Cleveland, W. S. (1979) Robust locally weighted regression and smoothing scatter plots.
Journal of the American Statistical Association. 74, 829–836. \Rhref{https://doi.org/10.1080/01621459.1979.10481038}{doi:10.1080\slash{}01621459.1979.10481038} \\{}
Cleveland (1979) explains how the robust locally weighted regression works
and how it can be used to smooth data sets. This theory is applied in the
\LinkA{loess\_auto}{loess.Rul.auto} function of the ‘WaverideR’ package.

\#'Hurvich, C.M., Simonoff, J.S., and Tsai, C.L. (1998), Smoothing Parameter
Selection in Nonparametric Regression Using an Improved Akaike Information
Criterion. Journal of the Royal Statistical Society B. 60, 271–293
\Rhref{https://doi.org/10.1111/1467-9868.00125}{doi:10.1111\slash{}1467\-9868.00125} \\{}
Hurvich et al., (1998) explains how
the Improved Akaike Information Criterion can be used to optimally smooth data sets
This theory is applied in the \LinkA{loess\_auto}{loess.Rul.auto} function of the ‘WaverideR’ package.

\#'Golub, G., Heath, M. and Wahba, G. (1979). Generalized cross validation as
a method for choosing a good ridge parameter. Technometrics. 21, 215–224. \Rhref{https://doi.org/10.2307/1268518}{doi:10.2307\slash{}1268518}\\{}
Golub et al., (1979) explains how the Generalized cross validation can be
used to optimally smooth data sets. This theory is applied in the \LinkA{loess\_auto}{loess.Rul.auto}
function of the ‘WaverideR’ package.

\# 2. Example data sets

Damien Pas, Linda Hinnov, James E. (Jed) Day, Kenneth Kodama, Matthias Sinnesael, Wei Liu,
Cyclostratigraphic calibration of the Famennian stage (Late Devonian, Illinois Basin, USA),
Earth and Planetary Science Letters,
Volume 488,2018,Pages 102-114,ISSN 0012-821X,
\Rhref{https://doi.org/10.1016/j.epsl.2018.02.010}{doi:10.1016\slash{}j.epsl.2018.02.010} \\{}
The data set of Pas et al, (2018) is a
magnetic susceptibility data measured on the Fammennian aged shales of
the from the Illinois basin in the USA. The data set contains the imprint of
astronomical cycles in the a Paleozoic succession making it a good example for
times (250Ma) when no astronomical solutions are available.


Steinhilber, Friedhelm \& Abreu, Jacksiel \& Beer, Juerg \& Brunner,
Irene \& Christl, Marcus \& Fischer, Hubertus \& Heikkilä, U. \& Kubik,
Peter \& Mann, Mathias \& Mccracken, K. \& Miller, Heinrich \& Miyahara,
Hiroko \& Oerter, Hans \& Wilhelms, Frank. (2012).
9,400 Years of cosmic radiation and solar activity from ice cores and tree rings.
Proceedings of the National Academy of Sciences of the United States of America.
109. 5967-71. 10.1073/pnas.1118965109.
\Rhref{https://doi.org/10.1073/pnas.1118965109}{doi:10.1073\slash{}pnas.1118965109} \\{}
The Total Solar Irradiance record of Steinhilber et al., (2012) is a Holocene record of normalized
Total Solar Irradiance in the time domain. The data set is a good example for
studying/extracting sub-Milankovitch 5000yr from a relatively (geologically) speaking young record.

Christian Zeeden, Frederik Hilgen, Thomas Westerhold, Lucas Lourens, Ursula Röhl, Torsten Bickert,
Revised Miocene splice, astronomical tuning and calcareous plankton biochronology of ODP Site 926 between 5 and 14.4Ma,
Palaeogeography, Palaeoclimatology, Palaeoecology,Volume 369,2013,Pages 430-451,ISSN 0031-0182,
10.1016/j.palaeo.2012.11.009\\{}
The record of Zeeden et al., (2013) consists of a grey scale record from Miocene sediment cores from offshore
Brazil. The record contains a clear imprint of astronomical cycles as such it is a good Neogene example data set
to demonstrate the functionalities of the 'WaverideR' R package

\# 3. The (continuous) wavelet transform

Morlet, Jean, Georges Arens, Eliane Fourgeau, and Dominique Glard.
"Wave propagation and sampling theory—Part I: Complex signal and scattering in multilayered media.
" Geophysics 47, no. 2 (1982): 203-221.
\url{https://pubs.geoscienceworld.org/geophysics/article/47/2/203/68601/Wave-propagation-and-sampling-theory-Part-I}\\{}
Morlet et al., (1982a) together with Morlet et al., (1982b) are the original publications which explain the
use of the wavelet to analyse signal.

J. Morlet, G. Arens, E. Fourgeau, D. Giard;
Wave propagation and sampling theory; Part II, Sampling theory and complex waves.
Geophysics 1982 47 (2): 222–236. \url{https://pubs.geoscienceworld.org/geophysics/article/47/2/222/68604/Wave-propagation-and-sampling-theory-Part-II}\\{}
Morlet et al., (1982a) together with Morlet et al., (1982b) are the original publications which explain the
use of the wavelet to analyse signal.

Torrence, C., and G. P. Compo. 1998. A Practical Guide to Wavelet Analysis.
Bulletin of the American Meteorological Society 79:61-78.
\url{https://paos.colorado.edu/research/wavelets/bams_79_01_0061.pdf} \\{}
'Torrence and Compo (1998) shows how the continuous wavelet transform can be used to analyse
cyclicity in paleo-climatic data-sets. The equations in this publication forms the basis for many
wavelet based packages/software applications.

Gouhier TC, Grinsted A, Simko V (2021).
R package biwavelet: Conduct Univariate and Bivariate Wavelet Analyses. (Version 0.20.21),
\url{https://github.com/tgouhier/biwavelet}\\{}
Gouhier et al., (2021) is the implementation of equations of Torrence and Compo (1998) in the form of the
'biwavelet' R package

Angi Roesch and Harald Schmidbauer (2018). WaveletComp: Computational
Wavelet Analysis. R package version 1.1.
\url{https://CRAN.R-project.org/package=WaveletComp}\\{}
Roesch and Schmidbauer et al., (2018) is the article of the 'WaveletComp' R package which
is a built upon the functionalities of the 'biwavelet' R package

Russell, Brian, and Jiajun Han. "Jean Morlet and the continuous wavelet transform.
" CREWES Res. Rep 28 (2016): 115. \url{https://www.crewes.org/Documents/ResearchReports/2016/CRR201668.pdf}\\{}
Russell and Han (2016) gives a concise summary of the work of Morlet et al., (1982a) and Morlet et al., (1982b) and the
developments since then. The publication also describes how the Gabor uncertainty principle (Gabor 1946) affects the frequency
uncertainty of the wavelet which can be used to calculate the analytical uncertainty of a given wavelet spectra.

Gabor, Dennis. "Theory of communication. Part 1: The analysis of information."
Journal of the Institution of Electrical Engineers-part III: radio and
communication engineering 93, no. 26 (1946): 429-441. \url{http://genesis.eecg.toronto.edu/gabor1946.pdf}\\{}
Gabor (1946) describes the Gabor uncertainty principle which states how the uncertainty in time and frequency are related
in time series analysis.

\# 4. Astronomical solutions

J. Laskar, P. Robutel, F. Joutel, M. Gastineau, A.C.M. Correia, and B. Levrard, B., 2004,
A long term numerical solution for the insolation quantities of the Earth: Astron. Astrophys.,
Volume 428, 261-285. \Rhref{https://doi.org/10.1051/0004-6361\%3A20041335}{doi:10.1051\slash{}0004\-6361:20041335}\\{}
Laskar et al., (2004) is an astronomical solution which can be used to anchor geological data to absolute ages.

Laskar, J., Fienga, A., Gastineau, M., Manche, H., 2011a,
La2010: A new orbital solution for the long-term motion of the Earth: Astron. Astrophys.,
Volume 532, A89 \Rhref{https://doi.org/10.1051/0004-6361/201116836}{doi:10.1051\slash{}0004\-6361\slash{}201116836}\\{}
Laskar et al., (2011a) is an astronomical solution which can be used to anchor geological data to absolute ages.

Laskar, J., Gastineau, M., Delisle, J.-B., Farres, A., Fienga, A.:
2011b, Strong chaos induced by close encounters with Ceres and Vesta, Astron: Astrophys.,
Volume 532, L4.  \Rhref{https://doi.org/10.1051/0004-6361/201117504}{doi:10.1051\slash{}0004\-6361\slash{}201117504}\\{}
Laskar et al., (2011b) is an astronomical solution which can be used to anchor geological data to absolute ages.

J. Laskar,Chapter 4 - Astrochronology,Editor(s): Felix M. Gradstein, James G. Ogg, Mark D. Schmitz, Gabi M. Ogg,Geologic Time Scale 2020,Elsevier,2020,Pages 139-158,ISBN 9780128243602,
'\Rhref{https://doi.org/10.1016/B978-0-12-824360-2.00004-8}{doi:10.1016\slash{}B978\-0\-12\-824360\-2.00004\-8}\\{}
Laskar et al., (2019) explains how astronomical solutions are created and how they should/can be used

Zeebe, Richard E. "Numerical solutions for the orbital motion of the Solar System over the past 100 Myr: limits and new results."
The Astronomical Journal 154, no. 5 (2017): 193. \Rhref{https://doi.org/10.3847/1538-3881/aa8cce}{doi:10.3847\slash{}1538\-3881\slash{}aa8cce} \\{}
Zeebe (2017) is an astronomical solution which can be used to anchor geological data to absolute ages.

Richard E. Zeebe Lucas J. Lourens ,Solar System chaos and the Paleocene–Eocene boundary age constrained by geology and astronomy.Science365,926-929(2019)
\Rhref{https://doi.org/10.1126/science.aax0612}{doi:10.1126\slash{}science.aax0612}\\{}
Zeebe and Lourens (2019) is an astronomical solution which can be used to anchor geological data to absolute ages.

Zeebe, R. E. and Lourens, L. J.
Geologically constrained astronomical solutions for the Cenozoic era,
Earth and Planetary Science Letters, 2022 \Rhref{https://doi.org/10.1016/j.epsl.2022.117595}{doi:10.1016\slash{}j.epsl.2022.117595} \\{}
Zeebe and Lourens (2022) is an astronomical solution which can be used to anchor geological data to absolute ages.
\end{References}
\inputencoding{utf8}
\HeaderA{WaverideR\_Datasets}{Example data sets for the 'WaverideR' package}{WaverideR.Rul.Datasets}
%
\begin{Description}
Data sets for testing the 'WaverideR' R package:\\{}
The \code{age\_model\_zeeden} data set is and age model (anchor points) for\\{}
the IODP 926 grey scale (154-174m) record of Zeeden et al., (2013)\\{}
\\{}
The \code{astrosignal\_example} data set consists of pre-generated ETP (eccentricity-tilt-precession)\\{}
data set based on the p-0.5t  la2004 solution and was generated using \\{}
the \LinkA{etp}{etp} function of the 'astrochron' R package \\{}
\\{}
The \code{depth\_rank\_example} data set is synthetic succession of sedimentary\\{}
The \code{grey} data set is the grey scale record of IODP 926 for the interval (154-174m) which originates\\{} from
Zeeden et al., (2013) \\{}
\\{}
The \code{grey\_track} data set consists of tracking points of the\\{}
precession (22 kyr cycle) in the IODP 926 grey scale (154-174m) record of Zeeden et al., (2013)\\{}
\\{}
The \code{mag} data set is the magnetic susceptibility record of Pas et al., (2018)\\{}
\\{}
The \code{mag\_track\_solution} is the period of the 405 kyr eccentricity cycle in\\{}
the magnetic susceptibility record of from Pas et al., (2018)\\{}
\\{}
The \code{TSI} data set is the Total Solar Irradiance record of Steinhilber et al., (2012)\\{}
\\{}
The \code{Bisciaro\_Mg\_wt\_track} data set is the 110-kyr (short eccentricity) \\{}
cycle tracked in the wavelet scalogram of the Magnesium (XRF) record of Arts (2014)\\{}
\\{}
The \code{Bisciaro\_Mn\_wt\_track} data set is the 110-kyr (short eccentricity) \\{}
cycle tracked in the wavelet scalogram of the Manganese (XRF)record of Arts (2014)\\{}
\\{}
The \code{Bisciaro\_al\_wt\_track} data set is the 110-kyr (short eccentricity) \\{}
cycle tracked in the wavelet scalogram of the Aluminum (XRF) record of Arts (2014)\\{}
\\{}
The \code{Bisciaro\_ca\_wt\_track} data set is the 110-kyr (short eccentricity) \\{}
cycle tracked in the wavelet scalogram of the Calcium (XRF) record of Arts (2014)\\{}
\\{}
The \code{Bisciaro\_sial\_wt\_track} data set is the 110-kyr (short eccentricity) \\{}
cycle tracked in the wavelet scalogram of the Silicon/Aluminum (XRF) record of Arts (2014)\\{}
\\{}
The \code{Bisciaro\_XRF} is the XRF data set of Arts (2014)\\{}
\\{}

The \code{anchor\_points\_Bisciaro\_al} data set consist of the tie points between the
Bisciaro\_al record of Arts (2014) and the la2011 solution of laskar et al., (20111)\\{}
\\{}

The \code{GTS\_info} data set contains the color coding and ages and uncertainties
of Geologic Time Scale 2020 of Ogg (et al., 2021)\\{}
\\{}
\end{Description}
%
\begin{References}
Damien Pas, Linda Hinnov, James E. (Jed) Day, Kenneth Kodama, Matthias Sinnesael, Wei Liu,
Cyclostratigraphic calibration of the Famennian stage (Late Devonian, Illinois Basin, USA),
Earth and Planetary Science Letters,
Volume 488,2018,Pages 102-114,ISSN 0012-821X,
<doi:10.1016/j.epsl.2018.02.010>

Steinhilber, Friedhelm \& Abreu, Jacksiel \& Beer, Juerg \& Brunner,
Irene \& Christl, Marcus \& Fischer, Hubertus \& Heikkilä, U. \& Kubik,
Peter \& Mann, Mathias \& Mccracken, K. \& Miller, Heinrich \& Miyahara,
Hiroko \& Oerter, Hans \& Wilhelms, Frank. (2012).
9,400 Years of cosmic radiation and solar activity from ice cores and tree rings.
Proceedings of the National Academy of Sciences of the United States of America.
109. 5967-71. 10.1073/pnas.1118965109.
<doi:10.1073/pnas.1118965109>

Christian Zeeden, Frederik Hilgen, Thomas Westerhold, Lucas Lourens, Ursula Röhl, Torsten Bickert,
Revised Miocene splice, astronomical tuning and calcareous plankton biochronology of ODP Site 926 between 5 and 14.4Ma,
Palaeogeography, Palaeoclimatology, Palaeoecology,Volume 369,2013,Pages 430-451,ISSN 0031-0182,
<doi:10.1016/j.palaeo.2012.11.009>

Stephen R. Meyers,Cyclostratigraphy and the problem of astrochronologic testing,
Earth-Science Reviews,Volume 190,2019,Pages 190-223,ISSN 0012-8252
<doi:10.1016/j.earscirev.2018.11.015>

J. Laskar, P. Robutel, F. Joutel, M. Gastineau, A.C.M. Correia, and B. Levrard, B., 2004,
A long term numerical solution for the insolation quantities of the Earth: Astron. Astrophys.,
Volume 428, 261-285. <doi:10.1051/0004-6361:20041335> \\{}

Laskar, J., M. Gastineau, J. B. Delisle, A. Farrés, and A. Fienga (2011b),
Strong chaos induced by close encounters with Ceres and Vesta, Astron. Astrophys.,
532, L4,<doi:10.1051/0004-6361/201117504> \\{}

M.C.M. Arts, 2014,
Magnetostratigrpahy and geochemical analysis of the early Miocene Bisciaro Formation
in the Contessa Valley (Northern Italy). Unpublished Bsc. thesis \\{}

Ogg, Gabi \& Ogg, James \& Gradstein, Felix. (2021).
Recommended color coding of stages - Appendix 1
from Geologic Time Scale 2020.
\end{References}
\inputencoding{utf8}
\HeaderA{win\_fft}{Windowed fft based spectral analysis}{win.Rul.fft}
%
\begin{Description}
The \code{\LinkA{win\_fft}{win.Rul.fft}} function for conducts a windowed spectral analysis based on the fft
\end{Description}
%
\begin{Usage}
\begin{verbatim}
win_fft(
  data = NULL,
  padfac = 5,
  window_size = NULL,
  run_multicore = FALSE,
  genplot = FALSE,
  x_lab = c("depth (m)"),
  y_lab = c("frequency cycle/metre"),
  plot_res = 1,
  perc_vis = 0,
  freq_max = NULL,
  freq_min = NULL,
  palette_name = "rainbow",
  color_brewer = "grDevices",
  keep_editable = FALSE,
  verbose = FALSE,
  dev_new = FALSE
)
\end{verbatim}
\end{Usage}
%
\begin{Arguments}
\begin{ldescription}
\item[\code{data}] Input data set  should consist of a matrix with 2 columns with first column being depth and the second column being a proxy

\item[\code{padfac}] Pad record with zero, zero padding smooths out the spectra

\item[\code{window\_size}] size of the running window

\item[\code{run\_multicore}] Run function using multiple cores \code{Default="FALSE"}

\item[\code{genplot}] Generate plot \code{Default="FALSE"}

\item[\code{x\_lab}] label for the y-axis \code{Default="depth"}

\item[\code{y\_lab}] label for the y-axis \code{Default="sedrate"}

\item[\code{plot\_res}] plot 1 of 8 options option 1: Amplitude matrix,
option 2: Power matrix,
option 3: Phase matrix,
option 4: AR1\_CL matrix,
option 5: AR1\_Fit matrix ,
option 6: AR1\_90\_power matrix,
option 7: AR1\_95\_power matrix,
option 8: AR1\_99\_power matrix, \code{Default=1}

\item[\code{perc\_vis}] Cutoff percentile when plotting \code{Default=0}

\item[\code{freq\_max}] Maximum frequency to plot

\item[\code{freq\_min}] Minimum frequency to plot

\item[\code{palette\_name}] Name of the color palette which is used for plotting.
The color palettes than can be chosen depends on which the R package is specified in
the color\_brewer parameter. The included R packages from which palettes can be chosen
from are; the 'RColorBrewer', 'grDevices', 'ColorRamps' and 'Viridis' R packages.
There are many options to choose from so please
read the documentation of these packages \code{Default=rainbow}.
The R package 'viridis' has the color palette options: “magma”, “plasma”,
“inferno”, “viridis”, “mako”, and “rocket”  and “turbo”
To see the color palette options of the The R pacakge 'RColorBrewer' run
the RColorBrewer::brewer.pal.info() function
The R package 'colorRamps' has the color palette options:"blue2green",
"blue2green2red", "blue2red",    "blue2yellow", "colorRamps",    "cyan2yellow",
"green2red", "magenta2green", "matlab.like", "matlab.like2" and    "ygobb"
The R package 'grDevices' has the built in  palette options:"rainbow",
"heat.colors", "terrain.colors","topo.colors" and "cm.colors"
To see even more color palette options of the The R pacakge 'grDevices' run
the grDevices::hcl.pals() function

\item[\code{color\_brewer}] Name of the R package from which the color palette is chosen from.
The included R packages from which palettes can be chosen
are; the RColorBrewer, grDevices, ColorRamps and Viridis R packages.
There are many options to choose from so please
read the documentation of these packages. "\code{Default=grDevices}

\item[\code{keep\_editable}] Keep option to add extra features after plotting  \code{Default=FALSE}

\item[\code{verbose}] Print text \code{Default=FALSE}.

\item[\code{dev\_new}] Opens a new plotting window to plot the plot, this guarantees  a "nice" looking plot however when plotting in an R markdown
document the plot might not plot  \code{Default=FALSE}
\end{ldescription}
\end{Arguments}
%
\begin{Value}
Returns a list which contains 10 elements
element 1: Amplitude matrix
element 2: Power matrix
element 3: Phase matrix
element 4: AR1\_CL matrix
element 5: AR1\_Fit matrix
element 6: AR1\_90\_power matrix
element 7: AR1\_95\_power matrix
element 8: AR1\_99\_power matrix
element 9: depth
element 10: y\_axis
If genplot is \code{Default=TRUE} then a plot of one of the elements 1:8 is plotted
\end{Value}
%
\begin{Author}
Based on the \LinkA{periodogram}{periodogram}
function of the 'astrochron' R package.
\end{Author}
%
\begin{References}
Routines for astrochronologic testing, astronomical time scale construction, and
time series analysis <doi:10.1016/j.earscirev.2018.11.015>
\end{References}
%
\begin{Examples}
\begin{ExampleCode}

#Conduct a windowed ftt on the magnetic susceptibility record
#of the Sullivan core of Pas et al., (2018).

mag_win_fft <- win_fft(data= mag,
                   padfac = 5,
                   window_size = 12.5,
                   run_multicore = FALSE,
                   genplot = FALSE,
                   x_lab = c("depth (m)"),
                   y_lab = c("frequency cycle/metre"),
                   plot_res = 1,
                   perc_vis = 0.5,
                   freq_max = 5,
                   freq_min = 0.001,
                   palette_name ="rainbow",
                   color_brewer= "grDevices",
                   keep_editable=FALSE,
                   verbose=FALSE,
                   dev_new=FALSE)


\end{ExampleCode}
\end{Examples}
\printindex{}
\end{document}
